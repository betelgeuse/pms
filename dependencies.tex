\chapter{Dependencies}
\label{dependencies}

\section{Dependency Classes}

There are three classes of dependencies supported by ebuilds:

\begin{bulletlist}
\item Build dependencies (\t{DEPEND}). These must be installed before the ebuild is installed.
\item Runtime dependencies (\t{RDEPEND}). These should usually be installed before the ebuild,
    but may be dropped to post dependencies where necessary to resolve cycles.
\item Post dependencies (\t{PDEPEND}). These should be installed at some point, usually after
    the ebuild if they are not already installed.
\end{bulletlist}

In addition, \t{SRC\_URI}, \t{PROVIDE} and \t{LICENSE} use the dependency specification format
to specify their values.

\note The term 'dependency specification' is perhaps not the best name, but it is the name
    most easily recognised by developers

\section{Dependency Specification Format}

The dependency specification format is a string containing zero or more of the following
items separated by whitespace:

\begin{bulletlist}
\item A package dependency specification (for \t{DEPEND} etc.), or a URI (for \t{SRC\_URI}),
    or a package name (for \t{PROVIDE}), or a license name (for \t{LICENSE}).
\item An all-of group, which consists of an open parenthesis, followed by whitespace,
    followed by zero or more dependency items of any kind, followed by whitespace, followed
    by a close parenthesis.
\item An any-of group, which consists of the string \t{||}, followed by whitespace,
    followed by an open parenthesis, followed by zero or more dependency items of any kind,
    followed by whitespace, followed by a close parenthesis.
\item A use-conditional group, which consists of an optional exclamation mark, followed by
    a use flag name, followed by a question mark, followed by whitespace, followed by
    an open parenthesis, followed by zero or more dependency items of any kind, followed by
    whitespace, followed by a close parenthesis.
\end{bulletlist}

In particular, note that whitespace is not optional.

\subsection{Package Dependency Specifications}

In EAPI-0, a package dependency can be in one of the following base formats:

\begin{bulletlist}
\item A simple \t{category/package} name.
\item An operator, followed immediately by \t{category/package}, followed by a hyphen,
   followed by a version specification.
\end{bulletlist}

\todiscuss Should slot dependencies be included in EAPI-0?

The following operators are available:

\begin{description}
\item[\t{<}] Strictly less than the specified version.
\item[\t{<=}] Less than or equal to the specified version.
\item[\t{=}] Exactly equal to the specified version. Special exception: if the version
    specified has an asterisk immediately following it, a string prefix comparison is
    used instead. (An asterisk used with any other operator is illegal.)
\item[\t{\~}] Equal to the specified version, except the revision part of the matching
    package may be greater than the revision part of the specified version (\t{-r0} is
    assumed if no revision is explicitly stated).
\item[\t{>=}] Greater than or equal to the specified version.
\item[\t{>}] Strictly greater than the specified version.
\end{description}

If the operator is prefixed with an exclamation mark, the named dependency is a block
rather than a requirement -- that is to say, the specified package must not be
installed, except with the following exceptions:

\begin{bulletlist}
\item Blocks on a package provided by the ebuild do not count.
\item Blocks on the ebuild itself do not count.
\end{bulletlist}

\subsection{All-of Dependency Specifications}

In an all-of group, all of the child elements must be matched.

\subsection{Use-conditional Dependency Specifications}

In a use-conditional group, if the associated use flag is enabled (or disabled if it has an
exclamation mark prefix), all of the child elements must be matched.

\subsection{Any-of Dependency Specifications}

Any-of dependency specifications only make sense for dependencies.

Any use-conditional group that is an immediate child of an any-of group, if not enabled (disabled
for an exclamation mark prefixed use flag name), is not considered a member of the any-of group
for dependency resolution purposes.

In an any-of group, at least one immediate child element must be matched. A blocker is
considered to be matched if it is not installed.

An empty any-of group counts as being installed.

% vim: set filetype=tex fileencoding=utf8 et tw=100 spell spelllang=en :

