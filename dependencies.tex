\chapter{Dependencies}
\label{sec:dependencies}

\section{Dependency Classes}

There are three classes of dependencies supported by ebuilds:

\begin{compactitem}
\item Build dependencies (\t{DEPEND}). These must be installed and usable before the ebuild
    is installed.
\item Runtime dependencies (\t{RDEPEND}). These must be installed and usable before the ebuild
    is treated as usable.
\item Post dependencies (\t{PDEPEND}). These must be installed at some point.
\end{compactitem}

In addition, \t{SRC\_URI}, \t{HOMEPAGE}, \t{PROVIDE}, \t{RESTRICT}, \t{PROPERTIES} and \t{LICENSE}
use dependency-style specifications to specify their values.

\section{Dependency Specification Format}

The following elements are recognised in at least one class of specification. All elements must
be surrounded on both sides by whitespace, except at the start and end of the string.

\begin{compactitem}
\item A package dependency specification. Permitted in \t{DEPEND}, \t{RDEPEND}, \t{PDEPEND}.
\item A simple qualified package name. Permitted in \t{PROVIDE} (and inside \t{DEPEND} etc
    via the previous item).
\item A URI, in the form \t{proto://host/path}. Permitted in \t{SRC\_URI} and \t{HOMEPAGE}. In EAPIs
    listed in table~\ref{tab:uri-arrows-table} as supporting \t{SRC\_URI} arrows, may optionally be
    followed by whitespace, then \t{->}, then whitespace, then a simple filename when in
    \t{SRC\_URI}.  For \t{SRC\_URI} behaviour, see section~\ref{sec:src-uri-behaviour}.
\item A flat filename. Permitted in \t{SRC\_URI}.
\item A license name (e.g. \t{GPL-2}). Permitted in \t{LICENSE}.
\item A simple string. Permitted in \t{RESTRICT} and \t{PROPERTIES}.
\item An all-of group, which consists of an open parenthesis, followed by whitespace,
    followed by zero or more of (a dependency item of any kind followed by whitespace), followed
    by a close parenthesis. More formally:
    \t{all-of ::= '(' whitespace (item whitespace)* ')'}. Permitted in all specification style
    variables.
\item An any-of group, which consists of the string \t{||}, followed by whitespace,
    followed by an open parenthesis, followed by whitespace, followed by zero or more
    of (a dependency item of any kind followed by whitespace), followed by a close parenthesis.
    More formally: \t{any-of ::= '||' whitespace '(' whitespace (item whitespace)* ')'}.
    Permitted in \t{DEPEND}, \t{RDEPEND}, \t{PDEPEND}, \t{LICENSE}.
\item A use-conditional group, which consists of an optional exclamation mark, followed by
    a use flag name, followed by a question mark, followed by whitespace, followed by
    an open parenthesis, followed by whitespace, followed by zero or more of (a dependency item
    of any kind followed by whitespace), followed by a close parenthesis. More formally:
    \t{use-conditional ::= '!'? flag-name '?' whitespace '(' whitespace (item whitespace)* ')'}.
    Permitted in all specification style variables.
\IFKDEBUILDELSE
{
    \item A label, which is a string without whitespace followed by a colon. Permitted in
    \t{SRC\_URI} in EAPIs shown in table~\ref{tab:uri-labels-table} as supporting \t{SRC\_URI} labels,
    and in \t{PDEPEND} in EAPIs shown in table~\ref{tab:pdepend-labels-table} as supporting \t{PDEPEND}
    labels.
}{
}
\end{compactitem}

In particular, note that whitespace is not optional.

\begin{centertable}{EAPIs supporting \t{SRC\_URI} arrows} \label{tab:uri-arrows-table}
\IFKDEBUILDELSE
{
    \begin{tabular}{ l l }
        \toprule
        \multicolumn{1}{c}{\textbf{EAPI}} &
        \multicolumn{1}{c}{\textbf{Supports \t{SRC\_URI} arrows?}} \\
        \midrule
    \t{0} & No \\
    \t{1} & No \\
    \t{kdebuild-1} & Yes \\
    \t{2} & Yes \\
    \t{3} & Yes \\
    \t{4} & Yes \\
    \bottomrule
    \end{tabular}
}{
    \begin{tabular}{ l l }
        \toprule
        \multicolumn{1}{c}{\textbf{EAPI}} &
        \multicolumn{1}{c}{\textbf{Supports \t{SRC\_URI} arrows?}} \\
        \midrule
    \t{0} & No \\
    \t{1} & No \\
    \t{2} & Yes \\
    \t{3} & Yes \\
    \t{4} & Yes \\
    \bottomrule
    \end{tabular}
}
\end{centertable}

\IFKDEBUILDELSE
{
    \begin{centertable}{EAPIs supporting \t{SRC\_URI} labels} \label{tab:uri-labels-table}
    \begin{tabular}{ l l }
        \toprule
        \multicolumn{1}{c}{\textbf{EAPI}} &
        \multicolumn{1}{c}{\textbf{Supports \t{SRC\_URI} labels?}} \\
        \midrule
    \t{0} & No \\
    \t{1} & No \\
    \t{kdebuild-1} & Yes \\
    \t{2} & No \\
    \t{3} & No \\
    \t{4} & No \\
    \bottomrule
    \end{tabular}
    \end{centertable}

    \begin{centertable}{EAPIs supporting \t{PDEPEND} labels} \label{tab:pdepend-labels-table}
    \begin{tabular}{ l l }
        \toprule
        \multicolumn{1}{c}{\textbf{EAPI}} &
        \multicolumn{1}{c}{\textbf{Supports \t{PDEPEND} labels?}} \\
        \midrule
    \t{0} & No \\
    \t{1} & No \\
    \t{kdebuild-1} & Yes \\
    \t{2} & No \\
    \t{3} & No \\
    \t{4} & No \\
    \bottomrule
    \end{tabular}
    \end{centertable}
}{
}

\subsection{All-of Dependency Specifications}

In an all-of group, all of the child elements must be matched.

\subsection{Use-conditional Dependency Specifications}

In a use-conditional group, if the associated use flag is enabled (or disabled if it has an
exclamation mark prefix), all of the child elements must be matched.

It is an error for a flag to be used if it is not included in \t{IUSE\_EFFECTIVE} as described in
section~\ref{sec:use-iuse-handling}.

\subsection{Any-of Dependency Specifications}

Any use-conditional group that is an immediate child of an any-of group, if not enabled (disabled
for an exclamation mark prefixed use flag name), is not considered a member of the any-of group
for match purposes.

In an any-of group, at least one immediate child element must be matched. A blocker is
considered to be matched if its associated package dependency specification is not matched.

An empty any-of group counts as being matched.

\IFKDEBUILDELSE
{
    \subsection{Labels}
    \label{sec:labels}

    A label alters the behaviour of any subsequent items (including those inside subgroups) in the
    current group. This is demonstrated by code listing~\ref{lst:labels-listing}.

\begin{listing}
  \caption{How labels are applied}\label{lst:labels-listing}
  \verbatiminput{labels.listing}
\end{listing}
}{
}

\subsection{Package Dependency Specifications}

A package dependency can be in one of the following base formats. A package manager must warn or
error on non-compliant input.

\begin{compactitem}
\item A simple \t{category/package} name.
\item An operator, as described in section~\ref{sec:dep-operator}, followed immediately by
    \t{category/package}, followed by a hyphen, followed by a version specification.
\end{compactitem}

In EAPIs shown in table~\ref{tab:slot-deps-table} as supporting \t{SLOT} dependencies, either of the
above formats may additionally be suffixed by a \t{:slot} restriction, as described in
section~\ref{sec:slot-dep}. A package manager must warn or error if slot dependencies are used with an
EAPI not supporting \t{SLOT} dependencies.

\IFKDEBUILDELSE
{
    In EAPIs shown in table~\ref{tab:range-deps-table} as supporting ranged dependencies, a
    specification that does not use an operator at the start may additionally be suffixed by one
    \t{[range]} restriction, as described in section~\ref{sec:range-dep}. A package manager must warn or
    error if this feature is used with an EAPI not supporting ranged dependencies.
}{
}

\IFKDEBUILDELSE
{
    \featurelabel{use-deps} In EAPIs shown in table~\ref{tab:use-deps-table} as supporting
    kdebuild-style \t{USE} dependencies, a specification may additionally be suffixed by one or more
    kdebuild-style \t{[use]} restrictions, as described in section~\ref{sec:kdebuild-use-dep}. A
    package manager must warn or error if this feature is used with an EAPI not supporting use
    dependencies.

    In EAPIs shown in table~\ref{tab:use-deps-table} as supporting 2-style or 4-style \t{USE}
    dependencies, a specification may additionally be suffixed by at most one 2-style or 4-style
    \t{[use]} restriction, as described in section~\ref{sec:use-dep}. A package manager must warn or
    error if this feature is used with an EAPI not supporting use dependencies.
}{
    \featurelabel{use-deps} In EAPIs shown in table~\ref{tab:use-deps-table} as supporting 2-style
    or 4-style \t{USE} dependencies, a specification may additionally be suffixed by at most one
    2-style or 4-style \t{[use]} restriction, as described in section~\ref{sec:use-dep}. A package
    manager must warn or error if this feature is used with an EAPI not supporting use dependencies.
}

\IFKDEBUILDELSE
{
    \note Order is important. The slot restriction must come before the range restriction, which
    must come before use dependencies.
}{
    \note Order is important. The slot restriction must come before use dependencies.
}

\begin{centertable}{EAPIs supporting \t{SLOT} dependencies} \label{tab:slot-deps-table}
\IFKDEBUILDELSE
{
    \begin{tabular}{ l l }
        \toprule
            \multicolumn{1}{c}{\textbf{EAPI}} &
            \multicolumn{1}{c}{\textbf{Supports \t{SLOT} dependencies?}} \\
            \midrule
    \t{0} & No \\
    \t{1} & Named only \\
    \t{kdebuild-1} & Named and operator \\
    \t{2} & Named only \\
    \t{3} & Named only \\
    \t{4} & Named and operator \\
    \bottomrule
    \end{tabular}
}{
    \begin{tabular}{ l l }
        \toprule
            \multicolumn{1}{c}{\textbf{EAPI}} &
            \multicolumn{1}{c}{\textbf{Supports \t{SLOT} dependencies?}} \\
            \midrule
    \t{0} & No \\
    \t{1} & Named only \\
    \t{2} & Named only \\
    \t{3} & Named only \\
    \t{4} & Named and operator \\
    \bottomrule
    \end{tabular}
}
\end{centertable}

\IFKDEBUILDELSE
{
    \begin{centertable}{EAPIs supporting ranged dependencies} \label{tab:range-deps-table}
    \begin{tabular}{ l l }
        \toprule
        \multicolumn{1}{c}{\textbf{EAPI}} &
        \multicolumn{1}{c}{\textbf{Supports ranged dependencies?}} \\
        \midrule
    \t{0} & No \\
    \t{1} & No \\
    \t{kdebuild-1} & Yes \\
    \t{2} & No \\
    \t{3} & No \\
    \t{4} & No \\
    \bottomrule
    \end{tabular}
    \end{centertable}
}{
}

\IFKDEBUILDELSE
{
    \begin{centertable}{EAPIs supporting \t{USE} dependencies} \label{tab:use-deps-table}
    \begin{tabular}{ l l }
        \toprule
        \multicolumn{1}{c}{\textbf{EAPI}} &
        \multicolumn{1}{c}{\textbf{Supports \t{USE} dependencies?}} \\
        \midrule
    \t{0} & No \\
    \t{1} & No \\
    \t{kdebuild-1} & kdebuild-style \\
    \t{2} & 2-style \\
    \t{3} & 2-style \\
    \t{4} & 4-style \\
    \bottomrule
    \end{tabular}
    \end{centertable}
}{
    \begin{centertable}{EAPIs supporting \t{USE} dependencies} \label{tab:use-deps-table}
    \begin{tabular}{ l l }
        \toprule
        \multicolumn{1}{c}{\textbf{EAPI}} &
        \multicolumn{1}{c}{\textbf{Supports \t{USE} dependencies?}} \\
        \midrule
    \t{0} & No \\
    \t{1} & No \\
    \t{2} & 2-style \\
    \t{3} & 2-style \\
    \t{4} & 4-style \\
    \bottomrule
    \end{tabular}
    \end{centertable}
}

\subsubsection{Operators}
\label{sec:dep-operator}

The following operators are available:

\begin{description}
\item[\t{<}] Strictly less than the specified version.
\item[\t{<=}] Less than or equal to the specified version.
\item[\t{=}] Exactly equal to the specified version. Special exception: if the version
    specified has an asterisk immediately following it, a string prefix comparison is
    used instead. When an asterisk is used, the specification must remain valid if the
    asterisk were removed. (An asterisk used with any other operator is illegal.)
\item[\t{\textasciitilde}] Equal to the specified version, except the revision part of the matching
    package may be greater than the revision part of the specified version (\t{-r0} is
    assumed if no revision is explicitly stated).
\item[\t{>=}] Greater than or equal to the specified version.
\item[\t{>}] Strictly greater than the specified version.
\end{description}

\subsubsection{Block Operator}

If the specification is prefixed with one or two exclamation marks, the named dependency is a block
rather than a requirement---that is to say, the specified package must not be installed, with the
following exceptions:

\begin{compactitem}
\item Blocks on a package provided exclusively by the ebuild do not count. \label{provided-blocks}
\item Blocks on the ebuild itself do not count.
\end{compactitem}

\featurelabel{bang-strength} There are two strengths of block: weak and strong. A weak block may be
ignored by the package manager, so long as any blocked package will be uninstalled later on. A
strong block must not be ignored. The mapping from one or two exclamation marks to strength is
described in table~\ref{tab:bang-strength-table}.

\begin{centertable}{Exclamation mark strengths for EAPIs} \label{tab:bang-strength-table}
\IFKDEBUILDELSE
{
    \begin{tabular}{ l l l }
        \toprule
            \multicolumn{1}{c}{\textbf{EAPI}} &
            \multicolumn{1}{c}{\textbf{\t{!}}} &
            \multicolumn{1}{c}{\textbf{\t{!!}}} \\
            \midrule
    \t{0} & Unspecified & Forbidden \\
    \t{1} & Unspecified & Forbidden \\
    \t{kdebuild-1} & Unspecified & Forbidden \\
    \t{2} & Weak & Strong \\
    \t{3} & Weak & Strong \\
    \t{4} & Weak & Strong \\
    \bottomrule
    \end{tabular}
}{
    \begin{tabular}{ l l l }
        \toprule
            \multicolumn{1}{c}{\textbf{EAPI}} &
            \multicolumn{1}{c}{\textbf{\t{!}}} &
            \multicolumn{1}{c}{\textbf{\t{!!}}} \\
            \midrule
    \t{0} & Unspecified & Forbidden \\
    \t{1} & Unspecified & Forbidden \\
    \t{2} & Weak & Strong \\
    \t{3} & Weak & Strong \\
    \t{4} & Weak & Strong \\
    \bottomrule
    \end{tabular}
}
\end{centertable}

\subsubsection{Slot Dependencies}
\label{sec:slot-dep}

\featurelabel{slot-deps} A named slot dependency consists of a colon followed by a slot name. A
specification with a named slot dependency matches only if the slot of the matched package is equal
to the slot specified. If the slot of the package to match cannot be determined (e.g. because it is
not a supported \t{EAPI}), the match is treated as unsuccessful.

\featurelabel{slot-operator-deps} An operator slot dependency consists of a colon followed by one of
the following operators:

\begin{description}
\item[*] Indicates that any slot value is acceptable. In addition, for runtime dependencies,
indicates that the package will not break if the matched package is uninstalled and replaced by
a different matching package in a different slot.
\item[=] Indicates that any slot value is acceptable. In addition, for runtime dependencies,
indicates that the package will break unless a matching package with slot equal to the slot of
the best installed version at the time the package was installed is available.
\end{description}

To implement the equals slot operator, the package manager will need to store the slot of the
best installed version of the matching package. The package manager may do this by appending
the appropriate slot after the equals sign when saving the package's dependencies. This syntax
is only for package manager use and must not be used by ebuilds.

\IFKDEBUILDELSE
{
    \subsubsection{Ranged Dependencies}
    \label{sec:range-dep}

    \featurelabel{ranged-deps} A ranged dependency consists of an open square bracket, followed by
    zero or more of (a depend operator followed by a version spec followed by a logical operator)
    followed by a depend operator followed by a version spec followed by a close equare bracket.
    More formally: \t{ranged-dep ::= '[' (op ver logical-op)* op ver ']' }.

    The following logical operators are supported:

    \begin{description}
    \item[\&] An 'and' match. All op-ver pairs must match for a successful match.
    \item[|] An 'or' match. At least one op-ver pair must match for a successful match.
    \end{description}

    A ranged dependency must not mix logical operators.
}{
}

\IFKDEBUILDELSE
{
    \subsubsection{kdebuild Style Use Dependencies}
    \label{sec:kdebuild-use-dep}

    A kdebuild-style use dependency consists of one of the following:

    \begin{description}
    \item[{[opt]}] The flag must be enabled.
    \item[{[opt=]}] The flag must be enabled if the flag is enabled for the package with the
        dependency, or disabled otherwise.
    \item[{[opt!=]}] The flag must be disabled if the flag is enabled for the package with the
        dependency, or enabled otherwise.
    \item[{[opt?]}] The flag must be enabled if the flag is enabled for the package with the
        dependency.
    \item[{[opt!?]}] The flag must be enabled if the use flag is disabled for the package with the
        dependency.
    \item[{[-opt]}] The flag must be disabled.
    \item[{[-opt?]}] The flag must be disabled if the flag is enabled for the package with the
        dependency.
    \item[{[-opt!?]}] The flag must be disabled if the flag is disabled for the package with the
        dependency.
    \end{description}

    When multiple use dependencies are specified, all must match for a successful match.

    It is an error for a use dependency to be applied to an ebuild which does not have the flag in
    question in \t{IUSE\_REFERENCEABLE}, or for an ebuild to use a conditional use dependency when
    that ebuild does not have the flag in \t{IUSE\_EFFECTIVE}.
}{
}

\subsubsection{2-Style and 4-Style Use Dependencies}
\label{sec:use-dep}

A 2-style or 4-style use dependency consists of one of the following:

\begin{description}
\item[{[opt]}] The flag must be enabled.
\item[{[opt=]}] The flag must be enabled if the flag is enabled for the package with the
    dependency, or disabled otherwise.
\item[{[!opt=]}] The flag must be disabled if the flag is enabled for the package with the
    dependency, or enabled otherwise.
\item[{[opt?]}] The flag must be enabled if the flag is enabled for the package with the
    dependency.
\item[{[!opt?]}] The flag must be disabled if the use flag is disabled for the package with the
    dependency.
\item[{[-opt]}] The flag must be disabled.
\end{description}

Multiple requirements may be combined using commas, e.g. \t{[first,-second,third?]}.

When multiple requirements are specified, all must match for a successful match.

\featurelabel{use-dep-defaults} In a 4-style use dependency, the flag name may immediately be
followed by a \e{default} specified by either \t{(+)} or \t{(-)}. The former indicates that, when
applying the use dependency to a package that does not have the flag in question in
\t{IUSE\_REFERENCEABLE}, the package manager shall behave as if the flag were present and enabled;
the latter, present and disabled.

Unless a 4-style default is specified, it is an error for a use dependency to be applied to an
ebuild which does not have the flag in question in \t{IUSE\_REFERENCEABLE}.

\note By extension of the above, a default that could reference an ebuild using an EAPI not
supporting profile \t{IUSE} injections cannot rely upon any particular behaviour for flags that
would not have to be part of \t{IUSE}.

It is an error for an ebuild to use a conditional use dependency when that ebuild does not have the
flag in \t{IUSE\_EFFECTIVE}.

\IFKDEBUILDELSE
{
    \subsection{Package Dependency Labels}

    In EAPIs supporting \t{PDEPEND} labels, the following labels are legal. See
    section~\ref{sec:labels} for label behaviour in general.

    \begin{description}
    \item[required] Indicates a required post dependency.
    \item[suggested] Indicates a suggested post dependency. The package manager may ignore the
        suggestion or install it at user option.
    \end{description}

    The default label is \t{required}.
}{
}

\subsection{Restrict}
\label{sec:restrict}

The following tokens are permitted inside \t{RESTRICT}:

\begin{description}
\item[mirror] The package's \t{SRC\_URI} entries may not be mirrored, and mirrors should not
    be checked when fetching.
\item[fetch] The package's \t{SRC\_URI} entries may not be downloaded automatically. If
    entries are not available, \t{pkg\_nofetch} is called.
\item[strip] No stripping of debug symbols from files to be installed may be performed.
\item[userpriv] The package manager may not drop root privileges when building the package.
\item[test] The \t{src\_test} phase must not be run.
\item[sandbox] The \t{sandbox} tool must not be used when building the package.
\end{description}

Package managers may recognise other tokens, but ebuilds may not rely upon them being supported.

\subsection{Properties}
\label{sec:properties}

The following tokens are permitted inside \t{PROPERTIES}:

\begin{description}
\item[interactive] The package may require interaction with the user via the tty.
\end{description}

Ebuilds may not rely upon any token being supported.

\subsection{SRC\_URI}
\label{sec:src-uri-behaviour}

All filename components that are enabled (i.e. not inside a use-conditional block that is not
matched) in \t{SRC\_URI} must be available in the \t{DISTDIR} directory. In addition, these
components are used to make the \t{A} and \t{AA} variables.

If a component contains a full URI with protocol, that download location must be used. Package
managers may also consult mirrors for their files.

The special \t{mirror://} protocol must be supported. See section~\ref{sec:thirdpartymirrors} for mirror
details.

If a simple filename rather than a full URI is provided, the package manager can only use mirrors to
download the file.

\IFKDEBUILDELSE
{
    The \t{RESTRICT} metadata key can be used to impose additional restrictions upon downloading---see
    section~\ref{sec:restrict} for details. Labels also alter behaviour---see below for details.
}{
    The \t{RESTRICT} metadata key can be used to impose additional restrictions upon downloading---see
    section~\ref{sec:restrict} for details.
}

\featurelabel{src-uri-arrows} In EAPIs supporting arrows, if an arrow is used, the filename used
when saving to \t{DISTDIR} shall instead be the name on the right of the arrow. When consulting
mirrors (except for those explicitly listed on the left of the arrow, if \t{mirror://} is used), the
filename to the right of the arrow shall be requested instead of the filename in the URI.

\IFKDEBUILDELSE
{
    \featurelabel{src-uri-labels} In EAPIs supporting labels, the following labels are legal. See
    section~\ref{sec:labels} for label behaviour in general.

    \begin{description}
    \item[mirrors-first] Mirrors shall be consulted before the listed URI.
    \item[mirrors-only] Only mirrors shall be consulted, and the listed URI shall be ignored.
    \item[listed-first] The listed URI shall be consulted before mirrors.
    \item[listed-only] Only the listed URI shall be consulted.
    \item[manual] No URI shall be consulted. A manual download is required.
    \end{description}

    The default label is \t{mirrors-first}, unless \t{RESTRICT} contains \t{mirror}, in which case it is
    \t{listed-only}, unless \t{RESTRICT} contains \t{fetch}, in which case it is \t{manual}.
}{
}

% vim: set filetype=tex fileencoding=utf8 et tw=100 spell spelllang=en :

%%% Local Variables:
%%% mode: latex
%%% TeX-master: "pms"
%%% LaTeX-indent-level: 4
%%% LaTeX-item-indent: 0
%%% TeX-brace-indent-level: 4
%%% End:
