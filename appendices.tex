\appendix

\chapter{metadata.xml}
\label{metadata-xml}

The \t{metadata.xml} file is used to contain extra package- or category-level information beyond
what is stored in ebuild metadata. Its exact format is strictly beyond the scope of this document,
and is described in the DTD file located at \url{http://www.gentoo.org/dtd/metadata.dtd}.

\chapter{Unspecified Items}

The following items are not specified by this document, and must not be relied upon by ebuilds.
This is, of course, an incomplete list---it covers only the things that the authors know have
been abused in the past.

\begin{compactitem}
\item The \t{FEATURES} variable. This is Portage specific.
\item Similarly, any \t{PORTAGE\_} variable not explicitly listed.
\item Any Portage configuration file.
\item The VDB (\t{/var/db/pkg}). Ebuilds must not access this or rely upon it existing or being
    in any particular format.
\item The \t{portageq} command. The \t{has\_version} and \t{best\_version} commands are
    available as functions.
\item The \t{emerge} command.
\item Binary packages.
\item The \t{PORTDIR\_OVERLAY} variable, and overlay behaviour in general.
\end{compactitem}

% vim: set filetype=tex fileencoding=utf8 et tw=100 spell spelllang=en :

%%% Local Variables:
%%% mode: latex
%%% TeX-master: "pms"
%%% End:
