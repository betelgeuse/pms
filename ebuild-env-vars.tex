\section{Defined Variables}
\label{ebuild-env-vars}

\subsection{Globally Defined Variables}
\label{global-ebuild-vars}

The following variables are available, read-only, in all phases of the ebuild environment. They
must not be modified by ebuild code.
\begin{description}
\item[P] Package name and version, without the revision part. For example, \t{vim-7.0.174}.
\item[PN] Package name, for example \t{vim}.
\item[PV] Package version, for example \t{7.0.174}.
\item[PR] Package revision, or \t{r0} if none exists.
\item[PVR] Package version and revision, for example \t{7.0.174-r0} or \t{7.0.174-r1}.
\item[PF] Package name, version, and revision, for example \t{vim-7.0.174-r1}.
\item[A] All source files available for the package, whitespace separated with no leading
    or trailing whitespace, and in the order in which the item first appears in a matched
    component of \t{SRC\_URI}. Does not include any that are disabled because
    of USE conditionals. The value is calculated from the base names of each element of the
    \t{SRC\_URI} ebuild metadata variable.
\item[AA] All source files that could be available for the package, including any that are disabled
    in \t{A} because of USE conditionals. The value is calculated from the base names of each element
    of the \t{SRC\_URI} ebuild metadata variable.
\item[CATEGORY] The package's category, for example \t{app-editors}.
\item[FILESDIR] The full path to the package's files directory, used for small support files or
    patches. See section \ref{package-dirs}. May or may not exist; if a repository provides no
    support files for the package in question then an ebuild must be prepared for the situation
    where \t{FILESDIR} points to a non-existent directory.
\item[WORKDIR] The full path to the ebuild's working directory, in which all build data should be
    contained. \label{env-var-WORKDIR}
\item[T] The full path to a temporary directory for use by the ebuild.
\item[USE] A whitespace-delimited list of all active USE flags for this ebuild, including those
    originating from variables in \t{USE\_EXPAND}.
\item[] All variables set in the active profiles' \t{make.defaults} files will be exported to the
    ebuild environment.
\item[HOME] The full path to an appropriate temporary directory for use by any programs invoked by
    the ebuild that may read or modify the home directory.
\item[PATH] Must be initialized by the package manager to a ``usable'' default.  The exact value here
    is left up to interpretation, but it should include the equivalent ``sbin'' and ``bin'' and any
    package manager specific directories.
\item[CHOST,CBUILD,CTARGET] If not set by profiles, must contain either an appropriate
    machine tuple (the definition of appropriate is beyond the scope of this specification) or
    be unset.
\item[EBUILD\_PHASE] Takes one of the values \t{setup}, \t{nofetch}, \t{unpack}, \t{compile},
    \t{test}, \t{install}, \t{preinst}, \t{postinst}, \t{prerm}, \t{postrm} according to the ebuild
    function currently being executed.
\item[TMPDIR] Must be set to the location of a usable temporary directory, for any applications
    called by an ebuild. Should not be used by ebuilds directly; see \t{T} above.
\item[GZIP,BZIP,BZIP2,CDPATH,GREP\_OPTIONS,GREP\_COLOR,GLOBIGNORE] must not be set.
\end{description}

\subsection{Phase-specific variables}

In addition to the global list in section \ref{global-ebuild-vars}, there are several extra
variables available in specific ebuild phases.

\subsubsection{src\_install}
\begin{description}
\item[D] Contains the full path to the image directory into which the package should be installed.
    Must be non-empty and end in a trailing slash.
\item[INSDESTTREE] Controls the location where doins installs things.
\end{description}

\subsubsection{pkg\_preinst}
\begin{description}
\item[D] Contains the full path to the image directory of the package about to be merged. Must be
    non-empty and end in a trailing slash.
\end{description}

\subsubsection{pkg\_*}
\begin{description}
\item[ROOT] The absolute path to the root directory into which the package is to be merged. \t{ROOT}
    is only guaranteed to be available in \t{pkg\_setup}, \t{pkg\_preinst}, \t{pkg\_postinst},
    \t{pkg\_prerm} and \t{pkg\_postrm}. Other phases should not make use of it, as any binary
    package generated should not depend on the value of \t{ROOT}. Those phases which run with full
    filesystem access should not touch any files outside of the directory given in \t{ROOT}. Also of
    note is that in a cross-compiling environment, binaries inside of \t{ROOT} will not be
    executable on the build machine, so ebuilds should not call them. \t{ROOT} must be non-empty
    and end in a trailing slash.
\end{description}

% vim: set filetype=tex fileencoding=utf8 et tw=100 spell spelllang=en :
