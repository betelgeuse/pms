\section{Defined Variables}
\label{ebuild-env-vars}

The package manager must define the following environment variables. Not all variables are
meaningful in all phases; variables that are not meaningful in a given phase may be unset or set to
any value. Ebuilds must not attempt to modify any of these variables.

Because of their special meanings, these variables may not be preserved consistently across all
phases as would normally happen due to environment saving (see~\ref{ebuild-env-state}). For example,
\t{EBUILD\_PHASE} is different for every phase, and \t{ROOT} may have changed between the various
different \t{pkg\_*} phases. Ebuilds must recalculate any variable they derive from an inconsistent
variable.

\topcaption{Defined variables}
\tablehead{
    \hline
    \multicolumn{1}{c}{\b{Variable}} &
    \multicolumn{1}{c}{\b{Legal in}} &
    \multicolumn{1}{c}{\b{Consistent?}} &
    \multicolumn{1}{c}{\b{Description}} \\
    \hline
}
\begin{mpxtabular}{ l p{0.15\textwidth} l p{0.5\textwidth} }
\t{P} &
    all &
    No\footnote{May change if a package has been updated (see~\ref{updates-dir})} &
    Package name and version, without the revision part. For example, \t{vim-7.0.174}. \\
\t{PN} &
    all &
    ditto &
    Package name, for example \t{vim}. \\
\t{CATEGORY} &
    all &
    ditto &
    The package's category, for example \t{app-editors}. \\
\t{PV} &
    all &
    Yes &
    Package version, with no revision. For example \t{7.0.174}. \\
\t{PR} &
    all &
    Yes &
    Package revision, or \t{r0} if none exists. \\
\t{PVR} &
    all &
    Yes &
    Package version and revision, for example \t{7.0.174-r0} or \t{7.0.174-r1}. \\
\t{PF} &
    all &
    Yes &
    Package name, version, and revision, for example \t{vim-7.0.174-r1}. \\
\t{A} &
    \t{src\_*} &
    Yes &
    All source files available for the package, whitespace separated with no leading or trailing
    whitespace, and in the order in which the item first appears in a matched component of
    \t{SRC\_URI}\@. Does not include any that are disabled because of USE conditionals. The value is
    calculated from the base names of each element of the \t{SRC\_URI} ebuild metadata variable. \\
\t{AA}\footnote{This variable is generally considered deprecated. However, ebuilds must still
    assume that the package manager sets it. For example, a few configure scripts use this variable
    to find the \t{aalib} package; ebuilds calling such configure scripts must thus work around
    this.} &
    \t{src\_*} &
    Yes &
    All source files that could be available for the package, including any that are disabled in
    \t{A} because of USE conditionals. The value is calculated from the base names of each element
    of the \t{SRC\_URI} ebuild metadata variable. \\
\t{FILESDIR} &
    \t{src\_*}\footnote{Not necessarily present when installing from a binary package} &
    No &
    The full path to the package's files directory, used for small support files or
    patches. See section~\ref{package-dirs}. May or may not exist; if a repository provides no
    support files for the package in question then an ebuild must be prepared for the situation
    where \t{FILESDIR} points to a non-existent directory. \\
\t{PORTDIR} &
    ditto &
    No &
    The full path to the master repository's base directory. \\
\t{DISTDIR} &
    ditto &
    No &
    The full path to the directory in which the files in the \t{A} variable are stored. \\
\t{ECLASSDIR} &
    ditto &
    No &
    The full path to the master repository's eclass directory. \\
\t{ROOT} &
   \t{pkg\_*} &
   No &
   The absolute path to the root directory into which the package is to be merged.  Phases which run
   with full filesystem access must not touch any files outside of the directory given in
   \t{ROOT}\@. Also of note is that in a cross-compiling environment, binaries inside of \t{ROOT}
   will not be executable on the build machine, so ebuilds must not call them. \t{ROOT} must be
   non-empty and end in a trailing slash. \\
\t{T} &
    All &
    Partially\footnote{Consistent and preserved across a single connected sequence of install or
    uninstall phases, but not between install and uninstall. When reinstalling a package, this
    variable must have different values for the install and the replacement.} &
    The full path to a temporary directory for use by the ebuild. \\
\t{TMPDIR} &
    All &
    Ditto &
    Must be set to the location of a usable temporary directory, for any applications
    called by an ebuild. Must not be used by ebuilds directly; see \t{T} above. \\
\t{HOME} &
    All &
    Ditto &
    The full path to an appropriate temporary directory for use by any programs invoked by the
    ebuild that may read or modify the home directory. \\
\t{D} &
    \t{src\_install} &
    No &
    Contains the full path to the image directory into which the package should be installed.
    Must be non-empty and end in a trailing slash. \\
\t{D} (continued) &
    \t{pkg\_preinst}, \t{pkg\_postinst} &
    Yes &
    Contains the full path to the image that is about to be or has just been merged. Must be
    non-empty and end in a trailing slash. \\
\t{IMAGE}\footnote{Deprecated in favour of \t{D}.} &
    \t{pkg\_preinst}, \t{pkg\_postinst} &
    Yes &
    Equal to \t{D}. \\
\t{INSDESTTREE} &
    \t{src\_install} &
    No &
    Controls the location where doins installs things. \\
\t{USE} &
    All &
    Yes &
    A whitespace-delimited list of all active USE flags for this ebuild, including those
    originating from variables in \t{USE\_EXPAND}. \\
\t{EBUILD\_PHASE} &
    All &
    No &
    Takes one of the values \t{config}, \t{setup}, \t{nofetch}, \t{unpack},
    \t{compile}, \t{test}, \t{install}, \t{preinst}, \t{postinst}, \t{prerm}, \t{postrm}, \t{info}
    according to the top level ebuild function that was executed by the package manager. May be
    unset or any single word that is not any of the above when the ebuild is being sourced for
    other (e.g. metadata or QA) purposes. \\
\t{WORKDIR} &
    \t{src\_*} &
    Yes &
    The full path to the ebuild's working directory, in which all build data should be
    contained. \label{env-var-WORKDIR} \\
\hline
\end{mpxtabular}

All variables set in the active profiles' \t{make.defaults} files must be exported to the ebuild
environment. \t{CHOST}, \t{CBUILD} and \t{CTARGET}, if not set by profiles, must contain either an
appropriate machine tuple (the definition of appropriate is beyond the scope of this specification)
or be unset.

\t{PATH} must be initialized by the package manager to a ``usable'' default.  The exact value here
is left up to interpretation, but it should include the equivalent ``sbin'' and ``bin'' and any
package manager specific directories.

\t{GZIP}, \t{BZIP}, \t{BZIP2}, \t{CDPATH}, \t{GREP\_OPTIONS}, \t{GREP\_COLOR} and \t{GLOBIGNORE}
must not be set.

% vim: set filetype=tex fileencoding=utf8 et tw=100 spell spelllang=en :
