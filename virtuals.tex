\chapter{Virtual packages}
\label{old-virtuals}

Old-style virtuals are pseudo-packages -- they can be depended upon or installed, but do not exist
in the ebuild repository. This is in contrast to \i{new-style} virtuals, which are simply ebuilds in
the \t{virtual} category which install no files and use their dependency strings to select
providers. An old-style virtual requires several things in the repository: at least
one ebuild must list the virtual, whose category must always be \t{virtual}, in its \t{PROVIDE}
variable, and there must be at least one entry in a profiles \t{virtuals} file listing the default
provider for each profile -- see sections \ref{ebuild-var-provide} and \ref{profiles-virtuals} for
specifics on these two. Old-style virtuals require special handling as regards dependencies; this is
described below.

\section{Dependencies on virtual packages}

When a dependency on a virtual package is encountered, it must be resolved into a real package
before it can be satisfied. There are two factors that affect this process: whether a package
providing the virtual is installed, and the \t{virtuals} file in the active profile (section
\ref{profiles-virtuals}). If a package is already installed which satisfies the virtual requirement
(via \t{PROVIDE}), then it should be used to satisfy the dependency. Otherwise, the profiles
\t{virtuals} file (section \ref{profiles-virtuals}) should be consulted to choose an appropriate
provider.

\subsection{Blocks on virtual packages}

\TODO{Blocks on virtual packages. This is sort of covered in dependencies...}


% vim: set filetype=tex fileencoding=utf8 et tw=100 spell spelllang=en :
