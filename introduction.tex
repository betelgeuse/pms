\chapter{Introduction}

\section{Aims and Motivation}

This document aims to fully describe the format of an ebuild repository and the ebuilds therein, as
well as certain aspects of package manager behaviour required to support such a repository.

This document is \i{not} designed to be an introduction to ebuild development. Prior knowledge of
ebuild creation and an understanding of how the package management system works is assumed; certain
less familiar terms are explained in the Glossary in chapter \ref{glossary}.

This document does not specify any user or package manager configuration information.

\section{Rationale}

At present the only definition of what an ebuild can assume about its environment,
and the only definition of what is valid in an ebuild, is the source code of the latest Portage release
and a general consensus about which features are too new to assume availability. This has several
drawbacks: not only is it impossible to change any aspect of Portage behaviour without verifying
that nothing in the tree relies upon it, but if a new package manager should appear it becomes
impossible to fully support such an ill-defined standard.

This document aims to address both of these concerns by defining almost all aspects of what an
ebuild repository looks like, and how an ebuild is allowed to behave. Thus, both Portage and other
package managers can change aspects of their behaviour not defined here without worry of
incompatibilities with any particular repository.

\section{Conventions}

Text in \t{teletype} is used for filenames or variable names. \i{Italic} text is used for terms
with a particular technical meaning in places where there may otherwise be ambiguity.

% vim: set filetype=tex fileencoding=utf8 et tw=100 spell spelllang=en :
