\documentclass[a4paper,notumble]{leaflet}
\input{vc}
\usepackage[T1]{fontenc}
\usepackage[utf8]{inputenc}
\usepackage{
    url,
    xr-hyper,
    hyperref,
    listings,
    ifthen,
    mathptmx,
    courier
}
\usepackage[orig,english]{isodate}
\usepackage[scaled=.90]{helvet}
\newcommand{\code}[1]{\texttt{#1}}
\newcommand{\version}{0.2}
\newcommand{\featureref}[1]{\textsc{#1} on page~\pageref{feat:#1}}
\renewcommand{\familydefault}{\sfdefault}
\urlstyle{sf}
\externaldocument{pms}

\title{EAPI Cheat Sheet}
\author{Christian
    Faulhammer\thanks{\href{mailto:fauli@gentoo.org}{fauli@gentoo.org}}}
\ifthenelse{\equal{\VCDateISO}{}}
{
    \date{Version \version{}, generated on: \\\today}
}{
    \date{Version \version\\\printdate{\VCDateISO}}
}
\CutLine*{1}
\CutLine*{3}
\CutLine*{4}
\CutLine*{6}
\hypersetup{%
    urlcolor=black,
    colorlinks=true,
    citecolor=black,
    linkcolor=black,
    pdftitle={EAPI Desk Reference},
    pdfauthor={Christian Faulhammer},
    pdfsubject={Making look-up faster for EAPI features},
    pdflang={en},
    pdfkeywords={Gentoo, package manager, reference},
    pdfproducer={pdfLaTeX and hyperref},
}
\begin{document}
\maketitle
\thispagestyle{empty}
\begin{abstract}
    An overview of the main EAPI changes in Gentoo, for ebuild
    authors, who should always use the highest officially approved
    EAPI available in new ebuilds.  For full details, consult the
    Package Manager
    Specification\footnote{\url{http://www.gentoo.org/proj/en/qa/pms/}};
    this is an incomplete summary only.

    Official Gentoo EAPIs are consecutively numbered integers (0, 1,
    2, \dots).  Except where otherwise noted, an EAPI is the same as
    the previous EAPI.  All labels refer to the PMS document itself,
    built from the same checkout as this overview.

    Please report mistakes in or enhancements to this document via the
    Gentoo bug tracking system\footnote{\url{http://bugs.gentoo.org/}}
    to the original author or the PMS team.

    This document is released under the Creative Commons
    Attribution-Share Alike 3.0
    Licence\footnote{\url{http://creativecommons.org/licenses/by-sa/3.0/}}.
\end{abstract}
\section{EAPI 0}
\label{sec:cs:eapi0}
If there is no EAPI explicitly specified, EAPI 0 is assumed.
Currently there is no full specification what EAPI 0 includes.
Portage, official ebuild documentation and existing ebuilds set the
standard.  If you think you found a bug, you should file a bug report
nonetheless.
\newpage
\section{EAPI 1}
\label{sec:cs:eapi1}
\subsection{Additions/Changes}
\label{sec:cs:eapi1-additions}
\begin{description}
    \item[IUSE defaults] A USE flag can be marked as mandatory (if
    not disabled explicitly by user configuration) with a \code{+}
    sign in front.  See \featureref{iuse-defaults}.
    \item[Named slot dependencies] Dependencies can explicitly request
    a specific slot by using the \code{dev-libs/foo:\emph{SLOT\_name}} syntax.  See
    \featureref{slot-deps}.
\end{description}
\section{EAPI 2 (2008-09-25)}
\label{sec:cs:eapi2}
\subsection{Additions/Changes}
\label{sec:cs:eapi2-additions}
\begin{description}
    \item[\code{SRC\_URI} arrows] Allows redirection of upstream file
    naming scheme.  By using \code{SRC\_URI="http://some/url -> foo"}
    the file is saved as \code{foo} in DISTDIR.  See
    \featureref{src-uri-arrows}.
    \item[USE dependencies] Dependencies can specify USE flag
    requirements on their target, removing the need for
    \code{built\_with\_use} checks. A more powerful syntax that does
    not require the flag to be in IUSE is in EAPI 4.
    \begin{description}
        \item[{[opt]}] The flag must be enabled.
        \item[{[opt=]}] The flag must be enabled if the flag is
        enabled for the package with the dependency, or disabled
        otherwise.
        \item[{[!opt=]}] The flag must be disabled if the flag is
        enabled for the package with the dependency, or enabled
        otherwise.
        \item[{[opt?]}] The flag must be enabled if the flag is
        enabled for the package with the dependency.
        \item[{[!opt?]}] The flag must be disabled if the use flag is
        disabled for the package with the dependency.
        \item[{[-opt]}] The flag must be disabled.
    \end{description}
    See \featureref{use-deps}.
    \item[Blocker syntax] A single exclamation mark as a blocker may
    be ignored by the package manager as long as the stated package is
    uninstalled later on.  Two exclamation marks are a strong blocker
    and will always be respected.  See \featureref{bang-strength}.
    \item[\code{src\_configure, src\_prepare}] Both new phases provide
    finer granularity in the ebuild's structure.  Configure calls
    should be moved from \code{src\_compile} to \code{src\_configure}.
    Patching and similar preparation must now be done in
    \code{src\_prepare}, not \code{src\_unpack}.  See
    \featureref{src-prepare} and \featureref{src-configure}.
    \item[Default phase functions] The default functions for the
    phases \code{pkg\_nofetch}, \code{src\_unpack},
    \code{src\_prepare}, \code{src\_configure}, \code{src\_compile}
    and \code{src\_test} can be called via
    \code{default\_\emph{phasename}}, so duplicating the standard
    implementation is no longer necessary for small additions.  The
    short-hand \code{default} function calls the current phase's
    \code{default\_} function automatically, so any small additions
    you need will not be accompanied by a complete reimplementation of
    the phase.  See \featureref{default-phase-funcs} and
    \featureref{default-func}.
    \item[\code{doman} language support] The \code{doman} installation
    function recognizes lanugage specific man page extensions and
    behaves accordingly.  See \featureref{doman-langs}
\end{description}
\newpage
\section{EAPI 4 (not yet approved)}
\label{sec:cs:eapi4}
\subsection{Additions/Changes}
\label{sec:cs:eapi4-additions}
\begin{description}
    \item[\code{pkg\_pretend}] Some useful checks (kernel options for
    example) can be placed in this new phase to inform the user early
    (when just pretending to emerge the package).  Most checks should
    usually be repeated in \code{pkg\_setup}.  See
    \featureref{pkg-pretend}.
    \item[\code{src\_install}] The \code{src\_install} phase is no
    longer empty but has a default now.  This comes along with an
    accompanying \code{default} function.  See
    \featureref{src-install-4}.
    \item[\code{pkg\_info} on non-installed packages] The
    \code{pkg\_info} phase can be called even for non-installed
    packages.  Be warned that dependencies might not have been
    installed at execution time.  See \featureref{pkg-info}.
    \item[No RDEPEND fall-back] The package manager will not fall back
    to \code{RDEPEND=DEPEND} if RDEPEND is undefined.  See
    \featureref{rdepend-depend}
    \item[Support for \code{.xz}] Unpack of \code{.xz} and
    \code{.tar.xz} files is possible.  See
    \featureref{unpack-extensions}.
    \item[Slot operators] Dependencies that are both DEPEND and
    RDEPEND and that can match multiple slots should specify one of:
    \begin{description}
        \item[\code{:*}] Indicates that any slot value is
        acceptable. In addition, for runtime dependencies, indicates
        that the package will not break if the matched package is
        uninstalled and replaced by a different matching package in a
        different slot.
        \item[\code{:=}] Indicates that any slot value is
        acceptable. In addition, for runtime dependencies, indicates
        that the package will break unless a matching package with
        slot equal to the slot of the best installed version at the
        time the package was installed is available.
    \end{description}
    See \featureref{slot-operator-deps}.
    \item[USE dependency defaults] In addition to the features offered
    in EAPI 2 for USE dependencies, a \code{(+)} or \code{(-)} can be
    added after a USE flag (mind the parentheses).  The former
    specifies that flags not in IUSE should be treated as enabled; the
    latter, disabled. Cannot be used with USE\_EXPAND flags.  This
    mimicks parts of the behaviour of \code{-{}-missing} in
    \code{built\_with\_use}.  See \featureref{use-dep-defaults}.
    \item[Controllable compression] All items in
    \code{/usr/share\{doc,info,man\}} may be compressed on-disk after
    \code{src\_install}, except for
    \code{/usr/share/doc/\$\{PF\}/html}.  \code{docompress path \dots}
    adds paths to the inclusion list for compression.
    \code{docompress -x path \dots} adds paths to the exclusion list.
    \featureref{controllable-compress}.
    \item[\code{dodoc} recursion] If the \code{-r} switch is given as
    first argument and followed by directories, files from there are
    installed recursively.  See \featureref{dodoc}.
    \item[\code{doins} symlink support] Symbolic links are now
    properly installed when using recursion (\code{-r} switch).  See
    \featureref{doins}.
    \item[\code{nonfatal} for commands] If you call \code{nonfatal}
    the command given as argument will not abort the build process in
    case of a failure (as is the default) but will return non-zero on
    failure rather than aborting the build.  See
    \featureref{nonfatal}.
\end{description}
\subsection{Removals/Bans}
\label{sec:cs:eapi4-removalsbans}
\begin{description}
    \item[\code{dohard}, \code{dosed}] Both functions are not allowed
    anymore.  See \featureref{banned-commands}.
\end{description}
\end{document}

% vim: set filetype=tex fileencoding=utf8 et tw=100 spell spelllang=en :

%%% Local Variables:
%%% mode: latex
%%% LaTeX-indent-level: 4
%%% LaTeX-item-indent: 0
%%% TeX-brace-indent-level: 4
%%% End:

