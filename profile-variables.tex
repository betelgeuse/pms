\section{Profile variables}
\label{profile-variables}

This section documents variables that have special meaning, or special behaviour, when defined in a
profile's \t{make.defaults} file.

\subsection{Incremental Variables}
\i{Incremental} variables must stack between parent and child profiles in the following manner:
Beginning with the highest parent profile, tokenise the variable's value based on whitespace and
concatenate the lists. Then, for any token $T$ beginning with a hyphen, remove it and any previous
tokens whose value is equal to $T$ with the hyphen removed, or, if $T$ is equal to $-*$, remove
all previous values. Note that because of this treatment, the order of tokens in the final result is
arbitrary, not necessarily related to the order of tokens in any given profile. The following
variables must be treated in this fashion:
\begin{bulletlist}
\item \t{USE}
\item \t{USE\_EXPAND}
\item \t{USE\_EXPAND\_HIDDEN}
\item \t{CONFIG\_PROTECT}
\item \t{CONFIG\_PROTECT\_MASK}
\item Any variable whose name is listed in \t{USE\_EXPAND}
\end{bulletlist}

Other variables, except where they affect only package-manager-specific functionality (such as
Portage's \t{FEATURES} variable), should not be treated incrementally---later definitions should
completely override those in parent profiles.

\subsection{Specific variables and their meanings}
The following variables have specific meanings when set in profiles.
\begin{description}
\item[ARCH] The system's architecture. Must be a value listed in \t{profiles/arch.list}; see section
    \ref{arch.list} for more information. Must be equal to the primary \t{KEYWORD} for this profile.
\item[CONFIG\_PROTECT, CONFIG\_PROTECT\_MASK] Contain whitespace-delimited lists used to control the
    configuration file protection. Described more fully in chapter \ref{config-protect}.
\item[USE] Defines the list of default USE flags for this profile. Flags may be added or removed by
    the user's configuration.
\item[USE\_EXPAND] \label{use-expand} Defines a list of variables which are to be treated
    incrementally and whose contents are to be expanded into the USE variable as passed to ebuilds.
    Expansion is done as per Algorithm \ref{alg:use-expand}.
    \begin{algorithm}
    \caption{USE\_EXPAND logic} \label{alg:use-expand}
    \begin{algorithmic}
        \FOR{each variable $V$ listed in \t{USE\_EXPAND}}
            \FOR{each token $T$ in $V$}
                \STATE append $v$\_$T$ to \t{USE}, where $v$ is the lowercase of $V$
            \ENDFOR
        \ENDFOR
    \end{algorithmic}
    \end{algorithm}
    So, for example, if \t{USE\_EXPAND} contains `ALSA\_CARDS', and the \t{ALSA\_CARDS} variable
    contains `foo', `alsa\_cards\_foo' will be appended to \t{USE}.
\item[USE\_EXPAND\_HIDDEN] Contains a list of variables contained in \t{USE\_EXPAND} which are not
    to be displayed to the end user.
\end{description}

Any other variables set in \t{make.defaults} must be passed on into the ebuild environment as-is,
and are not required to be interpreted by the package manager.

% vim: set filetype=tex fileencoding=utf8 et tw=100 spell spelllang=en :
