\chapter{Glossary}
\label{sec:glossary}

This section contains explanations of some of the terms used in this document whose meaning may not
be immediately obvious.

\begin{description}
\item[qualified package name] A package name along with its associated category. For example,
    \t{app-editors/vim} is a qualified package name.
\item[old-style virtual] An old-style virtual is a psuedo-package which exists if it is listed in an
    ebuild's \t{PROVIDE} variable. See chapter~\ref{sec:old-virtuals}.
\item[new-style virtual] A new-style virtual is a normal package in the \t{virtual} category which
    installs no files and uses its dependency requirements to pull in a `provider'. This is more
    flexible than the old-style virtuals described above, and requires no special package manager
    code.
\item[stand-alone repository] An (ebuild) repository which is intended to function on its own as the
    only, or primary, repository on a system. Contrast with \i{slave repository} below.
\item[slave repository, non-stand-alone repository] An (ebuild) repository which is not complete
    enough to function on its own, but needs one or more \i{master repositories} to
    satisfy dependencies and provide repository-level support files. Known in Portage as an overlay.
\item[master repository] See above.

\end{description}


% vim: set filetype=tex fileencoding=utf8 et tw=100 spell spelllang=en :

%%% Local Variables:
%%% mode: latex
%%% TeX-master: "pms"
%%% End:
