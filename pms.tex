\documentclass[a4paper]{book}
\input{vc}
% Definition of fonts, choose T1 encoding for fonts
\usepackage[T1]{fontenc}
%
% algorithmic and algorithm to be loaded last to avoid failures
\usepackage{appendix,
  booktabs,
  color,
  enumitem,
  float,
  fullpage,
  graphicx,
  hyperref,
  ifthen,
  longtable,
  paralist,
  parskip,
  verbatim,
  algorithm,
  algorithmic,
  lscape,
  marginnote
}
\usepackage[orig,english]{isodate}
\usepackage{typearea}
\usepackage[position=top]{caption}
\usepackage[utf8]{inputenc}

\newboolean{TEX4HT-HACKS}
\ifx\HCode\undefined
    \usepackage{mathptmx,
      courier
    }
    \usepackage[scaled=.90]{helvet}
    \setboolean{TEX4HT-HACKS}{false}
\else
    \setboolean{TEX4HT-HACKS}{true}
\fi

\floatstyle{plaintop}
\newfloat{listing}{tbp}{lol}[chapter]
\floatname{listing}{Listing}
\newcommand{\listoflistings}{\listof{listing}{Listings}}

\bibliographystyle{plainurl}

\renewcommand{\t}[1]{\texttt{#1}}
\renewcommand{\i}[1]{\textit{#1}}
\newcommand{\e}[1]{\emph{#1}}
\renewcommand{\b}[1]{\textbf{#1}}
\newcommand{\note}[1]{\paragraph{Note:} #1}

\newcommand{\featureref}[1]{\e{#1} on page~\pageref{feat:#1}}
\newcommand{\compactfeatureref}[1]{#1~p\pageref{feat:#1}}
\newcommand{\featurelabel}[1]{\marginnote{\framebox{#1}}\label{feat:#1}}

\definecolor{deepblue}{rgb}{0.0, 0.2, 0.7}
\definecolor{deeppurple}{rgb}{0.7, 0.0, 0.8}

\newboolean{ENABLE-ALL-OPTIONS}
\newboolean{ENABLE-KDEBUILD}

%%% Enable the below option if you'd like to see both sides of KDEBUILD conditionals shown in
%%% different colours. Disable it to either fully enable or fully disable KDEBUILD.
%%% Not compatible with HTML output.
\setboolean{ENABLE-ALL-OPTIONS}{false}

%%% Enable the below if you'd like to see KDEBUILD things.
\setboolean{ENABLE-KDEBUILD}{true}

\ifthenelse{\boolean{ENABLE-ALL-OPTIONS}\and\not\boolean{TEX4HT-HACKS}}
{
    \newcommand{\IFKDEBUILDELSE}[2]{{\def\mycolour{\color{deepblue}}\mycolour #1}{\def\mycolour{\color{deeppurple}}\mycolour #2}}
    \newcommand{\IFANYKDEBUILDELSE}[2]{#1}
    \newcommand{\IFKDEBUILDCOLOUR}[1]{{\def\mycolour{\color{deepblue}}\mycolour #1}}
}{
    \ifthenelse{\boolean{ENABLE-KDEBUILD}}
    {
        \newcommand{\IFKDEBUILDELSE}[2]{#1}
        \newcommand{\IFANYKDEBUILDELSE}[2]{#1}
        \newcommand{\IFKDEBUILDCOLOUR}[1]{#1}
    }{
        \newcommand{\IFKDEBUILDELSE}[2]{#2}
        \newcommand{\IFANYKDEBUILDELSE}[2]{#2}
        \newcommand{\IFKDEBUILDCOLOUR}[1]{#1}
    }
}

\newenvironment{centertable}[1]%
{
  \begin{table}
    \ifx\mycolour\undefined\else\mycolour\fi
    \centering
    \caption{#1}
  }{
  \end{table}
}

\hypersetup{%
  urlcolor=black,
  colorlinks=true,
  citecolor=black,
  linkcolor=black,
  pdftitle={Package Manager Specification},
  pdfauthor={Stephen P. Bennett, Ciaran McCreesh},
  pdfcreator={pdfLaTeX and hyperref},
  pdfsubject={Defining a feature set for package managers in the
    Gentoo world},
  pdflang={en},
  pdfkeywords={Gentoo, package manager, specification},
  pdfproducer={pdfLaTeX and hyperref},
}
\title{Package Manager Specification}
\author{Stephen P. Bennett\\\url{spb@exherbo.org}
\and Ciaran McCreesh\\\url{ciaran.mccreesh@googlemail.com}}
% Make the build succeed even when no Git repository is available
\ifthenelse{\equal{\VCDateISO}{}}
{
  \date{Generated on: \today}
}{
  \date{\printdate{\VCDateISO}}
}

\pagestyle{myheadings}
\markboth{\scshape Package Manager Specification}{\scshape Package
  Manager Specification}
% This is some kind of hack.  We set the proportions of the text area
% and then move it 20mm to the left to increase the right (outer)
% margin.
\areaset[-20mm]{400pt}{700pt}

\begin{document}
\maketitle

\tableofcontents
\listofalgorithms
\listoflistings
\listoftables

\chapter*{}

\section*{Acknowledgements}

Thanks to Mike Kelly (package manager provided utilities, section \ref{pkg-mgr-commands}),
Danny van Dyk (ebuild functions, section \ref{ebuild-functions}), and
Petteri R\"aty (environment state, section \ref{ebuild-env-state}) for contributions.

\section*{Copyright and Licence}

The bulk of this document is \textcopyright 2007 Stephen Bennett and Ciaran McCreesh. Contributions
are owned by their respective authors, and may have been changed substantially before inclusion.

When ready for public distribution, this document will be released under the Creative Commons
Attribution-Share Alike 3.0 License, found at \url{http://creativecommons.org/licenses/by-sa/3.0/}.
Until such time, it may not be distributed or published without permission from the authors. If no
modifications are made to this document for a period of three months, it may be considered to have
been released under the above licence.

\section*{Reporting Issues}

Issues (inaccuracies, wording problems, omissions etc.) in this document should be reported via
Gentoo bugzilla using product \e{Gentoo Hosted Projects}, component \e{PMS/EAPI} and the default
assignee. There should be one bug per issue, and one issue per bug.

When reporting issues, remember that this document is not the appropriate place for pushing
through changes to the tree or the package manager, except where those changes are bugs.


% vim: set filetype=tex fileencoding=utf8 et tw=100 spell spelllang=en :


\chapter{Introduction}

\section{Aims and Motivation}

This document aims to fully describe the format of an ebuild repository and the ebuilds therein, as
well as certain aspects of package manager behaviour required to support such a repository.

This document is \i{not} designed to be an introduction to ebuild development. Prior knowledge of
ebuild creation and an understanding of how the package management system works is assumed; certain
less familiar terms are explained in the Glossary in chapter~\ref{glossary}.

This document does not specify any user or package manager configuration information.

\section{Rationale}

At present the only definition of what an ebuild can assume about its environment,
and the only definition of what is valid in an ebuild, is the source code of the latest Portage release
and a general consensus about which features are too new to assume availability. This has several
drawbacks: not only is it impossible to change any aspect of Portage behaviour without verifying
that nothing in the tree relies upon it, but if a new package manager should appear it becomes
impossible to fully support such an ill-defined standard.

This document aims to address both of these concerns by defining almost all aspects of what an
ebuild repository looks like, and how an ebuild is allowed to behave. Thus, both Portage and other
package managers can change aspects of their behaviour not defined here without worry of
incompatibilities with any particular repository.

\section{Conventions}

Text in \t{teletype} is used for filenames or variable names. \i{Italic} text is used for terms
with a particular technical meaning in places where there may otherwise be ambiguity.

The term \i{package manager} is used throughout this document in a broad sense. Although some parts
of this document are only relevant to fully featured package managers, many items are equally
applicable to tools or other applications that interact with ebuilds or ebuild repositories.

\section{EAPIs}

An EAPI can be thought of as a `version' of this specification to which a package conforms. An EAPI
value is a string. The following EAPIs are defined by this specification:

\begin{description}
\item[0] The `original' base EAPI.
\item[1] EAPI `1' contains a number of extensions to EAPI `0'. Except where explicitly noted, it is
    in all other ways identical to EAPI `0'.
\IFKDEBUILDELSE
{
    \item[kdebuild-1] A series of extensions to EAPI `1' used by the Gentoo KDE project. Except where
        explicitly noted, it is in all other ways identical to EAPI `1'.
}{
}
\end{description}

\ifthenelse{\boolean{ENABLE-ALL-OPTIONS}}
{
    \note We're not sure whether \t{kdebuild-1} will end up in the final version of this
    specification. For now, it's included but can easily be hidden using a switch in the master
    \t{pms.tex} file. To make editing easier, we also have a mode that shows the document both with
    and without the \t{kdebuild-1} stuff enabled. You currently have that mode enabled---
    \IFKDEBUILDELSE{
        text only shown when \t{kdebuild-1} is enabled looks like this,
    }{
        and text only shown when it is disabled looks like this.
    }
}{
}

Except where explicitly noted, everything in this specification applies to all EAPIs.

If a package manager encounters a package version with an unrecognised EAPI, it must not attempt to
perform any operations upon it. It could, for example, ignore the package version entirely (although
this can lead to user confusion), or it could mark the package version as masked. A package manager
must not use any metadata generated from a package with an unrecognised EAPI.

The package manager must not attempt to perform any kind of comparison test other than equality upon
EAPIs.

\subsection{Reserved EAPIs}

\begin{compactitem}
\item EAPIs whose value consists purely of an integer are reserved for future versions of this
    specification.
\item EAPIs whose value starts with the string \t{paludis-} are reserved for experimental
    use by the Paludis package manager.
\item EAPIs whose value starts with the string \t{kdebuild-} are reserved for the Gentoo KDE
    project.
\end{compactitem}

% vim: set filetype=tex fileencoding=utf8 et tw=100 spell spelllang=en :

%%% Local Variables:
%%% mode: latex
%%% TeX-master: "pms"
%%% End:


\chapter{EAPIs}

\section{Definition}

An EAPI can be thought of as a `version' of this specification to which a package conforms. An EAPI
value is a string, and is part of an ebuild's metadata.

If a package manager encounters a package version with an unrecognised EAPI, it must not attempt to
perform any operations upon it. It could, for example, ignore the package version entirely (although
this can lead to user confusion), or it could mark the package version as masked. A package manager
must not use any metadata generated from a package with an unrecognised EAPI.

The package manager must not attempt to perform any kind of comparison test other than equality upon
EAPIs.

EAPIs are also used for profile directories, as described in section~\ref{sec:profile-eapi}.

\section{Defined EAPIs}

The following EAPIs are defined by this specification:

\begin{description}
\item[0] The `original' base EAPI.
\item[1] EAPI `1' contains a number of extensions to EAPI `0'. Except where explicitly noted, it is
    in all other ways identical to EAPI `0'.
\item[2] EAPI `2' contains a number of extensions to EAPI `1'. Except where explicitly noted, it is
    in all other ways identical to EAPI `1'.
\item[3] EAPI `3' contains a number of extensions to EAPI `2'. Except where explicitly noted, it is
    in all other ways identical to EAPI `2'.
\item[4] EAPI `4' contains a number of extensions to EAPI `3'. Except where explicitly noted, it is
    in all other ways identical to EAPI `3'.
\end{description}

Except where explicitly noted, everything in this specification
applies to all of the above EAPIs.%
\footnote{Another unofficial EAPI `kdebuild-1' was a series of
    extensions to EAPI `1' formerly used by the Gentoo KDE project.
    Some of its features have been included in EAPI `2' or later.}

\section{Reserved EAPIs}

\begin{compactitem}
\item EAPIs whose value consists purely of an integer are reserved for future versions of this
    specification.
\item EAPIs whose value starts with the string \t{paludis-} are reserved for experimental
    use by the Paludis package manager.
\end{compactitem}

% vim: set filetype=tex fileencoding=utf8 et tw=100 spell spelllang=en :

%%% Local Variables:
%%% mode: latex
%%% TeX-master: "pms"
%%% LaTeX-indent-level: 4
%%% LaTeX-item-indent: 0
%%% TeX-brace-indent-level: 4
%%% End:


\chapter{Names and Versions}

\section{Restrictions upon Names}

No name may be empty. Package managers must not impose fixed upper boundaries upon the length of any
name.

\subsection{Category Names}
A category name may contain any of the characters [\t{A-Za-z0-9+\_.-}]. It must not begin with
a hyphen or a dot.

\note A hyphen is \i{not} required because of the \t{virtual} category. Usually, however, category
names will contain a hyphen.

\subsection{Package Names}
A package name may contain any of the characters [\t{A-Za-z0-9+\_-}]. It must not begin with a
hyphen, and must not end in a hyphen followed by one or more digits.

\note A package name does not include the category. The term \i{qualified package name} is used
where a \t{category/package} pair is meant.

\subsection{Slot Names}
\label{slot-names}
A slot name may contain any of the characters [\t{A-Za-z0-9+\_.-}]. It must not begin with a
hyphen or a dot.

\subsection{USE Flag Names}
A USE flag name may contain any of the characters [\t{A-Za-z0-9+\_@-}]. It must begin with an
alphanumeric character. Underscores should be considered reserved for \t{USE\_EXPAND}, as
described in section \ref{use-expand}.

\note The at-sign is required for \t{LINGUAS}.

\subsection{Repository Names}
\label{repository-names}
A repository name may contain any of the characters [\t{A-Za-z0-9\_-}]. It must not begin with a
hyphen.

\subsection{Keyword Names}
\label{keyword-names}
A keyword name may contain any of the characters [\t{A-Za-z0-9\_-}]. It must not begin with a
hyphen. In contexts where it makes sense to do so, a keyword name may be prefixed by
a tilde or a hyphen. In \t{KEYWORDS}, $-*$ is also acceptable as a keyword, to indicate that
a package will only work on listed targets.

A tilde prefixed keyword is, by convention, used to indicate a less stable package. It is generally
assumed that any user accepting keyword \t{\textasciitilde{}foo} will also accept \t{foo}.

\section{Version Specifications}
The package manager must not impose fixed limits upon the number of version components. No
integer part of a version specification may contain more than eight digits.

A version starts with the number part, which is in the form \t{[0-9]+($\backslash$.[0-9]+)*} (a positive
integer, followed by zero or more dot-prefixed positive integers).

This may optionally be followed by one of \t{[a-z]} (a lowercase letter).

This may be followed by zero or more of the suffixes \t{\_alpha}, \t{\_beta}, \t{\_pre},
\t{\_rc} or \t{\_p}, which themselves may be suffixed by an optional integer (if not present,
this integer is assumed to be zero).

This may optionally be followed by the suffix \t{-r} followed immediately by an integer (the
``revision number``). If this suffix is not present, it is assumed to be \t{-r0}.

\TODO{Portage has a really nasty -cvs thing that was designed to work around Portage's lack of
    ability to ignore versions it doesn't recognise. This should probably remain excluded and replaced
    by a proper solution in future EAPIs.}

\section{Version Comparison}

Version specifications are compared component by component, moving from left to right.

The first component of the number part is compared using strict integer comparison.

Any subsequent components of the number part are compared as follows:

\begin{bulletlist}
\item If neither component has a leading zero, components are compared using strict integer
  comparison.
\item Otherwise, if a component has a leading zero, any trailing zeroes in that component
  are stripped (if this makes the component empty, proceed as if it were \t{0} instead),
  and the components are compared using a stringwise comparison.
\end{bulletlist}

Note in particular that \t{1.0} is less than \t{1.0.0}.

Letter suffixes are compared alphabetically, with any letter being newer than no letter.

If the letters are equal, suffixes are compared. The ordering is \t{\_alpha} is less than
\t{\_beta} is less than \t{\_pre} is less than \t{\_rc} is less than no suffix is less than
\t{\_p}. If a suffix string is equal, the associated integer parts (which default to zero)
are compared.

If at this point the two versions are still equal, the revision number is compared. The revision
number has an optional integer suffix as per the previous part. If the revision numbers are equal,
so are the two versions.

\section{Uniqueness of versions}

No two packages in a given repository may have the same qualified package name and equal versions.
For example, a repository may not contain more than one of \t{foo-bar/baz-1.0.2},
\t{foo-bar/baz-1.0.2-r0} and \t{foo-bar/baz-1.000.2}.

% vim: set filetype=tex fileencoding=utf8 et tw=100 spell spelllang=en :


\chapter{Tree Layout}

This chapter defines the layout on-disk of an ebuild repository. In all cases below where a file or
directory is specified, a symlink to a file or directory is also valid. In this case, the package
manager should follow the operating system's semantics for symbolic links and not behave differently
from normal.

\section{Top Level}

An ebuild repository shall occupy one directory on disk, with the following subdirectories:
\begin{bulletlist}
\item One directory per category, whose name shall be the name of the category. The layout of
    these directories shall be as described in section \ref{category-dirs}.
\item A \t{profiles} directory, described in section \ref{profiles-dir}.
\item A \t{licenses} directory (optional), described in section \ref{licenses-dir}.
\item An \t{eclass} directory (optional), described in section \ref{eclass-dir}.
\item A \t{metadata} directory (optional), described in section \ref{metadata-dir}.
\item Other optional support files and directories (skeleton ebuilds or ChangeLogs,
    for example) may exist but are not covered by this specification. The package manager must
    ignore any of these files or directories that it does not recognise.

\end{bulletlist}

\section{Category Directories}
\label{category-dirs}

Each category provided by the repository (see also: the \t{profiles/categories} file, section
\ref{profiles-categories}) shall be contained in one directory, whose name shall be that of the
category. Each category directory shall contain:
\begin{bulletlist}
\item A \t{metadata.xml} file, as described in appendix \ref{metadata-xml}. Optional.
\item Zero or more package directories, one for each package in the category, as described in section
    \ref{package-dirs}. The name of the package directory shall be the corresponding package name.
\end{bulletlist}

Category directories may contain additional files, whose purpose is not covered by this
specification. Additional directories that are not for a package may \i{not} be present, to avoid
conflicts with package name directories; an exception is made for filesystem components whose name
starts with a dot, which the package manager must ignore.

\TODO{Explicitly ignore CVS too?}

\section{Package Directories}
\label{package-dirs}

A package directory contains the following:
\begin{bulletlist}
\item One or more ebuilds. These are as described in section \ref{ebuild-format} and others.
\item A \t{metadata.xml} file, as described in appendix \ref{metadata-xml}. Optional only for
    legacy support.
\item A \t{ChangeLog}, in a format determined by the provider of the respository. Optional.
\item A \t{Manifest} file, whose format is described in \cite{Glep44}.
\item A \t{files} directory, containing any support files needed by the ebuilds. Optional.
\end{bulletlist}

\section{The Profiles Directory}
\label{profiles-dir}

The profiles directory shall contain zero or more profile directories as described in section
\ref{profiles}, as well as the following files and directories. In any line-based file, lines
beginning with a \# character are treated as comments, whilst blank lines are ignored. All contents
of this directory, with the exception of \t{repo\_name}, are optional if the repository is not
intended to be stand-alone; if they are not present their contents are to be taken, where necessary,
from the master repository. Other files may exist, but may not be relied upon. The package manager
must ignore any files in this directory that it does not recognise.

\begin{description}
\item[arch.list] \label{arch.list} Contains a list, one entry per line, of permissible values for
    the \t{ARCH} variable, and hence permissible keywords for packages in this repository.
\item[categories] \label{profiles-categories} Contains a list, one entry per line, of categories
    provided by this repository.
\item[info\_pkgs] Contains a list, one entry per line, of qualified package names. Any package
    matching one of these is to be listed when a package manager displays a `system information'
    listing.
\item[info\_vars] Contains a list, one entry per line, of profile, configuration, and environment
    variables which are considered to be of interest. The value of each of these is to be included
    when the package manager displays a `system information' listing.
\item[package.mask] \label{profiles-package.mask}
    Contains a list, one entry per line, of (EAPI-0) dependency atoms. Any package
    version matching one of these is considered to be masked, and will not be installed regardless
    of profile unless it is unmasked by the user configuration.
\item[profiles.desc] Described below in section \ref{profiles.desc}.
\item[repo\_name] Contains, on a single line, the name of this repository. The repository name must
    conform to section \ref{repository-names}.
\item[thirdpartymirrors] Described below in section \ref{thirdpartymirrors}.
\item[use.desc] Contains descriptions of valid global USE flags for this repository. The format is
    described in section \ref{use.desc}.
\item[use.local.desc] Contains descriptions of valid local USE flags for this repository, along with
    the packages to which they apply. The format is as described in section \ref{use.desc}.
\item[desc/] This directory contains files analogous to \t{use.desc} for the various \t{USE\_EXPAND}
    variables. Each file in it is named \t{<varname>.desc}, where \t{<varname>} is the variable
    name, in lowercase, whose possible values the file describes. The format of each file is as for
    \t{use.desc}, described in section \ref{use.desc}. The \t{USE\_EXPAND} name is \i{not}
    included as a prefix here.
\item[updates/] This directory is described in section \ref{updates-dir}.
\end{description}

\subsection{The profiles.desc file}
\label{profiles.desc}
\t{profiles.desc} is a line-based file, with the standard commenting rules from section
\ref{profiles-dir}, containing a list of profiles that are valid for use, along with their
associated architecture and status. Each line has the format:
\begin{verbatim}
<keyword> <profile path> stable|dev
\end{verbatim}
Where \t{<keyword>} is the default keyword for the profile and the \t{ARCH} for which the profile is
valid, \t{<profile path>} is the (relative) path from the \t{profiles} directory to the profile in
question, and the third field is either \t{stable} or \t{dev}, depending upon whether the profile is
reckoned to be `stable' for normal use. Fields are whitespace-delimited. The last field is of most
use to QA scanning tools, which can display certain errors with reduced severity should they appear
in a `dev' profile.

\subsection{The thirdpartymirrors file}
\label{thirdpartymirrors}
\t{thirdpartymirrors} is another simple line-based file, describing the valid mirrors for use with
\t{mirror://} URIs in this repository, and the associated download locations. The format of each
line is:
\begin{verbatim}
<mirror name> <mirror 1> <mirror 2> ... <mirror n>
\end{verbatim}
Fields are whitespace-delimited. When parsing a URI of the form \t{mirror://name/filename}, the
\t{thirdpartymirrors} file is searched for a line whose first field is \t{name}. Then the download
URIs in the subsequent fields have \t{filename} appended to them to generate the URIs from which a
download is attempted.

\subsection{use.desc and related files}
\label{use.desc}
\t{use.desc} contains descriptions of every valid global USE flag for this repository. It is a
line-based file with the standard rules for comments and blank lines. The format of each line is:
\begin{verbatim}
<flagname> - <description>
\end{verbatim}

\t{use.local.desc} contains descriptions of every valid local USE flag---those that apply only to a
small number of packages, or that have different meanings for different packages. Its format is:
\begin{verbatim}
<category/package>:<flagname> - <description>
\end{verbatim}
Flags must be listed once for each package to which they apply, or if a flag is listed in both
\t{use.desc} and \t{use.local.desc}, it must be listed once for each package for which its meaning
differs from that described in \t{use.desc}.

\subsection{The updates directory}
\label{updates-dir}
The \t{updates} directory is used to inform the package manager that a package has moved categories,
names, or that a version has changed SLOT. It contains one file per quarter year, named
\t{[1-4]Q-[YYYY]} for the first to fourth quarter of a given year, for example \t{1Q-2004} or
\t{3Q-2006}. The format of each file is again line-based, with each line having one of the following
formats:
\begin{verbatim}
move <qpn1> <qpn2>
slotmove <atom> <slot1> <slot2>
\end{verbatim}
The first form, where \t{qpn1} and \t{qpn2} are \i{qualified package names}, instructs the package
manager that the package \t{qpn1} has changed name, category, or both, and is now called \t{qpn2}.

The second form instructs the package manager that any currently installed package version matching
\t{atom} whose \t{SLOT} is set to \t{slot1} should have it updated to \t{slot2}.


\section{The Licenses Directory}
\label{licenses-dir}

The \t{licenses} directory shall contain copies of the licenses used by packages in the
repository. Each file will be named according to the name used in the \t{LICENSE} variable as
described in section \ref{ebuild-var-LICENSE}, and will contain the complete text of the license in
human-readable form. Plain text format is strongly preferred but not required.

\section{The Eclass Directory}
\label{eclass-dir}

The \t{eclass} directory shall contain copies of the eclasses provided by this repository. The
format of these files is described in section \ref{eclasses}. It may also contain, in their own
directory, support files needed by these eclasses.

\section{The Metadata Directory}
\label{metadata-dir}

The \t{metadata} directory contains various repository-level metadata that is not contained in
\t{profiles/}. All contents are optional. In this standard only the \t{cache} subdirectory is
described; other contents are optional but may include security advisories, DTD files for the
various XML files used in the repository, and repository timestamps.

\subsection{The metadata cache}

The \t{metadata/cache} directory contains a cached form of all important ebuild metadata variables.
The cache directory, if it exists, contains (up to) one directory per category in the repository---
not all categories and packages must be contained in it. Each subdirectory contains one file per
package version, named \t{<package>-<version>}, in the following format:

Each cache file contains the textual values of various metadata keys, one per line, in the following
order. Other lines may be present following these; their meanings are not defined here.

\begin{enumerate}
\item Build-time dependencies (\t{DEPEND})
\item Run-time dependencies (\t{RDEPEND})
\item Slot (\t{SLOT})
\item Source tarball URIs (\t{SRC\_URI})
\item \t{RESTRICT}
\item Package homepage (\t{HOMEPAGE})
\item Package license (\t{LICENSE})
\item Package description (\t{DESCRIPTION})
\item Package keywords (\t{KEYWORDS})
\item Inherited eclasses (\t{INHERITED})
\item Use flags that this package respects (\t{IUSE})
\item No longer used; this line is to be ignored.
\item Post dependencies (\t{PDEPEND})
\item Old-style virtuals provided by this package (\t{PROVIDE})
\item The ebuild API version to which this package conforms (\t{EAPI})
\end{enumerate}

% vim: set filetype=tex fileencoding=utf8 et tw=100 spell spelllang=en :


\chapter{Profiles}
\label{profiles}

\section{General principles}
Generally, a profile defines information specific to a certain `type' of system---it lies somewhere
between repository-level defaults and user configuration in that the information it contains is not
necessarily applicable to all machines, but is sufficiently general that it should not be left to
the user to configure it. Some parts of the profile can be overridden by user configuration, some
only by another profile.

The format of a profile is relatively simple. Each profile is a directory containing any number of
the files described in this chapter, and possibly inheriting another profile. The files themselves
follow a few basic conventions as regards inheritance and format; these are described in the next
section. It may also contain any number of subdirectories containing other profiles.

\section{Files that make up a profile}
\subsection{The parent file}
A profile may contain a \t{parent} file. Each line must contain a relative path to another profile
which will be considered as one of this profile's parents. Any settings from the parent are
inherited by this profile, and can be overridden by it. Precise rules for how settings are combined
with the parent profile vary between files, and are described below. Parents are handled depth
first, left to right, with duplicate parent paths being sourced for every time they are encountered.

It is illegal for a profile's parent tree to contain cycles. Package manager behaviour upon
encountering a cycle is undefined.

This file must not contain comments, blank lines or make use of line continuations.

\subsection{deprecated}
If a profile contains a file named \t{deprecated}, it is treated as such. The first line of this
file should contain the path from the \t{profiles} directory of the repository to a valid profile
that is the recommended upgrade path from this profile. The remainder of the file can contain any
text, which may be displayed to users using this profile by the package manager. This file is not
inherited---profiles which inherit from a deprecated profile are \e{not} deprecated.

This file must not contain comments or make use of line continuations.

\subsection{make.defaults}
\t{make.defaults} is used to define defaults for various environment and configuration variables.
This file is unusual in that it is not combined at a file level with the parent---instead, each
variable is combined or overridden individually as described in section \ref{profile-variables}.

The file itself is a line-based key-value format. Each line contains a single \verb|VAR="value"|
entry, where the value must be double quoted. A variable name must start with one of \t{a-zA-Z}
and may contain \t{a-zA-Z0-9\_-}. Additional syntax, which is a small subset of
bash syntax, is allowed as follows:

\begin{bulletlist}
\item Variables to the right of the equals sign in the form \t{\$\{foo\}} or \t{\$foo} are recognised and
  expanded from variables previously set in this or earlier \t{make.defaults} files.
\item One logical line may be continued over multiple physical lines by escaping the newline with a
  backslash. This is also permitted inside quoted strings.
\item Backslashes, except for line continuations, are not allowed.
\end{bulletlist}

\subsection{virtuals}
\label{profiles-virtuals}
The \t{virtuals} file defines default providers for ``old-style'' virtual packages. It is a simple
line-based file, with each line containing two whitespace-delimited tokens. The first is a virtual
package name (for example, \t{virtual/alsa}) and the second is a qualified package name. Blank lines
and those beginning with a \# character are ignored. When attempting to resolve a virtual name to a
concrete package, the specification defined in the active profile's \t{virtuals} list should be used if no
provider is already installed.

The \t{virtuals} file is inherited in the simplest manner: all entries from the parent profile are
loaded, then entries from the current profile. If a virtual package name appears in both, the entry
in the parent profile is discarded.

\subsection{use.defaults}
The \t{use.defaults} file is used to implement `autouse'---switching USE flags on or off depending
upon which packages are installed. It is considered deprecated, and is not used by default by any
current package manager. It is mentioned here for completeness only, and its format is not
discussed.

\subsection{Simple line-based files}
\label{line-stacking}
These files are a simple one-item-per-line list, which is inherited in the following manner: the
parent profile's list is taken, and the current profile's list appended. If any line begins with a
hyphen, then any lines previous to it whose contents are equal to the remainder of that line are
removed from the list. Once again, blank lines and those beginning with a \# are discarded.

\subsection{packages}
The \t{packages} file is used to define the `system set' for this profile.
After the above rules for inheritance and comments are applied, its lines must take one of two
forms: a package dependency specification prefixed by \t{*} denotes that the atom forms part of the
system set. A package dependency specification on its own may also appear for legacy reasons, but
should be ignored when calculating the system set.

\subsection{packages.build}
The \t{packages.build} file is used by Gentoo's Catalyst tool to generate stage1 tarballs, and has
no relevance to the operation of a package manager. It is thus outside the scope of this document,
but is mentioned here for completeness.

\subsection{package.mask}
\t{package.mask} is used to prevent packages from being installed on a given profile. Each line
contains one package dependency specification; anything matching this specification will not be
installed unless unmasked by the user's configuration.

Note that the \t{-spec} syntax can be used to remove a mask in a parent profile, but not
necessarily a global mask (from \t{profiles/package.mask}, section \ref{profiles-package.mask}).

\note Portage currently treats \t{profiles/package.mask} as being on the leftmost branch of the
    inherit tree when it comes to \t{-lines}. This behaviour may not be relied upon.

\subsection{package.provided}
\t{package.provided} is used to tell the package manager that a certain package version should be
considered to be provided by the system regardless of whether it is actually installed. Because it
has severe adverse effects on USE-based and slot-based dependencies, its use is strongly deprecated
and package manager support must be regarded as purely optional.

\subsection{USE masking and forcing}
\label{use-masking}
This section covers the four files \t{use.mask}, \t{use.force}, \t{package.use.mask} and
\t{package.use.force}. They are described together because they interact in a non-trivial manner.

Simply speaking, \t{use.mask} and \t{use.force} are used to say that a given USE flag must never or
always, respectively, be enabled when using this profile. \t{package.use.mask} and
\t{package.use.force} do the same thing on a per-package, or per-version, basis. The precise manner
in which they interact is less simple, and is best described in terms of the algorithm used to
determine whether a flag is masked for a given package version. This is described in Algorithm
\ref{alg:use-masking}.
\begin{algorithm}
\caption{USE masking logic} \label{alg:use-masking}
\begin{algorithmic}[1]
\STATE let masked = false
\FOR{each profile in the inheritance stack, ending with the current one}
    \IF{\t{use.mask} contains \i{flag}}
        \STATE let masked = true
    \ELSIF{\t{use.mask} contains \i{-flag}}
        \STATE let masked = false
    \ENDIF
    \FOR{each $line$ in package.use.mask, in order, for which the spec matches $package$}
        \IF{$line$ contains \i{flag}}
            \STATE let masked = true
        \ELSIF{$line$ contains \i{-flag}}
            \STATE let masked = false
        \ENDIF
    \ENDFOR
\ENDFOR
\end{algorithmic}
\end{algorithm}

The logic for \t{use.force} and \t{package.use.force} is identical. If a flag is both masked and
forced, the mask is considered to take precedence.

\section{Profile variables}
\label{profile-variables}

This section documents variables that have special meaning, or special behaviour, when defined in a
profile's \t{make.defaults} file.

\subsection{Incremental Variables}
\i{Incremental} variables must stack between parent and child profiles in the following manner:
Beginning with the highest parent profile, tokenise the variable's value based on whitespace and
concatenate the lists. Then, for any token $T$ beginning with a hyphen, remove it and any previous
tokens whose value is equal to $T$ with the hyphen removed, or, if $T$ is equal to $-*$, remove
all previous values. Note that because of this treatment, the order of tokens in the final result is
arbitrary, not necessarily related to the order of tokens in any given profile. The following
variables must be treated in this fashion:
\begin{bulletlist}
\item \t{USE}
\item \t{USE\_EXPAND}
\item \t{USE\_EXPAND\_HIDDEN}
\item \t{CONFIG\_PROTECT}
\item \t{CONFIG\_PROTECT\_MASK}
\item Any variable whose name is listed in \t{USE\_EXPAND}
\end{bulletlist}

Other variables, except where they affect only package-manager-specific functionality (such as
Portage's \t{FEATURES} variable), should not be treated incrementally---later definitions should
completely override those in parent profiles.

\subsection{Specific variables and their meanings}
The following variables have specific meanings when set in profiles.
\begin{description}
\item[ARCH] The system's architecture. Must be a value listed in \t{profiles/arch.list}; see section
    \ref{arch.list} for more information. Must be equal to the primary \t{KEYWORD} for this profile.
\item[CONFIG\_PROTECT, CONFIG\_PROTECT\_MASK] Contain whitespace-delimited lists used to control the
    configuration file protection. Described more fully in chapter \ref{config-protect}.
\item[USE] Defines the list of default USE flags for this profile. Flags may be added or removed by
    the user's configuration.
\item[USE\_EXPAND] \label{use-expand} Defines a list of variables which are to be treated
    incrementally and whose contents are to be expanded into the USE variable as passed to ebuilds.
    Expansion is done as per Algorithm \ref{alg:use-expand}.
    \begin{algorithm}
    \caption{USE\_EXPAND logic} \label{alg:use-expand}
    \begin{algorithmic}
        \FOR{each variable $V$ listed in \t{USE\_EXPAND}}
            \FOR{each token $T$ in $V$}
                \STATE append $v$\_$T$ to \t{USE}, where $v$ is the lowercase of $V$
            \ENDFOR
        \ENDFOR
    \end{algorithmic}
    \end{algorithm}
    So, for example, if \t{USE\_EXPAND} contains `ALSA\_CARDS', and the \t{ALSA\_CARDS} variable
    contains `foo', `alsa\_cards\_foo' will be appended to \t{USE}.
\item[USE\_EXPAND\_HIDDEN] Contains a list of variables contained in \t{USE\_EXPAND} which are not
    to be displayed to the end user.
\end{description}

Any other variables set in \t{make.defaults} should be passed on into the ebuild environment as-is,
and are not required to be interpreted by the package manager.

% vim: set filetype=tex fileencoding=utf8 et tw=100 spell spelllang=en :


% vim: set filetype=tex fileencoding=utf8 et tw=100 spell spelllang=en :


\chapter{Old-Style Virtual Packages}
\label{sec:old-virtuals}

Old-style virtuals are pseudo-packages---they can be depended upon or
installed, but do not exist in the ebuild repository.  An old-style
virtual requires several things in the repository: at least one ebuild
must list the virtual in its \t{PROVIDE} variable, and there must be
at least one entry in a profiles \t{virtuals} file listing the default
provider for each profile---see sections~\ref{ebuild-var-provide} and
\ref{sec:profiles-virtuals} for specifics on these two. Old-style virtuals
require special handling as regards dependencies; this is described
below.

All old-style virtuals must use the category \t{virtual}. Not all packages using the \t{virtual}
category may be assumed to be old style virtuals.

\note A \i{new-style} virtual is simply an ebuild which install no files and use its dependency
strings to select providers. By convention, and to ease migration, these are also placed in the
\t{virtual} category.

\section{Dependencies on virtual packages}

When a dependency on a virtual package is encountered, it must be
resolved into a real package before it can be satisfied. There are two
factors that affect this process: whether a package providing the
virtual is installed, and the \t{virtuals} file in the active profile
(section~\ref{sec:profiles-virtuals}). If a package is already installed
which satisfies the virtual requirement (via \t{PROVIDE}), then it
should be used to satisfy the dependency. Otherwise, the profiles
\t{virtuals} file (section~\ref{sec:profiles-virtuals}) should be
consulted to choose an appropriate provider.

Dependencies on old style virtuals must not use any kind of version restriction.

Blocks on provided virtuals have special behaviour documented in section~\ref{provided-blocks}.

% vim: set filetype=tex fileencoding=utf8 et tw=100 spell spelllang=en :

%%% Local Variables:
%%% mode: latex
%%% TeX-master: "pms"
%%% End:


\chapter{Ebuild File Format}
\label{ebuild-format}

The ebuild file format is in its basic form a subset of the format of a bash script. The interpreter
is assumed to be GNU bash, version 3.0 or later. The file encoding must be UTF-8 with Unix-style
newlines. When sourced, the
ebuild must define certain variables and functions (see sections \ref{ebuild-vars} and
\ref{ebuild-functions} for specific information), and must not call any external programs, write
anything to standard output or standard error, or modify the state of the system in any way.

% vim: set filetype=tex fileencoding=utf8 et tw=100 spell spelllang=en :


\chapter{Ebuild-defined Variables}
\label{ebuild-vars}

\note This section describes variables that may or must be defined by ebuilds. For
variables that are passed from the package manager to the ebuild, see section \ref{ebuild-env-vars}.

\section{Mandatory Ebuild-defined Variables}

All ebuilds must define at least the following variables:

\begin{description}
\item[DESCRIPTION] A short human-readable description of the package's purpose. May be defined by an
    eclass.
\item[SRC\_URI] A list of source URIs for the package. Valid protocols are \t{http://},
    \t{https://}, \t{ftp://} and \t{mirror://\} (see section \{thirdpartymirrors} for mirror behaviour).
    Fetch restricted packages may include URL parts consisting of just a filename.  Use-conditional and
    all-of groups are allowed (see section \ref{dependencies}); other constructs,
    such as any-of groups, are forbidden. May be defined by an eclass, and may be empty.
\item[HOMEPAGE] The package's homepage or (whitespace separated) homepages, including protocol. May
    be defined by an eclass.
\item[SLOT] The package's slot. Must be a valid slot name, as per section \ref{slot-names}. May
    be defined by an eclass.
\item[LICENSE] The package's license. Each text token must correspond to a tree ``licenses/`` entry
    (see section \ref{licenses-dir}). Use-conditional, any-of and all-of groups are allowed.
    May be defined by an eclass.
\item[IUSE] The \t{USE} flags used by the ebuild. Historically, \t{USE\_EXPAND} values and \t{ARCH}
    were not included; package managers should support this for backwards compatibility reasons. Ebuilds
    should list only flags used by the ebuild itself. Any eclass that works with \t{USE} flags
    should also set \t{IUSE}, listing only the variables used by that eclass. The package manager is
    responsible for merging these values.
\item[KEYWORDS] A whitespace separated list of keywords for the ebuild. Each token must be a
    valid keyword name, as per section \ref{keyword-names}. May include $-*$, which
    indicates that the package will only work on explicitly listed archs. May include $-arch$,
    which indicates that the package will not work on the specified arch.
\end{description}

If any of these variables are undefined, or if any of these variables are set to invalid values,
the package manager's behaviour is undefined; ideally, an error in one ebuild should not prevent
operations upon other ebuilds or packages.

\section{Optional Ebuild-defined Variables}

Ebuilds may define any of the following variables:

\begin{description}
\item[S] The path to the temporary build directory, used by \t{src\_compile}, \t{src\_install}
    etc. Defaults to \t{\$\{WORKDIR\}/\$\{P\}}.
\item[DEPEND] See section \ref{dependencies}.
\item[RDEPEND] See section \ref{dependencies}.
\item[PDEPEND] See section \ref{dependencies}.
\item[PROVIDE] Any \e{old style} virtuals provided by this package. May contain use conditionals.
    \label{ebuild-var-provide}
\item[EAPI] The EAPI, by default \t{0}.
\item[INHERITED] List of inherited eclass names. This is handled magically by \t{inherit} and
    shouldn't be modified manually.
\item[RESTRICT] Zero or more of \t{mirror}, \t{fetch}, \t{strip}, \t{userpriv}, \t{test}.
    Package managers may recognise other values, but ebuilds may not rely upon them being
    supported.
\end{description}

If any of these variables are set to invalid values, the package manager's behaviour is undefined;
ideally, an error in one ebuild should not prevent operations upon other ebuilds or packages.

\section{Dependencies}
\label{dependencies}

There are three categories of dependencies supported by ebuilds:

\begin{bulletlist}
\item Build dependencies (\t{DEPEND}). These must be installed before the ebuild is installed.
\item Runtime dependencies (\t{RDEPEND}). These should usually be installed before the ebuild,
    but may be dropped to post dependencies where necessary to resolve cycles.
\item Post dependencies (\t{PDEPEND}). These should be installed at some point, usually after
    the ebuild if they are not already installed.
\end{bulletlist}

The dependency specification format is a string containing zero or more of the following
items separated by whitespace:

\begin{bulletlist}
\item A package dependency specification (for \t{DEPEND} etc.), or a URI (for \t{SRC\_URI}),
    or a package name (for \t{PROVIDE}), or a license name (for \t{LICENSE}).
\item An all-of group, which consists of an open parenthesis, followed by whitespace,
    followed by zero or more dependency items of any kind, followed by whitespace, followed
    by a close parenthesis.
\item An any-of group, which consists of the string \t{||}, followed by whitespace,
    followed by an open parenthesis, followed by zero or more dependency items of any kind,
    followed by whitespace, followed by a close parenthesis.
\item A use-conditional group, which consists of an optional exclamation mark, followed by
    a use flag name, followed by a question mark, followed by whitespace, followed by
    an open parenthesis, followed by zero or more dependency items of any kind, followed by
    whitespace, followed by a close parenthesis.
\end{bulletlist}

A package dependency can be in one of the following base formats:

\begin{bulletlist}
\item A simple \t{category/package} name.
\item An operator, followed immediately by \t{category/package}, followed by a hyphen,
   followed by a version specification.
\end{bulletlist}

The following operators are available:

\begin{description}
\item[\t{<}] Strictly less than the specified version.
\item[\t{<=}] Less than or equal to the specified version.
\item[\t{=}] Exactly equal to the specified version. Special exception: if the version
    specified has an asterisk immediately following it, a string prefix comparison is
    used instead.
\item[\t{\~}] Equal to the specified version, except the revision part of the matching
    package may be greater than the revision part of the specified version.
\item[\t{>=}] Greater than or equal to the specified version.
\item[\t{>}] Strictly greater than the specified version.
\end{description}

If the operator is prefixed with an exclamation mark, the named dependency is a block
rather than a requirement -- that is to say, the specified package must not be
installed, except with the following exceptions:

\begin{bulletlist}
\item Blocks on a package provided by the ebuild do not count.
\item Blocks on the ebuild itself do not count.
\end{bulletlist}

In an all-of group, all of the child elements must be installed.

In a use-conditional group, if the use flag is enabled (or disabled if it has an exclamation mark
prefix), all of the child elements must be installed.

In an any-of group, at least one child element must be installed. An empty any-of group
counts as being installed. Any use-conditional group that is an immediate child of an
any-of group, if not enabled, is not considered a member of the any-of group.

% vim: set filetype=tex fileencoding=utf8 et tw=100 spell spelllang=en :


\chapter{Dependencies}
\label{dependencies}

\section{Dependency Classes}

There are three classes of dependencies supported by ebuilds:

\begin{bulletlist}
\item Build dependencies (\t{DEPEND}). These must be installed before the ebuild is installed.
\item Runtime dependencies (\t{RDEPEND}). These should usually be installed before the ebuild,
    but may be dropped to post dependencies where necessary to resolve cycles.
\item Post dependencies (\t{PDEPEND}). These should be installed at some point, usually after
    the ebuild if they are not already installed.
\end{bulletlist}

In addition, \t{SRC\_URI}, \t{PROVIDE} and \t{LICENSE} use the dependency specification format
to specify their values.

\note The term 'dependency specification' is perhaps not the best name, but it is the name
    most easily recognised by developers

\section{Dependency Specification Format}

The dependency specification format is a string containing zero or more of the following
items separated by whitespace:

\begin{bulletlist}
\item A package dependency specification (for \t{DEPEND} etc.), or a URI (for \t{SRC\_URI}),
    or a package name (for \t{PROVIDE}), or a license name (for \t{LICENSE}).
\item An all-of group, which consists of an open parenthesis, followed by whitespace,
    followed by zero or more dependency items of any kind, followed by whitespace, followed
    by a close parenthesis.
\item An any-of group, which consists of the string \t{||}, followed by whitespace,
    followed by an open parenthesis, followed by zero or more dependency items of any kind,
    followed by whitespace, followed by a close parenthesis.
\item A use-conditional group, which consists of an optional exclamation mark, followed by
    a use flag name, followed by a question mark, followed by whitespace, followed by
    an open parenthesis, followed by zero or more dependency items of any kind, followed by
    whitespace, followed by a close parenthesis.
\end{bulletlist}

In particular, note that whitespace is not optional.

\subsection{Package Dependency Specifications}

In EAPI-0, a package dependency can be in one of the following base formats:

\begin{bulletlist}
\item A simple \t{category/package} name.
\item An operator, followed immediately by \t{category/package}, followed by a hyphen,
   followed by a version specification.
\end{bulletlist}

\TODOBUG{170161}{Should slot dependencies be included in EAPI-0?}

The following operators are available:

\begin{description}
\item[\t{<}] Strictly less than the specified version.
\item[\t{<=}] Less than or equal to the specified version.
\item[\t{=}] Exactly equal to the specified version. Special exception: if the version
    specified has an asterisk immediately following it, a string prefix comparison is
    used instead. (An asterisk used with any other operator is illegal.)
\item[\t{\~}] Equal to the specified version, except the revision part of the matching
    package may be greater than the revision part of the specified version (\t{-r0} is
    assumed if no revision is explicitly stated).
\item[\t{>=}] Greater than or equal to the specified version.
\item[\t{>}] Strictly greater than the specified version.
\end{description}

If the operator is prefixed with an exclamation mark, the named dependency is a block
rather than a requirement -- that is to say, the specified package must not be
installed, except with the following exceptions:

\begin{bulletlist}
\item Blocks on a package provided by the ebuild do not count.
\item Blocks on the ebuild itself do not count.
\end{bulletlist}

\subsection{All-of Dependency Specifications}

In an all-of group, all of the child elements must be matched.

\subsection{Use-conditional Dependency Specifications}

In a use-conditional group, if the associated use flag is enabled (or disabled if it has an
exclamation mark prefix), all of the child elements must be matched.

\subsection{Any-of Dependency Specifications}

Any-of dependency specifications only make sense for dependencies and licenses.

Any use-conditional group that is an immediate child of an any-of group, if not enabled (disabled
for an exclamation mark prefixed use flag name), is not considered a member of the any-of group
for dependency resolution purposes.

In an any-of group, at least one immediate child element must be matched. A blocker is
considered to be matched if it is not installed (or, for licenses, matched).

An empty any-of group counts as being installed (or, for licenses, matched).

% vim: set filetype=tex fileencoding=utf8 et tw=100 spell spelllang=en :



\chapter{Ebuild-defined Functions}
\label{sec:ebuild-functions}

\section{List of Functions}
\label{sec:functions}

The following is a list of functions that an ebuild, or eclass, may define, and which will be called
by the package manager as part of the build and/or install process. In all cases the package manager
must provide a default implementation of these functions; unless otherwise stated this must be a
no-op. Most functions must assume only that they have write access to the package's working
directory (the \t{WORKDIR} environment variable; see section~\ref{env-var-WORKDIR}), and the
temporary directory \t{T}; exceptions are noted below. All functions may assume that they have read
access to all system libraries, binaries and configuration files that are accessible to normal
users.

The environment for functions run outside of the build sequence (that is, \t{pkg\_config},
\t{pkg\_info}, \t{pkg\_prerm} and \t{pkg\_postrm}) must be the environment used for the build of the
package, not the current configuration.

Ebuilds must not call nor assume the existence of any phase functions.

\subsection{Initial Working Directories}
\label{sec:s-to-workdir-fallback}

Some functions may assume that their initial working directory is set to a particular location;
these are noted below. If no initial working directory is mandated, it may be set to anything and
the ebuild must not rely upon a particular location for it. The ebuild \e{may} assume that the
initial working directory for any phase is a trusted location that may only be written to by a
privileged user and group.

\featurelabel{s-workdir-fallback} Some functions are described as having an initial working
directory of \t{S} with an error or fallback to \t{WORKDIR}. For EAPIs listed in
table~\ref{tab:s-fallback-table} as having the fallback, this means that if \t{S} is not a directory
before the start of the phase function, the initial working directory shall be \t{WORKDIR} instead.
For EAPIs where it is a conditional error, if \t{S} is not a directory before the start of the phase
function, it is a fatal error, unless all of the following conditions are true, in which case the
fallback to \t{WORKDIR} is used:

\begin{compactitem}
\item The \t{A} variable contains no items.
\item The phase function in question is not in \t{DEFINED\_PHASES}.
\item None of the phase functions \t{unpack}, \t{prepare}, \t{configure}, \t{compile} or \t{install},
    if supported by the EAPI in question and occurring prior to the phase about to be executed, are
    in \t{DEFINED\_PHASES}.
\end{compactitem}

\begin{centertable}{EAPIs with \t{S} to \t{WORKDIR} fallbacks} \label{tab:s-fallback-table}
\IFKDEBUILDELSE
{
    \begin{tabular}{ l l }
        \toprule
        \multicolumn{1}{c}{\textbf{EAPI}} &
        \multicolumn{1}{c}{\textbf{Fallback to \t{WORKDIR} permitted?}} \\
        \midrule
    \t{0} & Always \\
    \t{1} & Always \\
    \t{kdebuild-1} & Always \\
    \t{2} & Always \\
    \t{3} & Conditional error \\
    \bottomrule
    \end{tabular}
}{
    \begin{tabular}{ l l }
        \toprule
        \multicolumn{1}{c}{\textbf{EAPI}} &
        \multicolumn{1}{c}{\textbf{Fallback to \t{WORKDIR} permitted?}} \\
        \midrule
    \t{0} & Always \\
    \t{1} & Always \\
    \t{2} & Always \\
    \t{3} & Conditional error \\
    \bottomrule
    \end{tabular}
}
\end{centertable}

\subsection{pkg\_pretend}
\label{sec:pkg-pretend-function}

\featurelabel{pkg-pretend} The \t{pkg\_pretend} function is only called for EAPIs listed in
table~\ref{tab:pkg-pretend-table} as supporting it.

The \t{pkg\_pretend} function may be used to carry out sanity checks early on in the install
process. For example, if an ebuild requires a particular kernel configuration, it may perform that
check in \t{pkg\_pretend} and call \t{eerror} and then \t{die} with appropriate messages if the
requirement is not met.

\t{pkg\_pretend} is run separately from the main phase function sequence, and does not participate
in any kind of environment saving. There is no guarantee that any of an ebuild's dependencies will
be met at this stage, and no guarantee that the system state will not have changed substantially
before the next phase is executed.

\t{pkg\_pretend} must not write to the filesystem.

\begin{centertable}{EAPIs supporting \t{pkg\_pretend}} \label{tab:pkg-pretend-table}
\IFKDEBUILDELSE
{
    \begin{tabular}{ l l }
        \toprule
        \multicolumn{1}{c}{\textbf{EAPI}} &
        \multicolumn{1}{c}{\textbf{Supports \t{pkg\_pretend}?}} \\
        \midrule
    \t{0} & No \\
    \t{1} & No \\
    \t{kdebuild-1} & No \\
    \t{2} & No \\
    \t{3} & Yes \\
    \bottomrule
    \end{tabular}
}{
    \begin{tabular}{ l l }
        \toprule
        \multicolumn{1}{c}{\textbf{EAPI}} &
        \multicolumn{1}{c}{\textbf{Supports \t{pkg\_pretend}?}} \\
        \midrule
    \t{0} & No \\
    \t{1} & No \\
    \t{2} & No \\
    \t{3} & Yes \\
    \bottomrule
    \end{tabular}
}
\end{centertable}

\subsection{pkg\_setup}
\label{sec:pkg-setup-function}
The \t{pkg\_setup} function sets up the ebuild's environment for all following functions, before
the build process starts. Further, it checks whether any necessary prerequisites not covered
by the package manager, e.g.\ that certain kernel configuration options are fulfilled.

\t{pkg\_setup} must be run with full filesystem permissions, including the ability to add new users
and/or groups to the system.

\subsection{src\_unpack}
\label{sec:src-unpack-function}

\featurelabel{src-unpack} The \t{src\_unpack} function extracts all of the package's sources,
applies patches and sets up the package's build system for further use.

The initial working directory must be \t{WORKDIR}, and the default implementation used when
the ebuild lacks the \t{src\_unpack} function shall behave as:

\begin{verbatim}
src_unpack() {
    if [[ -n ${A} ]]; then
        unpack ${A}
    fi
}
\end{verbatim}

\subsection{src\_prepare}
\label{sec:src-prepare-function}

\featurelabel{src-prepare} The \t{src\_prepare} function is only called for EAPIs listed in
table~\ref{tab:src-prepare-table} as supporting it.

The \t{src\_prepare} function can be used for post-unpack source preparation. The default
implementation does nothing.

The initial working directory is \t{S}, with an error or fallback to \t{WORKDIR} as discussed in
section~\ref{sec:s-to-workdir-fallback}.

\begin{centertable}{EAPIs supporting \t{src\_prepare}} \label{tab:src-prepare-table}
\IFKDEBUILDELSE
{
    \begin{tabular}{ l l }
        \toprule
        \multicolumn{1}{c}{\textbf{EAPI}} &
        \multicolumn{1}{c}{\textbf{Supports \t{src\_prepare}?}} \\
        \midrule
    \t{0} & No \\
    \t{1} & No \\
    \t{kdebuild-1} & No \\
    \t{2} & Yes \\
    \t{3} & Yes \\
    \bottomrule
    \end{tabular}
}{
    \begin{tabular}{ l l }
        \toprule
        \multicolumn{1}{c}{\textbf{EAPI}} &
        \multicolumn{1}{c}{\textbf{Supports \t{src\_prepare}?}} \\
        \midrule
    \t{0} & No \\
    \t{1} & No \\
    \t{2} & Yes \\
    \t{3} & Yes \\
    \bottomrule
    \end{tabular}
}
\end{centertable}

\subsection{src\_configure}
\label{sec:src-configure-function}

\featurelabel{src-configure} The \t{src\_configure} function is only called for EAPIs listed in
table~\ref{tab:src-configure-table} as supporting it.

The initial working directory is \t{S}, with an error or fallback to \t{WORKDIR} as discussed in
section~\ref{sec:s-to-workdir-fallback}.

The \t{src\_configure} function configures the package's build environment. The default
implementation used when the ebuild lacks the \t{src\_configure} function shall behave as:

\begin{verbatim}
src_configure() {
    if [[ -x ${ECONF_SOURCE:-.}/configure ]]; then
        econf
    fi
}
\end{verbatim}

\begin{centertable}{EAPIs supporting \t{src\_configure}} \label{tab:src-configure-table}
\IFKDEBUILDELSE
{
    \begin{tabular}{ l l }
        \toprule
        \multicolumn{1}{c}{\textbf{EAPI}} &
        \multicolumn{1}{c}{\textbf{Supports \t{src\_configure}?}} \\
        \midrule
    \t{0} & No \\
    \t{1} & No \\
    \t{kdebuild-1} & No \\
    \t{2} & Yes \\
    \t{3} & Yes \\
    \bottomrule
    \end{tabular}
}{
    \begin{tabular}{ l l }
        \toprule
        \multicolumn{1}{c}{\textbf{EAPI}} &
        \multicolumn{1}{c}{\textbf{Supports \t{src\_configure}?}} \\
        \midrule
    \t{0} & No \\
    \t{1} & No \\
    \t{2} & Yes \\
    \t{3} & Yes \\
    \bottomrule
    \end{tabular}
}
\end{centertable}

\subsection{src\_compile}
\label{sec:src-compile-function}

\featurelabel{src-compile} The \t{src\_compile} function configures the package's build environment
in EAPIs lacking \t{src\_configure}, and builds the package in all EAPIs.

The initial working directory is \t{S}, with an error or fallback to \t{WORKDIR} as discussed in
section~\ref{sec:s-to-workdir-fallback}.

\featurelabel{src-compile-0} For EAPIs listed in table~\ref{tab:src-compile-table} as using format
0, the default implementation used when the ebuild lacks the \t{src\_compile} function shall behave
as:

\begin{verbatim}
src_compile() {
    if [[ -x ./configure ]]; then
        econf
    fi
    if [[ -f Makefile ]] || [[ -f GNUmakefile ]] || [[ -f makefile ]]; then
        emake || die "emake failed"
    fi
}
\end{verbatim}

\featurelabel{src-compile-1} For EAPIs listed in table~\ref{tab:src-compile-table} as using format
1, the default implementation used when the ebuild lacks the \t{src\_compile} function shall behave
as:

\begin{verbatim}
src_compile() {
    if [[ -x ${ECONF_SOURCE:-.}/configure ]]; then
        econf
    fi
    if [[ -f Makefile ]] || [[ -f GNUmakefile ]] || [[ -f makefile ]]; then
        emake || die "emake failed"
    fi
}
\end{verbatim}

\featurelabel{src-compile-2} For EAPIs listed in table~\ref{tab:src-compile-table} as using format
2, the default implementation used when the ebuild lacks the \t{src\_compile} function shall behave
as:

\begin{verbatim}
src_compile() {
    if [[ -f Makefile ]] || [[ -f GNUmakefile ]] || [[ -f makefile ]]; then
        emake || die "emake failed"
    fi
}
\end{verbatim}

\begin{centertable}{\t{src\_compile} behaviour for EAPIs} \label{tab:src-compile-table}
\IFKDEBUILDELSE
{
    \begin{tabular}{ l l }
        \toprule
        \multicolumn{1}{c}{\textbf{EAPI}} &
        \multicolumn{1}{c}{\textbf{Format}} \\
        \midrule
    \t{0} & 0 \\
    \t{1} & 1 \\
    \t{kdebuild-1} & 1 \\
    \t{2} & 2 \\
    \t{3} & 2 \\
    \bottomrule
    \end{tabular}
}{
    \begin{tabular}{ l l }
        \toprule
        \multicolumn{1}{c}{\textbf{EAPI}} &
        \multicolumn{1}{c}{\textbf{Format}} \\
        \midrule
    \t{0} & 0 \\
    \t{1} & 1 \\
    \t{2} & 2 \\
    \t{3} & 2 \\
    \bottomrule
    \end{tabular}
}
\end{centertable}

\subsection{src\_test}
\label{sec:src-test-function}

The \t{src\_test} function runs unit tests for the newly built but not yet installed package as
provided.

The initial working directory must be \t{S} if that exists, falling back to \t{WORKDIR} otherwise.
The default implementation used when the ebuild lacks the \t{src\_test} function must, if tests are
enabled, run \t{make check} if and only if such a target is available, or if not run \t{make test},
if and only such a target is available. In both cases, if make returns non-zero the build must be
aborted.

The \t{src\_test} function may be disabled by \t{RESTRICT}. See section~\ref{sec:restrict}.

\IFKDEBUILDELSE
{
    \featurelabel{src-test-required} In some EAPIs, \t{src\_test} should only be run at user option
    (and never if restrictions are in place). In others, it must always be run (excepting
    restrictions). See table~\ref{tab:test-required-table} for which EAPIs fit into which category.

    \begin{centertable}{EAPIs requiring \t{src\_test}} \label{tab:test-required-table}
    \begin{tabular}{ l l }
        \toprule
        \multicolumn{1}{c}{\textbf{EAPI}} &
        \multicolumn{1}{c}{\textbf{Requires \t{src\_test}?}} \\
        \midrule
    \t{0} & At user option \\
    \t{1} & At user option \\
    \t{kdebuild-1} & Required \\
    \t{2} & At user option \\
    \t{3} & At user option \\
    \bottomrule
    \end{tabular}

    \end{centertable}
}{
}

\subsection{src\_install}
\label{sec:src-install-function}

\featurelabel{src-install} The \t{src\_install} function installs the package's content to a
directory specified in \t{D}.

The initial working directory is \t{S}, with an error or fallback to \t{WORKDIR} as discussed in
section~\ref{sec:s-to-workdir-fallback}.

\featurelabel{src-install-3} For EAPIs listed in table~\ref{tab:src-install-table} as using format
3, the default implementation used when the ebuild lacks the \t{src\_install} function shall behave
as:

\begin{verbatim}
src_install() {
    if [[ -f Makefile ]] || [[ -f GNUmakefile ]] || [[ -f makefile ]]; then
        emake DESTDIR="${D}" install
    fi

    if ! declare -p DOCS >/dev/null 2>&1 ; then
        local d
        for d in README* ChangeLog AUTHORS NEWS TODO CHANGES \
                THANKS BUGS FAQ CREDITS CHANGELOG ; do
            [[ -s "${d}" ]] && dodoc "${d}"
        done
    elif declare -p DOCS | grep -q '^declare -a ' ; then
        dodoc "${DOCS[@]}"
    else
        dodoc ${DOCS}
    fi
}
\end{verbatim}

For other EAPIs, the default implementation used when the ebuild lacks the \t{src\_install} function
is a no-op.

\begin{centertable}{\t{src\_install} behaviour for EAPIs} \label{tab:src-install-table}
\IFKDEBUILDELSE
{
    \begin{tabular}{ l l }
        \toprule
        \multicolumn{1}{c}{\textbf{EAPI}} &
        \multicolumn{1}{c}{\textbf{Format}} \\
        \midrule
    \t{0} & no-op \\
    \t{1} & no-op \\
    \t{kdebuild-1} & no-op \\
    \t{2} & no-op \\
    \t{3} & 3 \\
    \bottomrule
    \end{tabular}
}{
    \begin{tabular}{ l l }
        \toprule
        \multicolumn{1}{c}{\textbf{EAPI}} &
        \multicolumn{1}{c}{\textbf{Format}} \\
        \midrule
    \t{0} & no-op \\
    \t{1} & no-op \\
    \t{2} & no-op \\
    \t{3} & 3 \\
    \bottomrule
    \end{tabular}
}
\end{centertable}

\subsection{pkg\_preinst}
\label{sec:pkg-preinst-function}

The \t{pkg\_preinst} function performs any special tasks that are required immediately before
merging the package to the live filesystem. It must not write outside of the directories specified
by the \t{ROOT} and \t{D} environment variables.

\t{pkg\_preinst} must be run with full access to all files and directories below that specified by
the \t{ROOT} and \t{D} environment variables.

\subsection{pkg\_postinst}
\label{sec:pkg-postinst-function}

The \t{pkg\_postinst} function performs any special tasks that are required immediately after
merging the package to the live filesystem. It must not write outside of the directory specified
in the \t{ROOT} environment variable.

\t{pkg\_postinst}, like, \t{pkg\_preinst}, must be run with full access to all files and directories
below that specified by the \t{ROOT} environment variable.

\subsection{pkg\_prerm}
\label{sec:pkg-prerm-function}

The \t{pkg\_prerm} function performs any special tasks that are required immediately before
unmerging the package from the live filesystem. It must not write outside of the directory specified
by the \t{ROOT} environment variable.

\t{pkg\_prerm} must be run with full access to all files and directories below that specified by
the \t{ROOT} environment variable.

\subsection{pkg\_postrm}
\label{sec:pkg-postrm-function}

The \t{pkg\_postrm} function performs any special tasks that are required immediately after
unmerging the package from the live filesystem. It must not write outside of the directory specified
by the \t{ROOT} environment variable.

\t{pkg\_postrm} must be run with full access to all files and directories below that specified by
the \t{ROOT} environment variable.

\subsection{pkg\_config}
\label{sec:pkg-config-function}

The \t{pkg\_config} function performs any custom steps required to configure a package after it has been
fully installed. It is the only ebuild function which may be interactive and prompt for user input.

\t{pkg\_config} must be run with full access to all files and directories inside of \t{ROOT}.

\subsection{pkg\_info}
\label{sec:pkg-info-function}

\featurelabel{pkg-info} The \t{pkg\_info} function may be called by the package manager when
displaying information about an installed package. In EAPIs listed in table~\ref{tab:pkg-info-table}
as supporting \t{pkg\_info} on non-installed packages, it may also be called by the package manager
when displaying information about a non-installed package. In this case, ebuild authors should note
that dependencies may not be installed.

\t{pkg\_info} must not write to the filesystem.

\begin{centertable}{EAPIs supporting \t{pkg\_info} on non-installed packages} \label{tab:pkg-info-table}
\IFKDEBUILDELSE
{
    \begin{tabular}{ l l }
        \toprule
        \multicolumn{1}{c}{\textbf{EAPI}} &
        \multicolumn{1}{c}{\textbf{Supports \t{pkg\_info} on non-installed packages?}} \\
        \midrule
    \t{0} & No \\
    \t{1} & No \\
    \t{kdebuild-1} & Yes \\
    \t{2} & No \\
    \t{3} & Yes \\
    \bottomrule
    \end{tabular}
}{
    \begin{tabular}{ l l }
        \toprule
        \multicolumn{1}{c}{\textbf{EAPI}} &
        \multicolumn{1}{c}{\textbf{Supports \t{pkg\_info} on non-installed packages?}} \\
        \midrule
    \t{0} & No \\
    \t{1} & No \\
    \t{2} & No \\
    \t{3} & Yes \\
    \bottomrule
    \end{tabular}
}
\end{centertable}

\subsection{pkg\_nofetch}
\label{sec:pkg-nofetch-function}

The \t{pkg\_nofetch} function is run when the fetch phase of an fetch-restricted ebuild is run, and
the relevant source files are not available. It should direct the user to download all relevant
source files from their respective locations, with notes concerning licensing if applicable.

\t{pkg\_nofetch} must require no write access to any part of the filesystem.

\subsection{\t{default\_} Phase Functions}
\label{sec:default-phase-functions}

\featurelabel{default-phase-functions} In EAPIs listed in
table~\ref{tab:default-phase-function-table} as supporting \t{default\_} phase functions, a function
named \t{default\_}(phase) that behaves as the default implementation for that EAPI shall be defined
when executing any ebuild phase listed in the table. Ebuilds must not call these functions except
when in the phase in question.

\begin{centertable}{EAPIs supporting \t{default\_} phase functions} \label{tab:default-phase-function-table}
\IFKDEBUILDELSE
{
    \begin{tabular}{ l l }
        \toprule
            \multicolumn{1}{c}{\textbf{EAPI}} &
            \multicolumn{1}{c}{\textbf{Supports \t{default\_} functions in phases}} \\
            \midrule
    \t{0} & None \\
    \t{1} & None \\
    \t{kdebuild-1} & None \\
    \t{2} & \parbox[t]{3in}{\t{pkg\_nofetch}, \t{src\_unpack}, \t{src\_prepare}, \t{src\_configure},
        \t{src\_compile}, \t{src\_test}} \\
    \t{3} & \parbox[t]{3in}{\t{pkg\_nofetch}, \t{src\_unpack}, \t{src\_prepare}, \t{src\_configure},
        \t{src\_compile}, \t{src\_install}, \t{src\_test}} \\
    \bottomrule
    \end{tabular}
}{
    \begin{tabular}{ l l }
        \toprule
            \multicolumn{1}{c}{\textbf{EAPI}} &
            \multicolumn{1}{c}{\textbf{Supports \t{default\_} functions in phases}} \\
            \midrule
    \t{0} & None \\
    \t{1} & None \\
    \t{2} & \parbox[t]{3in}{\t{pkg\_nofetch}, \t{src\_unpack}, \t{src\_prepare}, \t{src\_configure},
        \t{src\_compile}, \t{src\_test}} \\
    \t{3} & \parbox[t]{3in}{\t{pkg\_nofetch}, \t{src\_unpack}, \t{src\_prepare}, \t{src\_configure},
        \t{src\_compile}, \t{src\_install}, \t{src\_test}} \\
    \bottomrule
    \end{tabular}
}
\end{centertable}

\section{Call Order}

The call order for installing a package is:

\begin{compactitem}
\item \t{pkg\_pretend} (only for EAPIs listed in table~\ref{tab:pkg-pretend-table}), which is called
    outside of the normal call order process.
\item \t{pkg\_setup}
\item \t{src\_unpack}
\item \t{src\_prepare} (only for EAPIs listed in table~\ref{tab:src-prepare-table})
\item \t{src\_configure} (only for EAPIs listed in table~\ref{tab:src-configure-table})
\item \t{src\_compile}
\item \t{src\_test} (except if \t{RESTRICT=test})
\item \t{src\_install}
\item \t{pkg\_preinst}
\item \t{pkg\_postinst}
\end{compactitem}

The call order for uninstalling a package is:

\begin{compactitem}
\item \t{pkg\_prerm}
\item \t{pkg\_postrm}
\end{compactitem}

The call order for reinstalling a package is:

\begin{compactitem}
\item \t{pkg\_pretend} (only for EAPIs listed in table~\ref{tab:pkg-pretend-table}), which is called
    outside of the normal call order process.
\item \t{pkg\_setup}
\item \t{src\_unpack}
\item \t{src\_prepare} (only for EAPIs listed in table~\ref{tab:src-prepare-table})
\item \t{src\_configure} (only for EAPIs listed in table~\ref{tab:src-configure-table})
\item \t{src\_compile}
\item \t{src\_test} (except if \t{RESTRICT=test})
\item \t{src\_install}
\item \t{pkg\_preinst}
\item \t{pkg\_prerm} for the package being replaced
\item \t{pkg\_postrm} for the package being replaced
\item \t{pkg\_postinst}
\end{compactitem}

The call order for upgrading or downgrading a package is:

\begin{compactitem}
\item \t{pkg\_pretend} (only for EAPIs listed in table~\ref{tab:pkg-pretend-table}), which is called
    outside of the normal call order process.
\item \t{pkg\_setup}
\item \t{src\_unpack}
\item \t{src\_prepare} (only for EAPIs listed in table~\ref{tab:src-prepare-table})
\item \t{src\_configure} (only for EAPIs listed in table~\ref{tab:src-configure-table})
\item \t{src\_compile}
\item \t{src\_test} (except if \t{RESTRICT=test})
\item \t{src\_install}
\item \t{pkg\_preinst}
\item \t{pkg\_postinst}
\item \t{pkg\_prerm} for the package being replaced
\item \t{pkg\_postrm} for the package being replaced
\end{compactitem}

The \t{pkg\_config}, \t{pkg\_info} and \t{pkg\_nofetch} functions are not called in a normal
sequence. The \t{pkg\_pretend} function is called some unspecified time before a (possibly
hypothetical) normal sequence.

For installing binary packages, the \t{src} phases are not called.

When building binary packages that are not to be installed locally, the \t{pkg\_preinst}
and \t{pkg\_postinst} functions are not called.

% vim: set filetype=tex fileencoding=utf8 et tw=100 spell spelllang=en :

%%% Local Variables:
%%% mode: latex
%%% TeX-master: "pms"
%%% End:


\chapter{Eclasses}
\label{sec:eclasses}

Eclasses serve to store common code that is used by more than one ebuild, which greatly aids
maintainability and reduces the tree size. However, due to metadata cache issues, care must be taken
in their use. In format they are similar to an ebuild, and indeed are sourced as part of any ebuild
using them. The interpreter is therefore the same, and the same requirements for being parseable
hold.

Eclasses must be located in the \t{eclass} directory in the top level of the repository---see
section~\ref{sec:eclass-dir}. Each eclass is a single file named \t{<name>.eclass}, where \t{<name>} is
the name of this eclass, used by \t{inherit} and \t{EXPORT\_FUNCTIONS} among other places.

\section{The inherit command}
\label{sec:inherit}

An ebuild wishing to make use of an eclass does so by using the \t{inherit} command in global scope.
This will cause the eclass to be sourced as part of the ebuild---any function or variable
definitions in the eclass will appear as part of the ebuild, with exceptions for certain metadata
variables, as described below.

The \t{inherit} command takes one or more parameters, which must be the names of eclasses (excluding
the \t{.eclass} suffix and the path). For each parameter, in order, the named eclass is sourced.

Eclasses may end up being sourced multiple times.

The \t{inherit} command must also ensure that:

\begin{compactitem}
\item The \t{ECLASS} variable is set to the name of the current eclass, when sourcing that eclass.
\item Once all inheriting has been done, the \t{INHERITED} metadata variable contains the name of
    every eclass used, separated  by whitespace.
\end{compactitem}

\section{Eclass-defined Metadata Keys}

The \t{IUSE}, \t{DEPEND}, \t{RDEPEND} and \t{PDEPEND} variables are handled specially
when set by an eclass. They must be accumulated across eclasses, appending the value set by each
eclass to the resulting value after the previous one is loaded. Then the eclass-defined value is
appended to that defined by the ebuild. In the case of \t{RDEPEND}, this is done after the
implicit \t{RDEPEND} rules in section~\ref{sec:rdepend-depend} are applied.
\IFKDEBUILDELSE
{
    \featurelabel{pdepend-labels}
    In EAPIs shown in table~\ref{tab:pdepend-labels-table} as supporting \t{PDEPEND}
    labels, the values of \t{PDEPEND} from the ebuild and each eclass must be wrapped
    in parentheses, so that the labels only apply within the ebuild/eclass in which
    they appear.
}{
}

\section{EXPORT\_FUNCTIONS}

There is one command available in the eclass environment that is neither available nor meaningful
in ebuilds---\t{EXPORT\_FUNCTIONS}. This can be used to alias ebuild phase functions from the
eclass so that an ebuild inherits a default definition whilst retaining the ability to override and
call the eclass-defined version from it. The use of it is best illustrated by an example; this is
given in listing~\ref{lst:export-functions} and is a snippet from a hypothetical \t{foo.eclass}.

\begin{listing}
  \caption{EXPORT\_FUNCTIONS example: foo.eclass}\label{lst:export-functions}
  \begin{verbatim}
foo_src_compile()
{
    econf --enable-gerbil \
            $(use_enable fnord)
    emake gerbil || die "Couldn't make a gerbil"
    emake || die "emake failed"
}

EXPORT_FUNCTIONS src_compile
  \end{verbatim}
\end{listing}

This example defines an eclass \t{src\_compile} function and uses \t{EXPORT\_FUNCTIONS} to alias
it. Then any ebuild that inherits \t{foo.eclass} will have a default \t{src\_compile} defined, but
should the author wish to override it he can access the function in \t{foo.eclass} by calling
\t{foo\_src\_compile}.

\t{EXPORT\_FUNCTIONS} must only be used on ebuild phase functions. The function that is aliased
must be named \t{eclassname\_phasefunctionname}, where \t{eclassname} is the name of the eclass.

\t{EXPORT\_FUNCTIONS} must be used at most once per eclass.

% vim: set filetype=tex fileencoding=utf8 et tw=100 spell spelllang=en :

%%% Local Variables:
%%% mode: latex
%%% TeX-master: "pms"
%%% End:


\chapter{The Ebuild Environment}

\section{Defined Variables}
\label{sec:ebuild-env-vars}

The package manager must define the following environment variables. Not all variables are
meaningful in all phases; variables that are not meaningful in a given phase may be unset or set to
any value. Ebuilds must not attempt to modify any of these variables, unless otherwise specified.

Because of their special meanings, these variables may not be preserved consistently across all
phases as would normally happen due to environment saving (see~\ref{sec:ebuild-env-state}). For example,
\t{EBUILD\_PHASE} is different for every phase, and \t{ROOT} may have changed between the various
different \t{pkg\_*} phases. Ebuilds must recalculate any variable they derive from an inconsistent
variable.

\begin{landscape}
\begin{longtable}{l p{0.15\textwidth} l p{0.5\textwidth}}
\caption{Defined variables\label{tab:defined_vars}}\\
\toprule
\multicolumn{1}{c}{\b{Variable}} &
\multicolumn{1}{c}{\b{Legal in}} &
\multicolumn{1}{c}{\b{Consistent?}} &
\multicolumn{1}{c}{\b{Description}} \\
\midrule
\endfirsthead
\midrule
\multicolumn{1}{c}{\b{Variable}} &
\multicolumn{1}{c}{\b{Legal in}} &
\multicolumn{1}{c}{\b{Consistent?}} &
\multicolumn{1}{c}{\b{Description}} \\
\midrule
\endhead
\midrule
\endfoot
\bottomrule
\endlastfoot
\t{P} &
    all &
    No\footnote{May change if a package has been updated (see~\ref{sec:updates-dir})} &
    Package name and version, without the revision part. For example, \t{vim-7.0.174}. \\
\t{PN} &
    all &
    ditto &
    Package name, for example \t{vim}. \\
\t{CATEGORY} &
    all &
    ditto &
    The package's category, for example \t{app-editors}. \\
\t{PV} &
    all &
    Yes &
    Package version, with no revision. For example \t{7.0.174}. \\
\t{PR} &
    all &
    Yes &
    Package revision, or \t{r0} if none exists. \\
\t{PVR} &
    all &
    Yes &
    Package version and revision (if any), for example \t{7.0.174} or \t{7.0.174-r1}. \\
\t{PF} &
    all &
    Yes &
    Package name, version, and revision (if any), for example \t{vim-7.0.174-r1}. \\
\t{A} &
    \t{src\_*} &
    Yes &
    All source files available for the package, whitespace separated with no leading or trailing
    whitespace, and in the order in which the item first appears in a matched component of
    \t{SRC\_URI}\@. Does not include any that are disabled because of USE conditionals. The value is
    calculated from the base names of each element of the \t{SRC\_URI} ebuild metadata variable. \\
\t{AA}\footnote{This variable is generally considered deprecated. However, ebuilds must still
    assume that the package manager sets it. For example, a few configure scripts use this variable
    to find the \t{aalib} package; ebuilds calling such configure scripts must thus work around
    this.} &
    \t{src\_*} &
    Yes &
    \featurelabel{aa} All source files that could be available for the package, including any that
    are disabled in \t{A} because of USE conditionals. The value is calculated from the base names
    of each element of the \t{SRC\_URI} ebuild metadata variable. Only for EAPIs listed in
    table~\ref{tab:env-vars-table} as supporting \t{AA}. \\
\t{FILESDIR} &
    \t{src\_*}\footnote{Not necessarily present when installing from a binary package} &
    No &
    The full path to the package's files directory, used for small support files or
    patches. See section~\ref{sec:package-dirs}. May or may not exist; if a repository provides no
    support files for the package in question then an ebuild must be prepared for the situation
    where \t{FILESDIR} points to a non-existent directory. \\
\t{PORTDIR} &
    ditto &
    No &
    The full path to the master repository's base directory. \\
\t{DISTDIR} &
    ditto &
    No &
    The full path to the directory in which the files in the \t{A} variable are stored. \\
\t{ECLASSDIR} &
    ditto &
    No &
    The full path to the master repository's eclass directory. \\
\t{ROOT} &
   \t{pkg\_*} &
   No &
   The absolute path to the root directory into which the package is to be merged.  Phases which run
   with full filesystem access must not touch any files outside of the directory given in
   \t{ROOT}\@. Also of note is that in a cross-compiling environment, binaries inside of \t{ROOT}
   will not be executable on the build machine, so ebuilds must not call them. \t{ROOT} must be
   non-empty and end in a trailing slash. \\
\t{EROOT} &
    \t{pkg\_*} &
    No &
    Like \t{ROOT}, but with \t{EPREFIX} appended. This is a convenience variable. See also the
    \t{EPREFIX} variable. \\
\t{T} &
    All &
    Partially\footnote{Consistent and preserved across a single connected sequence of install or
    uninstall phases, but not between install and uninstall. When reinstalling a package, this
    variable must have different values for the install and the replacement.} &
    The full path to a temporary directory for use by the ebuild. \\
\t{TMPDIR} &
    All &
    Ditto &
    Must be set to the location of a usable temporary directory, for any applications
    called by an ebuild. Must not be used by ebuilds directly; see \t{T} above. \\
\t{HOME} &
    All &
    Ditto &
    The full path to an appropriate temporary directory for use by any programs invoked by the
    ebuild that may read or modify the home directory. \\
\t{EPREFIX} &
    All &
    Yes &
    The normalised offset-prefix path of an offset installation.  When \t{EPREFIX} is not set in the
    calling environment, \t{EPREFIX} defaults to the built-in offset-prefix that was set during
    installation of the package manager. When a different \t{EPREFIX} value than the built-in value is set
    in the calling environment, a cross-prefix build is performed where using the existing utilities, a
    package is built for the given \t{EPREFIX}, akin to \t{ROOT}. See also~\ref{sec:offset-vars}. \\
\t{D} &
    \t{src\_install} &
    No &
    Contains the full path to the image directory into which the package should be installed.
    Must be non-empty and end in a trailing slash. \\
\t{D} (continued) &
    \t{pkg\_preinst}, \t{pkg\_postinst} &
    Yes &
    Contains the full path to the image that is about to be or has just been merged. Must be
    non-empty and end in a trailing slash. \\
\t{ED} &
    \t{src\_install} &
    See \t{D} &
    Like \t{D}, but with \t{EPREFIX} appended. This is a convenience variable. See also the
    \t{EPREFIX} variable. \\
\t{IMAGE}\footnote{Deprecated in favour of \t{D}.} &
    \t{pkg\_preinst}, \t{pkg\_postinst} &
    Yes &
    Equal to \t{D}. \\
\t{INSDESTTREE} &
    \t{src\_install} &
    No &
    Controls the location where doins installs things. \\
\t{USE} &
    All &
    Yes &
    A whitespace-delimited list of all active USE flags for this ebuild. See
    section~\ref{sec:use-iuse-handling} for details. \\
\t{EBUILD\_PHASE} &
    All &
    No &
    Takes one of the values \t{config}, \t{setup}, \t{nofetch}, \t{unpack}, \t{prepare},
    \t{configure}, \t{compile}, \t{test}, \t{install}, \t{preinst}, \t{postinst}, \t{prerm},
    \t{postrm}, \t{info}, \t{pretend} according to the top level ebuild function that was executed
    by the package manager. May be unset or any single word that is not any of the above when the
    ebuild is being sourced for other (e.\,g.\ metadata or QA) purposes. \\
\t{WORKDIR} &
    \t{src\_*} &
    Yes &
    The full path to the ebuild's working directory, in which all build data should be
    contained. \label{env-var-WORKDIR} \\
\t{KV} &
    All &
    Yes &
    \featurelabel{kv} The version of the running kernel at the time the ebuild was first executed,
    as returned by the \t{uname~-r} command or equivalent.  May be modified by ebuilds.  Only for
    EAPIs listed in table~\ref{tab:env-vars-table} as supporting \t{KV}. \\
\t{REPLACING\_VERSIONS} &
    \t{pkg\_*} (see text) &
    Yes &
    A whitespace-separated list of versions of this package (including revision, if specified) that
    are being replaced (uninstalled or overwritten) as a result of this install. See
    section~\ref{sec:replacing-versions}.  Only for EAPIs listed in table~\ref{tab:env-vars-table}
    as supporting \t{REPLACING\_VERSIONS}. \\
\t{REPLACED\_BY\_VERSION} &
    \t{pkg\_prerm}, \t{pkg\_postrm} &
    Yes &
    The single version of this package (including revision, if specified) that is replacing us, if
    we are being uninstalled as part of an install, or an empty string otherwise. See
    section~\ref{sec:replacing-versions}.  Only for EAPIs listed in table~\ref{tab:env-vars-table}
    as supporting \t{REPLACED\_BY\_VERSION}.
\end{longtable}
\end{landscape}

\begin{centertable}{EAPIs supporting various env variables} \label{tab:env-vars-table}
    \begin{tabular}{ l l l l l }
        \toprule
        \multicolumn{1}{c}{\textbf{EAPI}} &
        \multicolumn{1}{c}{\textbf{\t{AA}?}} &
        \multicolumn{1}{c}{\textbf{\t{KV}?}} &
        \multicolumn{1}{c}{\textbf{\t{REPLACING\_VERSIONS}?}} &
        \multicolumn{1}{c}{\textbf{\t{REPLACED\_BY\_VERSION}?}} \\
        \midrule
    \t{0} & Yes & Yes & No & No \\
    \t{1} & Yes & Yes & No & No \\
    \t{2} & Yes & Yes & No & No \\
    \t{3} & Yes & Yes & No & No \\
    \t{4} & No & No & Yes & Yes \\
    \bottomrule
    \end{tabular}
\end{centertable}

\begin{centertable}{EAPIs supporting offset-prefix env variables}
    \label{tab:offset-env-vars-table}
    \begin{tabular}{ l l l l }
        \toprule
        \multicolumn{1}{c}{\textbf{EAPI}} &
        \multicolumn{1}{c}{\textbf{\t{EPREFIX}?}} &
        \multicolumn{1}{c}{\textbf{\t{EROOT}?}} &
        \multicolumn{1}{c}{\textbf{\t{ED}?}} \\
        \midrule
        \t{0} & No & No & No \\
        \t{1} & No & No & No \\
        \t{2} & No & No & No \\
        \t{3} & Yes & Yes & Yes \\
        \t{4} & Yes & Yes & Yes \\
        \bottomrule
    \end{tabular}
\end{centertable}

Except where otherwise noted, all variables set in the active profiles' \t{make.defaults} files must
be exported to the ebuild environment. \t{CHOST}, \t{CBUILD} and \t{CTARGET}, if not set by
profiles, must contain either an appropriate machine tuple (the definition of appropriate is beyond
the scope of this specification) or be unset.

\t{PATH} must be initialized by the package manager to a ``usable'' default.  The exact value here
is left up to interpretation, but it should include the equivalent ``sbin'' and ``bin'' and any
package manager specific directories.

\t{GZIP}, \t{BZIP}, \t{BZIP2}, \t{CDPATH}, \t{GREP\_OPTIONS}, \t{GREP\_COLOR} and \t{GLOBIGNORE}
must not be set.

\subsection{USE and IUSE Handling}
\label{sec:use-iuse-handling}

This section discusses the handling of four variables:

\begin{description}
\item[IUSE] is the variable calculated from the \t{IUSE} values defined in ebuilds and eclasses.
\item[IUSE\_REFERENCEABLE] is a variable calculated from \t{IUSE} and a variety of other sources
    described below. It is purely a conceptual variable; it is not exported to the ebuild
    environment. Values in \t{IUSE\_REFERENCEABLE} may legally be used in queries from other
    packages about an ebuild's state (for example, for use dependencies).
\item[IUSE\_EFFECTIVE] is another conceptual, unexported variable. Values in \t{IUSE\_EFFECTIVE} are
    those which an ebuild may legally use in queries about itself (for example, for the \t{use}
    function, and for use in dependency specification conditional blocks).
\item[USE] is a variable calculated by the package manager and exported to the ebuild environment.
\end{description}

In all cases, the values of \t{IUSE\_REFERENCEABLE} and \t{IUSE\_EFFECTIVE} are undefined during
metadata generation.

For EAPIs listed in table~\ref{tab:profile-iuse-injection-table} as not supporting profile defined
\t{IUSE} injection, \t{IUSE\_REFERENCEABLE} is equal to the calculated \t{IUSE} value. For EAPIs
where profile defined \t{IUSE} injection is supported, \t{IUSE\_REFERENCEABLE} is equal to
\t{IUSE\_EFFECTIVE}.

For EAPIs listed in table~\ref{tab:profile-iuse-injection-table} as not supporting profile defined
\t{IUSE} injection, \t{IUSE\_EFFECTIVE} contains the following values:

\begin{compactitem}
\item All values in the calculated \t{IUSE} value.
\item All possible values for the \t{ARCH} variable.
\item All legal use flag names whose name starts with the lowercase equivalent of any value in
    the profile \t{USE\_EXPAND} variable followed by an underscore.
\end{compactitem}

\featurelabel{profile-iuse-injection} For EAPIs listed in
table~\ref{tab:profile-iuse-injection-table} as supporting profile defined \t{IUSE} injection,
\t{IUSE\_EFFECTIVE} contains the following values:

\begin{compactitem}
\item All values in the calculated \t{IUSE} value.
\item All values in the profile \t{IUSE\_IMPLICIT} variable.
\item All values in the profile variable named \t{USE\_EXPAND\_VALUES\_\$\{v\}}, where \t{\$\{v\}}
    is any value in the intersection of the profile \t{USE\_EXPAND\_UNPREFIXED} and
    \t{USE\_EXPAND\_IMPLICIT} variables.
\item All values for \t{\$\{lower\_v\}\_\$\{x\}}, where \t{\$\{x\}} is all values in the profile
    variable named \t{USE\_EXPAND\_VALUES\_\$\{v\}}, where \t{\$\{v\}} is any value in the
    intersection of the profile \t{USE\_EXPAND} and \t{USE\_EXPAND\_IMPLICIT} variables and
    \t{\$\{lower\_v\}} is the lowercase equivalent of \t{\$\{v\}}.
\end{compactitem}

The \t{USE} variable is set by the package manager. For each value in \t{IUSE\_EFFECTIVE}, \t{USE}
shall contain that value if the flag is to be enabled for the ebuild in question, and shall not
contain that value if it is to be disabled. In EAPIs listed in
table~\ref{tab:profile-iuse-injection-table} as not supporting profile defined \t{IUSE} injection,
\t{USE} may contain other flag names that are not relevant for the ebuild.

For EAPIs listed in table~\ref{tab:profile-iuse-injection-table} as supporting profile defined
\t{IUSE} injection, the variables named in \t{USE\_EXPAND} and \t{USE\_EXPAND\_UNPREFIXED} shall
have their profile-provided values reduced to contain only those values that are present in
\t{IUSE\_EFFECTIVE}.

For EAPIs listed in table~\ref{tab:profile-iuse-injection-table} as supporting profile defined
\t{IUSE} injection, the package manager must save the calculated value of \t{IUSE\_EFFECTIVE} when
installing a package. Details are beyond the scope of this specification.

\subsection{\t{REPLACING\_VERSIONS} and \t{REPLACED\_BY\_VERSION}}
\label{sec:replacing-versions}

\featurelabel{replace-version-vars} In EAPIs listed in table~\ref{tab:env-vars-table} as supporting
it, the \t{REPLACING\_VERSIONS} variable shall be defined in \t{pkg\_preinst} and \t{pkg\_postinst}.
In addition, it \e{may} be defined in \t{pkg\_pretend} and \t{pkg\_setup}, although ebuild authors
should take care to handle binary package creation and installation correctly when using it in these
phases.

\t{REPLACING\_VERSIONS} is a list, not a single optional value, to handle pathological cases such as
installing \t{foo-2:2} to replace \t{foo-2:1} and \t{foo-3:2}.

In EAPIs listed in table~\ref{tab:env-vars-table} as supporting it, the \t{REPLACED\_BY\_VERSION}
variable shall be defined in \t{pkg\_prerm} and \t{pkg\_postrm}. It shall contain at most one value.

\subsection{Offset-prefix variables \t{EPREFIX}, \t{EROOT} and \t{ED}}
\label{sec:offset-vars}

\begin{centertable}{EAPIs supporting offset-prefix}
    \label{tab:offset-support-table}
    \begin{tabular}{ l l }
        \toprule
        \multicolumn{1}{c}{\textbf{EAPI}} &
        \multicolumn{1}{c}{\textbf{Supports offset-prefix?}}\\
        \midrule
        \t{0} & No \\
        \t{1} & No \\
        \t{2} & No \\
        \t{3} & Yes \\
        \t{4} & Yes \\
        \bottomrule
    \end{tabular}
\end{centertable}

\featurelabel{offset-prefix-vars} Table~\ref{tab:offset-support-table} lists the EAPIs which support
offset-prefix installations. This support was initially added in EAPI 3, in the form of three extra
variables.  Two of these, \t{EROOT} and \t{ED}, are convenience variables using the variable
\t{EPREFIX}. In EAPIs that do not support an offset-prefix, the installation offset is hardwired to
\t{/usr}. In offset-prefix supporting EAPIs the installation offset is set as \t{\$\{EPREFIX\}/usr}
and hence can be adjusted using the variable \t{EPREFIX}. Note that the behaviour of offset-prefix
aware and agnostic is the same when \t{EPREFIX} is set to the empty string in offset-prefix aware
EAPIs.  The latter do have the variables \t{ED} and \t{EROOT} properly set, though.

% vim: set filetype=tex fileencoding=utf8 et tw=100 spell spelllang=en :

%%% Local Variables:
%%% mode: latex
%%% TeX-master: "pms"
%%% LaTeX-indent-level: 4
%%% LaTeX-item-indent: 0
%%% TeX-brace-indent-level: 4
%%% End:


\section{The state of variables between functions}
\label{ebuild-env-state}

Exported and default scope variables are saved between functions. A non-local variable set in a
function earlier in the call sequence must have its value preserved for later functions, including
functions executed as part of a later uninstall. Variables that were exported must remain exported
in later functions; variables with default visibility may retain default visibility or be exported.

Variables with special meanings to the package manager are excluded from this rule.

Global variables must only contain invariant values (see \ref{metadata-invariance}). If a global
variable's value is invariant, it may have the value that would be generated at any given point
in the build sequence.

This is demonstrated by code listing \ref{lst:env-saving}.

\begin{lstlisting}[float,caption=Environment state between functions,label=lst:env-saving]

GLOBAL_VARIABLE="a"

src_compile()
{
    GLOBAL_VARIABLE="b"
    DEFAULT_VARIABLE="c"
    export EXPORTED_VARIABLE="d"
    local LOCAL_VARIABLE="e"
}

src_install(){
    [[ ${GLOBAL_VARIABLE} == "a" ]] \
        || [[ ${GLOBAL_VARIABLE} == "b" ]] \
        || die "broken env saving for globals"

    [[ ${DEFAULT_VARIABLE} == "c" ]] \
        || die "broken env saving for default"

    [[ ${EXPORTED_VARIABLE} == "d" ]] \
        || die "broken env saving for exported"

    [[ $(printenv EXPORTED_VARIABLE ) == "d" ]] \
        || die "broken env saving for exported"

    [[ -z ${LOCAL_VARIABLE} ]] \
        || die "broken env saving for locals"
}
\end{lstlisting}

% vim: set filetype=tex fileencoding=utf8 et tw=100 spell spelllang=en :

%%% Local Variables:
%%% mode: latex
%%% TeX-master: "pms"
%%% End:


\section{Available commands}
\label{sec:ebuild-env-commands}

This section documents the commands available to an ebuild. Unless otherwise specified, they may be
aliases, shell functions, or executables in the ebuild's \t{PATH}.

When an ebuild is being sourced for metadata querying rather than for a build (that is to say,
when none of the \t{src\_} or \t{pkg\_} functions are to be called), no external command may
be executed. The package manager may take steps to enforce this.

\subsection{System commands}

Any ebuild not listed in the system set for the active profile(s) may assume the presence of every
command that is always provided by the system set for that profile. However, it must target the
lowest common denominator of all systems on which it might be installed---in most cases this means
that the only packages that can be assumed to be present are those listed in the \t{base} profile or
equivalent, which is inherited by all available profiles. If an ebuild requires any applications not
provided by the system profile, or that are provided conditionally based on USE flags, appropriate
dependencies must be used to ensure their presence.

\subsubsection{Guaranteed system commands}
\label{sec:guaranteed-system-commands}

The following commands must always be available in the ebuild environment:
\begin{compactitem}
\item All builtin commands in GNU bash, version 3.2\footnote{The
    required bash version was retroactively updated from 3.0 to 3.2 in
    November 2009 (see \url{http://www.gentoo.org/proj/en/council/meeting-logs/20091109.txt}).\label{fn:bash3.2}}.
\item \t{sed} must be available, and must support all forms of invocations valid for GNU sed
    version 4 or later.
\item \t{patch} must be available, and must support all inputs valid for GNU patch.
\end{compactitem}

\subsection{Commands provided by package dependencies}

In some cases a package's build process will require the availability of executables not provided by
the core system, a common example being autotools. Commands provided by dependencies are available
in the following cases:
\begin{compactitem}
\item In the \t{src} phases, any command provided by a package listed in \t{DEPEND} is available.
\item In the \t{pkg} phases, at least one of the following conditions must be met:
    \begin{compactitem}
    \item Any command provided by a package listed in \t{DEPEND} is available.
    \item Any command provided by a package listed in \t{RDEPEND} is available.
    \end{compactitem}
\end{compactitem}

\subsection{Ebuild-specific Commands}
\label{pkg-mgr-commands}

The following commands will always be available in the ebuild environment, provided by the package
manager. Except where otherwise noted, they may be internal (shell functions or aliases) or external
commands available in \t{PATH}; where this is not specified, ebuilds may not rely upon either
behaviour.

\subsubsection{Sandbox commands}
These commands affect the behaviour of the sandbox. Each command takes a single directory as
argument.
\begin{description}
\item[addread] Add a directory to the permitted read list.
\item[addwrite] Add a directory to the permitted write list.
\item[addpredict] Add a directory to the predict list. Any write to a location in this list will be
    denied, but will not trigger access violation messages or abort the build process.
\item[adddeny] Add a directory to the deny list.
\end{description}

\subsubsection{Package manager query commands}
These commands are used to extract information about the host system.
\begin{description}
\item[has\_version] Takes exactly one package dependency specification as an argument. Returns
    true if a package matching the atom is installed in \t{\$ROOT}, and false otherwise.
\item[best\_version] Takes exactly one package dependency specification as an argument. If a
    matching package is installed, prints the category, package name and version of the highest
    matching version.
\end{description}

\subsubsection{Output commands}
These commands display messages to the user. Unless otherwise stated, the entire argument list is
used as a message, as in the simple invocations of \t{echo}.
\begin{description}
\item[einfo] Displays an informational message.
\item[einfon] Displays an informational message without a trailing newline.
\item[elog] Displays an informational message of slightly higher importance. The package manager may
    choose to log \t{elog} messages by default where \t{einfo} messages are not, for example.
\item[ewarn] Displays a warning message.
\item[eerror] Displays an error message.
\item[ebegin] Displays an informational message. Should be used when beginning a possibly lengthy
    process, and followed by a call to \t{eend}.
\item[eend] Indicates that the process begun with an \t{ebegin} message has completed. Takes one
    fixed argument, which is a numeric return code, and an optional message in all subsequent
    arguments. If the first argument is 0, print a success indicator; otherwise, print the message
    followed by a failure indicator.
\end{description}

\subsubsection{Error commands}
These commands are used when an error is detected that will prevent the build process from
completing.
\begin{description}
\item[die] Displays a failure message provided in its first and only argument, and then aborts the
    build process. \t{die} is \e{not} guaranteed to work correctly if called from a subshell
    environment.
\item[assert] Checks the value of the shell's pipe status variable, and if any component is non-zero
    (indicating failure), calls \t{die} with its first argument as a failure message.
\end{description}

\subsubsection{Build commands}
These commands are used during the \t{src\_compile} and \t{src\_install} phases to run the
package's build commands.

\begin{description}
\item[econf] Calls the program's \t{./configure} script. This is designed to work with GNU
    Autoconf-generated scripts. Any additional parameters passed to \t{econf} are passed directly
    to \t{./configure}. \t{econf} will look in the current working directory for a configure script
    unless the \t{ECONF\_SOURCE} environment variable is set, in which case it is taken to be the
    directory containing it. \t{econf} must pass the following options to the configure script:
    \begin{itemize}
    \item -{}-prefix must default to \t{/usr} unless overridden by \t{econf}'s caller.
    \item -{}-mandir must be \t{/usr/share/man}
    \item -{}-infodir must be \t{/usr/share/info}
    \item -{}-datadir must be \t{/usr/share}
    \item -{}-sysconfdir must be \t{/etc}
    \item -{}-localstatedir must be \t{/var/lib}
    \item -{}-host must be the value of the \t{CHOST} environment variable.
    \item -{}-libdir must be set according to Algorithm \ref{alg:econf-libdir}.
    \end{itemize}

    \t{econf} must be implemented internally---that is, as a bash function and not an external
    script. Should any portion of it fail, it must abort the build using \t{die}.

\begin{algorithm}
\caption{econf --libdir logic} \label{alg:econf-libdir}
\begin{algorithmic}[1]
\STATE let prefix=/usr
\IF{the caller specified -{}-prefix=\$p}
    \STATE let prefix=\$p
\ENDIF
\STATE let libdir=
\IF{the ABI environment variable is set}
    \STATE let libvar=LIBDIR\_\$ABI
    \IF{the environment variable named by libvar is set}
        \STATE let libdir=the value of the variable named by libvar
    \ENDIF
\ENDIF
\IF{libdir is non-empty}
    \STATE pass -{}-libdir=\$prefix/\$libdir to configure
\ENDIF
\end{algorithmic}
\end{algorithm}

\item[emake] Calls GNU make. Any arguments given are passed directly to the make
    command, as are the user's chosen \t{MAKEOPTS}\@. Arguments given to \t{emake} override user
    configuration. See also section \ref{guaranteed-system-commands}. \t{emake} must be an external
    program and cannot be a function or alias---it must be callable from e.g. \t{xargs}.
\item[einstall] A shortcut for the command given in Listing \ref{lst:einstall}. Any arguments given
    to \t{einstall} are passed verbatim to \t{emake}, as shown.

\begin{lstlisting}[caption=einstall command,label=lst:einstall]
emake \
   prefix="${D}"/usr \
   mandir="${D}"/usr/share/man \
   infodir="${D}"/usr/share/info \
   libdir="${D}"/usr/$(get_libdir) \
   "$@" \
   install
\end{lstlisting}

\end{description}

\subsubsection{Installation commands}
These commands are used to install files into the staging area, in cases where the package's \t{make
install} target cannot be used or does not install all needed files. Except where otherwise stated,
all filenames created or modified are relative to the staging directory, given by \t{\$\{D\}}. These
commands must all be external programs and not bash functions or aliases---that is, they must be
callable from \t{xargs}.

\begin{description}
\item[dobin] Installs the given files into \t{DESTTREE/bin}, where \t{DESTTREE} defaults to
    \t{/usr}. Also makes the files executable.

\item[doconfd] Installs the given files into /etc/conf.d/.

\item[dodir] Creates the given directories, by default with file mode \t{0755}. This can be overridden
    by setting \t{DIROPTIONS} with the \t{diropts} function.

\item[dodoc] Installs the given files into a subdirectory under \t{/usr/share/doc/\$PF/}. The
    subdirectory is set by the most recent call to \t{docinto}. If \t{docinto} has not
    yet been called, instead installs to the directory \t{/usr/share/doc/\$PF/}.

\item[doexe] Installs the given files into the directory specified by the most recent
    \t{exedesttree} call. If \t{exedesttree} has not yet been called, behaviour is undefined.

\item[dohard] Takes two parameters. Creates a hardlink from the second to the first.

\item[dohtml] \TODO{Write dohtml.}

\item[doinfo] Installs a GNU Info file into the \t{/usr/share/info} area.

\item[doinitd] Installs an initscript into \t{/etc/init.d}.

\item[doins] Takes any number of files as arguments and installs them into \t{\$INSDESTTREE}\@. If
    the first argument is \t{-r}, then operates recursively, descending into any directories given.

\item[dolib] For each argument, installs it into the appropriate library directory as determined by
    Algorithm \ref{ebuild-libdir}. Any symlinks are installed into the same directory as relative
    links to their original target.

\item[dolib.so] As for dolib. Installs the file with mode \t{0755}.

\item[dolib.a] As for dolib. Installs the file with mode \t{0644}.

\begin{algorithm}
\caption{Determining the library directory} \label{ebuild-libdir}
\begin{algorithmic}[1]
\IF{CONF\_LIBDIR\_OVERRIDE is set in the environment}
    \STATE return CONF\_LIBDIR\_OVERRIDE
\ENDIF
\IF{CONF\_LIBDIR is set in the environment}
    \STATE let LIBDIR\_default=CONF\_LIBDIR
\ELSE
    \STATE let LIBDIR\_default=``lib''
\ENDIF
\IF{ABI is set in the environment}
    \STATE let abi=ABI
\ELSIF{DEFAULT\_ABI is set in the environment}
    \STATE let abi=DEFAULT\_ABI
\ELSE
    \STATE let abi=``default''
\ENDIF
\STATE return the value of LIBDIR\_\$abi
\end{algorithmic}
\end{algorithm}

\item[doman] Installs a man page into the appropriate subdirectory of \t{/usr/share/man} depending
    upon its apparent section suffix.

\item[domo] Installs a \t{.mo} file into the appropriate subdirectory of \t{DESTTREE/share/locale},
    generated by taking the basename of the file, removing the \t{.*} suffix, and appending
    \t{/LC\_MESSAGES}.

\item[dosbin] Installs a file into \t{DESTTREE/sbin}, making it executable if necessary.

\item[dosym] Creates a symbolic link named as for its second parameter, pointing to the first. If
    the directory containing the new link does not exist, creates it.

\item[fowners] Acts as for \t{chown}, but takes paths relative to the image directory.

\item[fperms] Acts as for \t{chmod}, but takes paths relative to the image directory.

\item[newbin] As for \t{dobin}, but takes two parameters. The first is the file to install; the
    second is the new filename under which it will be installed.

\item[newconfd] As for \t{doconfd}, but takes two parameters as for \t{newbin}.

\item[newdoc] As above, for \t{dodoc}.

\item[newenvd] As above, for \t{doenvd}.

\item[newexe] As above, for \t{doexe}.

\item[newinitd] As above, for \t{doinitd}.

\item[newins] As above, for \t{doins}.

\item[newman] As above, for \t{doman}.

\item[newsbin] As above, for \t{dosbin}.

\item[keepdir] Creates a directory as for \t{dodir}, and an empty file whose name starts with
    \t{.keep} in that directory to ensure that the directory does not get removed by the
    package manager should it be empty at any point.

\subsubsection{Commands affecting install destinations}
The following commands are used to set the various destination trees, all relative to \t{\$\{D\}},
used by the above installation commands. They must be shell functions or aliases, due to the need to
set variables read by the above commands.

\item[into] Sets the value of \t{DESTTREE} for future invocations of the above utilities.

\item[insinto] Sets the value of \t{INSDESTTREE} for future invocations of the above utilities.

\item[exeinto] Sets the install path for \t{doexe} and \t{newexe}.

\item[docinto] Sets the install subdirectory for \t{dodoc} et al.

\item[insopts] Sets the options passed by \t{doins} et al. to the \t{install} command.

\item[diropts] Sets the options passed by \t{dodir} et al. to the \t{install} command.

\item[exeopts] Sets the options passed by \t{doexe} et al. to the \t{install} command.

\item[libopts] Sets the options passed by \t{dolib} et al. to the \t{install} command.

\end{description}

\subsubsection{List Functions}
These functions work on variables containing whitespace-delimited lists (e.g. \t{USE}).

\begin{description}
\item[use] Returns shell true (0) if the first argument (a \t{USE} flag name) is enabled, false
    otherwise.  If the flag name is prefixed with \t{!}, returns true if the flag is disabled, and
    false if it is enabled.
\item[usev] The same as \t{use}, but also prints the flag name if it is enabled.
\item[useq] Deprecated synonym for \t{use}.
\\
\item[has] Returns shell true (0) if the first argument (a word) is found in the list of subsequent
    arguments, false otherwise.
\item[hasv] The same as \t{has}, but also prints the first argument if found.
\item[hasq] Deprecated synonym for \t{has}.
\\
\item[use\_with] Has one-, two-, and three-argument forms. The first argument is a USE flag name,
    the second a \t{configure} option name (\t{\$opt}), defaulting to the same as the first argument
    if not provided, and the third is a string value (\t{\$value}), defaulting to nothing. If the
    USE flag is set, outputs \t{--with-\$opt=\$value} if the third argument was provided, and
    \t{--with-\$opt} otherwise. If the flag is not set, then it outputs \t{--without-\$opt}.

\item[use\_enable] Works the same as \t{use\_with()}, but outputs \t{--enable-} or \t{--disable-}
instead of \t{--with-} or \t{--without-}.
\end{description}

\subsubsection{Misc Commands}
The following commands are always available in the ebuild environment, but don't really fit in any
of the above categories.

\begin{description}
\item[dosed] Takes any number of arguments, which can be files or \t{sed} expressions. For each
    argument, if it names, relative to \t{D} a file which exists, then \t{sed} is run with the
    current expression on that file. Otherwise, the current expression is set to the text of the
    argument. The initial value of the expression is \t{s:\$\{D\}::g}.
\item[unpack] Unpacks a source archive into the current directory. Must be able to unpack the
    following file formats, if the relevant binaries are available:
    \begin{itemize}
    \item tar files (\t{*.tar})
    \item gzip-compressed tar files (\t{*.tar.gz, *.tgz, *.tar.Z, *.tbz})
    \item bzip2-compressed tar files (\t{*.tar.bz2, *.tbz2, *.tar.bz})
    \item zip files (\t{*.zip, *.ZIP, *.jar}) (with \t{unzip} available)
    \item gzip files (\t{*.gz, *.Z, *.z})
    \item bzip2 files (\t{*.bz, *.bz2})
    \item 7zip files (\t{*.7z, *.7Z})
    \item rar files (\t{*.rar, *.RAR}) (with \t{unrar} available)
    \item LHA archives (\t{*.LHA, *.LHa, *.lha, *.lhz}) (with \t{lha} available)
    \item ar archives (\t{*.a, *.deb})
    \end{itemize}
    It is up to the ebuild to ensure that the relevant external utilities are available, whether by
    being in the system set or via dependencies.
\end{description}

\subsubsection{Debug Commands}
The following commands are available for debugging. Normally all of these commands should be no ops;
a package manager may provide a special debug mode where these commands instead do something.

\begin{description}
\item[debug-print] If in a special debug mode, the arguments should be outputted or recorded using
    some kind of debug logging.
\item[debug-print-function] Calls \t{debug-print} with \t{\$1: entering function} as the first
    argument and the remaining arguments as additional arguments.
\item[debug-print-section] Calls \t{debug-print} with \t{now in section \$*}.
\end{description}

% vim: set filetype=tex fileencoding=utf8 et tw=100 spell spelllang=en :


% vim: set filetype=tex fileencoding=utf8 et tw=100 spell spelllang=en :

%%% Local Variables:
%%% mode: latex
%%% TeX-master: "pms"
%%% LaTeX-indent-level: 4
%%% LaTeX-item-indent: 0
%%% TeX-brace-indent-level: 4
%%% End:


\section{The state of the system between functions}
\label{sec:ebuild-env-invariancy}

For the sake of this section:

\begin{compactitem}
\item Variancy is any package manager action that modifies either
    \t{ROOT} or \t{/} in any way that isn't merely a simple addition of
    something that doesn't alter other packages. This includes any
    non-default call to any \t{pkg} phase function except \t{pkg\_setup},
    a merge of any package or an unmerge of any package.
\item As an exception, changes to \t{DISTDIR} do not count as variancy.
\item The \t{pkg\_setup} function may be assumed not to introduce variancy.
    Thus, ebuilds must not perform variant actions in this phase.
\end{compactitem}

The following exclusivity and invariancy requirements are mandated:

\begin{compactitem}
\item No variancy shall be introduced at any point between a package's
    \t{pkg\_setup} being started up to the point that that package is
    merged, except for any variancy introduced by that package.
\item There must be no variancy between a package's \t{pkg\_setup} and
    a package's \t{pkg\_postinst}, except for any variancy introduced
    by that package.
\item Any non-default \t{pkg} phase function must be run exclusively.
\item Each phase function must be called at most once during the build
    process for any given package.
\end{compactitem}

% vim: set filetype=tex fileencoding=utf8 et tw=100 spell spelllang=en :


%%% Local Variables:
%%% mode: latex
%%% TeX-master: "pms"
%%% End:


% vim: set filetype=tex fileencoding=utf8 et tw=100 spell spelllang=en :

%%% Local Variables:
%%% mode: latex
%%% TeX-master: "pms"
%%% End:


\chapter{Merging and Unmerging}

\note{In this chapter, \e{file} and \e{regular file} have their Unix meanings.}

\section{Overview}

The merge process merges the contents of the \t{D} directory onto the filesystem under \t{ROOT}.
This is not a straight copy; there are various subtleties which must be addressed.

The unmerge process removes an installed package's files. It is not covered in detail in this
specification.

\section{Directories}

Directories are merged recursively onto the filesystem. The method used to perform the merge is not
specified, so long as the end result is correct. In particular, merging a directory may alter or
remove the source directory under \t{D}.

Ebuilds must not attempt to merge a directory on top of any existing file that is not either a
directory or a symlink to a directory.

\subsection{Permissions}

The owner, group and mode (including set*id and sticky bits) of the directory must be preserved,
except as follows:

\begin{compactitem}
\item Any directory owned by the user used to perform the build must become owned by the root user.
\item Any directory whose group is the primary group of the user used to perform the build must have
    its group be that of the root user.
\end{compactitem}

On SELinux systems, the SELinux context must also be preserved. Other directory attributes, including
modification time, may be discarded.

\subsection{Empty Directories}

Behaviour upon encountering an empty directory is undefined. Ebuilds must not attempt to install an
empty directory.

\section{Regular Files}

Regular files are merged onto the filesystem (but see the notes on configuration file protection,
below). The method used to perform the merge is not specified, so long as the end result is correct.
In particular, merging a regular file may alter or remove the source file under \t{D}.

Ebuilds must not attempt to merge a regular file on top of any existing file that is not either a
regular file or a symlink to a regular file.

\subsection{Permissions}

The owner, group and mode (including set*id and sticky bits) of the file must be preserved, except
as follows:

\begin{compactitem}
\item Any file owned by the user used to perform the build must become owned by the root user.
\item Any file whose group is the primary group of the user used to perform the build must have
    its group be that of the root user.
\item The package manager may reduce read and write permissions on executable files that have a
    set*id bit set.
\end{compactitem}

On SELinux systems, the SELinux context must also be preserved. Other file attributes, including
modification time, may be discarded.

\subsection{Configuration File Protection}
\label{sec:config-protect}

The package manager must provide a means to prevent user configuration files from being
overwritten by any package updates. The profile variables \t{CONFIG\_PROTECT} and
\t{CONFIG\_PROTECT\_MASK} (section~\ref{sec:profile-variables}) control the paths for which this
must be enforced.

In order to ensure interoperability with configuration update tools, the following scheme must be
used by all package managers when merging any regular file:

\begin{compactenum}
\item If the directory containing the file to be merged is not listed in \t{CONFIG\_PROTECT}, and
     is not a subdirectory of any such directory, and if the file is not listed in \t{CONFIG\_PROTECT},
     the file is merged normally.
\item If the directory containing the file to be merged is listed in \t{CONFIG\_PROTECT\_MASK}, or
    is a subdirectory of such a directory, or if the file is listed in \t{CONFIG\_PROTECT\_MASK},
    the file is merged normally.
\item If no existing file with the intended filename exists, or the existing file has identical
    content to the one being merged, the file is installed normally.
\item Otherwise, prepend the filename with \t{.\_cfg0000\_}. If no file with the new name exists,
    then the file is merged with this name.
\item Otherwise, increment the number portion (to form \t{.\_cfg0001\_<name>}) and repeat step 4.
    Continue this process until a usable filename is found.
\item If 9999 is reached in this way, behaviour is undefined.
\end{compactenum}

\section{Symlinks}

Symlinks are merged as symlinks onto the filesystem. The link destination for a merged link shall be
the same as the link destination for the link under \t{D}, except as noted below. The method used to
perform the merge is not specified, so long as the end result is correct; in particular, merging a
symlink may alter or remove the symlink under \t{D}.

Ebuilds must not attempt to merge a symlink on top of a directory.

\subsection{Rewriting}

Any absolute symlink whose link starts with \t{D} must be rewritten with the leading \t{D} removed.
The package manager should issue a notice when doing this.

\section{Hard links}

A hard link may be merged either as a single file with links or as multiple independent files.

\section{Other Files}

Ebuilds must not attempt to install any other type of file (FIFOs, device nodes etc).

% vim: set filetype=tex fileencoding=utf8 et tw=100 spell spelllang=en :

%%% Local Variables:
%%% mode: latex
%%% TeX-master: "pms"
%%% LaTeX-indent-level: 4
%%% LaTeX-item-indent: 0
%%% TeX-brace-indent-level: 4
%%% End:


\chapter{Metadata Cache}
\label{metadata-cache}

\section{Directory Contents}

The \t{profiles/metadata/cache} directory, if it exists, contains directories whose names are the
same as categories in the repository. Each subdirectory may optionally contain one file per package
version in that category, named \t{<package>-<version>}, in the format described below.

The metadata cache may be incomplete or non-existent, and may contain additional bogus entries.

\section{Cache File Format}

Each cache file contains the textual values of various metadata keys, one per line, in the following
order. Other lines may be present following these; their meanings are not defined here.

\begin{compactenum}
\item Build-time dependencies (\t{DEPEND})
\item Run-time dependencies (\t{RDEPEND})
\item Slot (\t{SLOT})
\item Source tarball URIs (\t{SRC\_URI})
\item \t{RESTRICT}
\item Package homepage (\t{HOMEPAGE})
\item Package license (\t{LICENSE})
\item Package description (\t{DESCRIPTION})
\item Package keywords (\t{KEYWORDS})
\item Inherited eclasses (\t{INHERITED})
\item Use flags that this package respects (\t{IUSE})
\item No longer used; this line is to be ignored.
\item Post dependencies (\t{PDEPEND})
\item Old-style virtuals provided by this package (\t{PROVIDE})
\item The ebuild API version to which this package conforms (\t{EAPI})
\item Properties (\t{PROPERTIES}). In some EAPIs, may optionally be blank, regardless of ebuild
    metadata; see table~\ref{tab:properties-table}.
\item Defined phases (\t{DEFINED\_PHASES}). In some EAPIs, may optionally be blank, regardless of
    ebuild metadata; see table~\ref{tab:defined-phases-table}.
\item Blank lines to pad the file to 22 lines long
\end{compactenum}

Future EAPIs may define new variables, remove existing variables, change the line number or
format used for a particular variable, add or reduce the total length of the file and so on.
Any future EAPI that uses this cache format will continue to place the EAPI value on
line 15 if such a concept makes sense for that EAPI, and will place a value that is clearly
not a supported EAPI on line 15 if it does not.

% vim: set filetype=tex fileencoding=utf8 et tw=100 spell spelllang=en :

%%% Local Variables:
%%% mode: latex
%%% TeX-master: "pms"
%%% End:


\chapter{Glossary}
\label{glossary}

This section contains explanations of some of the terms used in this document whose meaning may not
be immediately obvious.

\begin{description}
\item[qualified package name] A package name along with its associated category. For example,
    \t{app-editors/vim} is a qualified package name.
\item[old-style virtual] An old-style virtual is a psuedo-package which exists if it is listed in an
    ebuild's \t{PROVIDE} variable. See chapter \ref{old-virtuals}.
\item[new-style virtual] A new-style virtual is a normal package in the \t{virtual} category which
    installs no files and uses its dependency requirements to pull in a `provider'. This is more
    flexible than the old-style virtuals described above, and requires no special package manager
    code.
\item[stand-alone repository] An (ebuild) repository which is intended to function on its own as the
    only, or primary, repository on a system. Contrast with \i{slave repository} below.
\item[slave repository, non-stand-alone repository] An (ebuild) repository which is not complete
    enough to function on its own, but needs one or more \i{master repositories} to
    satisfy dependencies and provide repository-level support files. Known in Portage as an overlay.
\item[master repository] See above.

\end{description}


% vim: set filetype=tex fileencoding=utf8 et tw=100 spell spelllang=en :


\appendix

\chapter{metadata.xml}
\label{metadata-xml}

The \t{metadata.xml} file is used to contain extra package- or category-level information beyond
what is stored in ebuild metadata. Its exact format is strictly beyond the scope of this document,
and is described in the DTD file located at \url{http://www.gentoo.org/dtd/metadata.dtd}.

\chapter{Unspecified Items}

The following items are not specified by this document, and must not be relied upon by ebuilds.
This is, of course, an incomplete list---it covers only the things that the authors know have
been abused in the past.

\begin{compactitem}
\item The \t{FEATURES} variable. This is Portage specific.
\item Similarly, any \t{PORTAGE\_} variable not explicitly listed.
\item Any Portage configuration file.
\item The VDB (\t{/var/db/pkg}). Ebuilds must not access this or rely upon it existing or being
    in any particular format.
\item The \t{portageq} command. The \t{has\_version} and \t{best\_version} commands are
    available as functions.
\item The \t{emerge} command.
\item Binary packages.
\item The \t{PORTDIR\_OVERLAY} variable, and overlay behaviour in general.
\end{compactitem}

\chapter{Historical Curiosities}

The items described in this chapter are included for information only. They were deprecated or
abandoned long before \t{EAPI} was introduced. Ebuilds must not use these features, and package
managers should not be changed to support them.

\section{If-else use blocks}

Historically, Portage supported if-else use conditionals, as shown by
listing~\ref{lst:if-else-use-listing}. The block before the colon would be taken if the condition
was met, and the block after the colon would be taken if the condition was not met.

This feature was deprecated and removed from the tree long before the introduction of \t{EAPI}.

\lstinputlisting[float,caption=If-else use blocks,label=lst:if-else-use-listing]{if-else-use.listing}

\section{cvs Versions}

Portage has very crude support for CVS packages. The package \t{foo} could contain a file named
\t{foo-cvs.1.2.3.ebuild}. This version would order \i{higher} than any non-CVS version (including
\t{foo-2.ebuild}). This feature has not seen real world use and breaks versioned dependencies, so
it must not be used.

\IFKDEBUILDELSE
{
    The \t{scm} version rules specified in section~\ref{scm-versions} solve all of these issues.
}{
}

% vim: set filetype=tex fileencoding=utf8 et tw=100 spell spelllang=en :

%%% Local Variables:
%%% mode: latex
%%% TeX-master: "pms"
%%% End:


\chapter{Feature Availability by EAPI}

\note This chapter is informative and for convenience only. Refer to the main text for specifics.

\begin{longtable}{\IFANYKDEBUILDELSE{llllll}{lllll}}
\caption{Features in EAPIs}\\
\toprule
\multicolumn{1}{c}{\b{Feature}} &
\multicolumn{1}{c}{\b{Reference}} &
\IFANYKDEBUILDELSE{\multicolumn{4}{c}{\b{EAPIs}} \\}{\multicolumn{3}{c}{\b{EAPIs}} \\}
\multicolumn{1}{c}{} &
\multicolumn{1}{c}{} &
\multicolumn{1}{c}{0} &
\multicolumn{1}{c}{1} &
\IFANYKDEBUILDELSE{\multicolumn{1}{c}{\IFKDEBUILDCOLOUR{kdebuild-1}} &}{}
\multicolumn{1}{c}{2} \\
\midrule
\endfirsthead
\midrule
\multicolumn{1}{c}{\b{Feature}} &
\multicolumn{1}{c}{\b{Reference}} &
\IFANYKDEBUILDELSE{\multicolumn{4}{c}{\b{EAPIs}} \\}{\multicolumn{3}{c}{\b{EAPIs}} \\}
\multicolumn{1}{c}{} &
\multicolumn{1}{c}{} &
\multicolumn{1}{c}{0} &
\multicolumn{1}{c}{1} &
\IFANYKDEBUILDELSE{\multicolumn{1}{c}{\IFKDEBUILDCOLOUR{kdebuild-1}} &}{}
\multicolumn{1}{c}{2} \\
\midrule
\endhead
\midrule
\endfoot
\bottomrule
\endlastfoot

\IFANYKDEBUILDELSE{
    \IFKDEBUILDCOLOUR{\t{scm} support} &
    \IFKDEBUILDCOLOUR{table~\ref{scm-table}} &
    \IFKDEBUILDCOLOUR{Optional} &
    \IFKDEBUILDCOLOUR{Optional} &
    \IFKDEBUILDCOLOUR{Required} &
    \IFKDEBUILDCOLOUR{Optional} \\
}{}

\IFANYKDEBUILDELSE{
    \IFKDEBUILDCOLOUR{File extension} &
    \IFKDEBUILDCOLOUR{section~\ref{file-extension}} &
    \IFKDEBUILDCOLOUR{\t{.ebuild}} &
    \IFKDEBUILDCOLOUR{\t{.ebuild}} &
    \IFKDEBUILDCOLOUR{\t{.kdebuild-1}} &
    \IFKDEBUILDCOLOUR{\t{.ebuild}} \\
}{}

\t{IUSE} defaults & table~\ref{iuse-defaults-table} & No & Yes & \IFANYKDEBUILDELSE{\IFKDEBUILDCOLOUR{Yes} &}{} Yes \\

\IFANYKDEBUILDELSE{
    \IFKDEBUILDCOLOUR{\t{PROVIDE} support} &
    \IFKDEBUILDCOLOUR{table~\ref{provide-table}} &
    \IFKDEBUILDCOLOUR{Yes} &
    \IFKDEBUILDCOLOUR{Yes} &
    \IFKDEBUILDCOLOUR{No} &
    \IFKDEBUILDCOLOUR{Yes} \\
}{}

\t{PROPERTIES} & table~\ref{properties-table} & Optionally & Optionally & \IFANYKDEBUILDELSE{\IFKDEBUILDCOLOUR{Optionally} &}{} Optionally \\

\IFANYKDEBUILDELSE{
    \IFKDEBUILDCOLOUR{Pre-source \t{EAPI}} &
    \IFKDEBUILDCOLOUR{section~\ref{pre-source-eapi}} &
    \IFKDEBUILDCOLOUR{0 or unset} &
    \IFKDEBUILDCOLOUR{0 or unset} &
    \IFKDEBUILDCOLOUR{kdebuild-1} &
    \IFKDEBUILDCOLOUR{0 or unset} \\
}{}

\t{SRC\_URI} arrows & table~\ref{uri-arrows-table} & No & No & \IFANYKDEBUILDELSE{\IFKDEBUILDCOLOUR{Yes} &}{} Yes \\

\IFANYKDEBUILDELSE{
    \IFKDEBUILDCOLOUR{\t{SRC\_URI} labels} &
    \IFKDEBUILDCOLOUR{table~\ref{uri-labels-table}} &
    \IFKDEBUILDCOLOUR{No} &
    \IFKDEBUILDCOLOUR{No} &
    \IFKDEBUILDCOLOUR{Yes} &
    \IFKDEBUILDCOLOUR{No} \\
}{}

\IFANYKDEBUILDELSE{
    \IFKDEBUILDCOLOUR{\t{PDEPEND} labels} &
    \IFKDEBUILDCOLOUR{table~\ref{pdepend-labels-table}} &
    \IFKDEBUILDCOLOUR{No} &
    \IFKDEBUILDCOLOUR{No} &
    \IFKDEBUILDCOLOUR{Yes} &
    \IFKDEBUILDCOLOUR{No} \\
}{}

\IFANYKDEBUILDELSE{
    \IFKDEBUILDCOLOUR{Ranged Dependencies} &
    \IFKDEBUILDCOLOUR{table~\ref{range-deps-table}} &
    \IFKDEBUILDCOLOUR{No} &
    \IFKDEBUILDCOLOUR{No} &
    \IFKDEBUILDCOLOUR{Yes} &
    \IFKDEBUILDCOLOUR{No} \\
}{}

Slot dependencies &
    table~\ref{slot-deps-table} &
    No &
    \IFKDEBUILDELSE{Named}{Yes} &
    \IFANYKDEBUILDELSE{\IFKDEBUILDCOLOUR{Named and Operator} &}{}
    \IFKDEBUILDELSE{Named}{Yes} \\

Use dependencies & table~\ref{use-deps-table} & No & No &
    \IFANYKDEBUILDELSE{\IFKDEBUILDCOLOUR{kdebuild-style} &}{} 2-style \\

\t{!} blockers & table~\ref{bang-strength-table} & Unspecified & Unspecified &
    \IFANYKDEBUILDELSE{\IFKDEBUILDCOLOUR{Unspecified} &}{} Weak \\

\t{!!} blockers & table~\ref{bang-strength-table} & Forbidden & Forbidden &
    \IFANYKDEBUILDELSE{\IFKDEBUILDCOLOUR{Forbidden} &}{} Strong \\

\t{src\_prepare} & table~\ref{src-prepare-table} & No & No & \IFANYKDEBUILDELSE{\IFKDEBUILDCOLOUR{No} &}{} Yes \\

\t{src\_configure} & table~\ref{src-configure-table} & No & No & \IFANYKDEBUILDELSE{\IFKDEBUILDCOLOUR{No} &}{} Yes \\

\t{src\_compile} style & table~\ref{src-compile-table} & 0 & 1 &
    \IFANYKDEBUILDELSE{\IFKDEBUILDCOLOUR{1} &}{} 2 \\

\IFANYKDEBUILDELSE{
    \IFKDEBUILDCOLOUR{\t{src\_test}} &
    \IFKDEBUILDCOLOUR{table~\ref{test-required-table}} &
    \IFKDEBUILDCOLOUR{User option} &
    \IFKDEBUILDCOLOUR{User option} &
    \IFKDEBUILDCOLOUR{Required} &
    \IFKDEBUILDCOLOUR{User option} \\
}{}

\IFANYKDEBUILDELSE{
    \IFKDEBUILDCOLOUR{\t{pkg\_info}} &
    \IFKDEBUILDCOLOUR{table~\ref{pkg-info-table}} &
    \IFKDEBUILDCOLOUR{Installed} &
    \IFKDEBUILDCOLOUR{Installed} &
    \IFKDEBUILDCOLOUR{Both} &
    \IFKDEBUILDCOLOUR{Installed} \\
}{}

\t{default\_} phase functions & table~\ref{default-phase-function-table} & None & None &
    \IFANYKDEBUILDELSE{\IFKDEBUILDCOLOUR{None} &}{} \parbox[t]{1in}{\t{pkg\_nofetch}, \t{src\_unpack},
    \t{src\_prepare}, \t{src\_configure}, \t{src\_compile}, \t{src\_test}} \\

\IFANYKDEBUILDELSE{
    \IFKDEBUILDCOLOUR{\t{dohard}} &
    \IFKDEBUILDCOLOUR{table~\ref{banned-commands-table}} &
    \IFKDEBUILDCOLOUR{Yes} &
    \IFKDEBUILDCOLOUR{Yes} &
    \IFKDEBUILDCOLOUR{Banned} &
    \IFKDEBUILDCOLOUR{Yes} \\
}{}

\IFANYKDEBUILDELSE{
    \IFKDEBUILDCOLOUR{\t{dohtml}} &
    \IFKDEBUILDCOLOUR{table~\ref{banned-commands-table}} &
    \IFKDEBUILDCOLOUR{Yes} &
    \IFKDEBUILDCOLOUR{Yes} &
    \IFKDEBUILDCOLOUR{Banned} &
    \IFKDEBUILDCOLOUR{Yes} \\
}{}

\IFANYKDEBUILDELSE{
    \IFKDEBUILDCOLOUR{\t{dosed}} &
    \IFKDEBUILDCOLOUR{table~\ref{banned-commands-table}} &
    \IFKDEBUILDCOLOUR{Yes} &
    \IFKDEBUILDCOLOUR{Yes} &
    \IFKDEBUILDCOLOUR{Banned} &
    \IFKDEBUILDCOLOUR{Yes} \\
}{}

\t{doman} languages & table~\ref{doman-table} & No & No &
    \IFANYKDEBUILDELSE{\IFKDEBUILDCOLOUR{No} &}{} Yes \\

\IFANYKDEBUILDELSE{
    \IFKDEBUILDCOLOUR{\t{dosym} does \t{dodir}} &
    \IFKDEBUILDCOLOUR{table~\ref{dosym-table}} &
    \IFKDEBUILDCOLOUR{Yes} &
    \IFKDEBUILDCOLOUR{Yes} &
    \IFKDEBUILDCOLOUR{No} &
    \IFKDEBUILDCOLOUR{Yes} \\
}{}

\t{default} function & table~\ref{default-function-table} & No & No &
    \IFANYKDEBUILDELSE{\IFKDEBUILDCOLOUR{No} &}{} Yes \\

\end{longtable}

\chapter{Differences Between EAPIs}

\note This chapter is informative and for convenience only. Refer to the main text for specifics.

\section*{EAPI 0}

EAPI 0 is the base EAPI.

\section*{EAPI 1}

EAPI 1 is EAPI 0 with the following changes:

\begin{compactitem}
\item \t{IUSE} defaults, table~\ref{iuse-defaults-table}.
\item Slot dependencies, table~\ref{slot-deps-table}.
\item Different \t{src\_compile} implementation, table~\ref{src-compile-table}.
\end{compactitem}

\IFKDEBUILDELSE
{
    \section*{EAPI kdebuild-1}

    EAPI kdebuild-1 is EAPI 1 with the following changes:

    \begin{compactitem}
    \item \t{scm} support, table~\ref{scm-table}.
    \item \t{kdebuild-1} file extension, section~\ref{file-extension}.
    \item \t{PROVIDE} banned, table~\ref{provide-table}.
    \item Pre-source EAPI is \t{kdebuild-1}, section~\ref{pre-source-eapi}.
    \item \t{SRC\_URI} arrows, table~\ref{uri-arrows-table}.
    \item \t{SRC\_URI} labels, table~\ref{uri-labels-table}.
    \item \t{PDEPEND} labels, table~\ref{pdepend-labels-table}.
    \item Ranged dependencies, table~\ref{range-deps-table}.
    \item Use dependencies, table~\ref{use-deps-table}.
    \item \t{src\_test} mandatory, table~\ref{test-required-table}.
    \item \t{pkg\_info} can run on uninstalled packages, table~\ref{pkg-info-table}.
    \item \t{dohard}, \t{dohtml}, \t{dosed} banned, table~\ref{banned-commands-table}.
    \item \t{dosym} will not do \t{dodir}, table~\ref{dosym-table}.
    \end{compactitem}
}{
}

\section*{EAPI 2}

EAPI 2 is EAPI 1 with the following changes:

\begin{compactitem}
\item \t{SRC\_URI} arrows, table~\ref{uri-arrows-table}.
\item Use dependencies, table~\ref{use-deps-table}.
\item \t{!} and \t{!!} blockers, table~\ref{bang-strength-table}.
\item \t{src\_prepare}, table~\ref{src-prepare-table}.
\item \t{src\_configure}, table~\ref{src-configure-table}.
\item Different \t{src\_compile} implementation, table~\ref{src-compile-table}.
\item \t{default\_} phase functions for phases \t{pkg\_nofetch}, \t{src\_unpack}, \t{src\_prepare},
    \t{src\_configure}, \t{src\_compile} and \t{src\_test}; table~\ref{default-phase-function-table}.
\item \t{doman} languages support, table~\ref{doman-table}.
\item \t{default} function, table~\ref{default-function-table}.
\end{compactitem}

% vim: set filetype=tex fileencoding=utf8 et tw=100 spell spelllang=en :

%%% Local Variables:
%%% mode: latex
%%% TeX-master: "pms"
%%% End:


\bibliography{pms}

\end{document}

% vim: set filetype=tex fileencoding=utf8 et tw=100 spell spelllang=en :


%%% Local Variables:
%%% mode: latex
%%% TeX-master: t
%%% End:
