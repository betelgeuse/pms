The following commands will always be available in the ebuild environment, provided by the package
manager:

\subsubsection{Sandbox commands}
These commands affect the behaviour of the sandbox. Each command takes a single directory as
argument.
\begin{description}
\item[addread] Add a directory to the permitted read list.
\item[addwrite] Add a directory to the permitted write list.
\item[addpredict] Add a directory to the predict list. Any write to a location in this list will be
    denied, but will not trigger access violation messages or abort the build process.
\item[adddeny] Add a directory to the deny list.
\end{description}

\subsubsection{Package manager query commands}
These commands are used to extract information about the host system.
\begin{description}
\item[has\_version] Takes exactly one dependency atom as an argument. Returns true if a package
    matching the atom is installed in \t{\$ROOT}, and false otherwise.
\item[best\_version] Takes exactly one dependency atom as an argument. If a matching package is
    installed, prints the category, package name and version of the highest matching version.
\end{description}

\subsubsection{Output commands}
These commands display messages to the user. Unless otherwise stated, the entire argument list is
used as a message, as in the simple invocations of \t{echo}.
\begin{description}
\item[einfo] Displays an informational message.
\item[einfon] Displays an informational message without a trailing newline.
\item[elog] Displays an informational message of slightly higher importance. The package manager may
    choose to log \t{elog} messages by default where \t{einfo} messages are not, for example.
\item[ewarn] Displays a warning message.
\item[eerror] Displays an error message.
\item[ebegin] Displays an informational message. Should be used when beginning a possibly lengthy
    process, and followed by a call to \t{eend}.
\item[eend] Indicates that the process begun with an \t{ebegin} message has completed. Takes one
    fixed argument, which is a numeric return code, and an optional message in all subsequent
    arguments. If the first argument is 0, print a success indicator; otherwise, print the message
    followed by a failure indicator.
\end{description}

\subsubsection{Error commands}
These commands are used when an error is detected that will prevent the build process from
completing.
\begin{description}
\item[die] Displays a failure message provided in its first and only argument, and then aborts the
    build process. \t{die} is \e{not} guaranteed to work correctly if called from a subshell
    environment.
\item[assert] Checks the value of the shell's pipe status variable, and if any component is non-zero
    (indicating failure), calls \t{die} with its first argument as a failure message.
\end{description}

\subsubsection{Build commands}
These commands are used during the \t{src\_compile} and \t{src\_install} phases to run the
package's build commands.

\begin{description}
\item[econf] Calls the program's \t{./configure} script. This is designed to work with GNU
    Autoconf-generated scripts. Any additional parameters passed to \t{econf} are passed directly
    to \t{./configure}. \t{econf} will look in the current working directory for a configure script
    unless the \t{ECONF\_SOURCE} environment variable is set, in which case it is taken to be the
    directory containing it. \t{econf} must pass the following options to the configure script:
    \begin{itemize}
    \item --prefix should default to \t{/usr} unless overridden by \t{econf}'s caller.
    \item --mandir should be \t{/usr/share/man}
    \item --infodir should be \t{/usr/share/info}
    \item --datadir should be \t{/usr/share}
    \item --sysconfdir should be \t{/etc}
    \item --localstatedir should be \t{/var/lib}
    \item --host should be the value of the \t{CHOST} environment variable.
    \item --libdir should be set according to Algorithm \ref{alg:econf-libdir}.
    \end{itemize}

\begin{algorithm}
\caption{econf --libdir logic} \label{alg:econf-libdir}
\begin{algorithmic}[1]
\STATE let prefix=/usr
\IF{the caller specified --prefix=\$p}
    \STATE let prefix=\$p
\ENDIF
\STATE let libdir=
\IF{the ABI environment variable is set}
    \STATE let libvar=LIBDIR\_\$ABI
    \IF{the environment variable named by libvar is set}
        \STATE let libdir=the value of the variable named by libvar
    \ENDIF
\ENDIF
\IF{libdir is non-empty}
    \STATE pass --libdir=\$prefix/\$libdir to configure
\ENDIF
\end{algorithmic}
\end{algorithm}

\item[emake] Calls the system make command. Any arguments given are passed directly to the make
    command, as are the user's chosen \t{MAKEOPTS}. Arguments given to \t{emake} override user
    configuration. See also section \ref{guaranteed-system-commands}.
\item[einstall] A shortcut for the command given in Listing \ref{lst:einstall}. Its use is
    discouraged in favour of using \t{make DESTDIR=\$\{D\} install}.

\begin{lstlisting}[caption=einstall command,label=lst:einstall]
emake \
   prefix="${D}"/usr \
   mandir="${D}"/usr/share/man \
   infodir="${D}"/usr/share/info \
   libdir="${D}"/usr/$(get_libdir) \
   install
\end{lstlisting}

\end{description}

\subsubsection{Installation commands}
These commands are used to install files into the staging area, in cases where the package's \t{make
install} target cannot be used or does not install all needed files. Except where otherwise stated,
all filenames created or modified are relative to the staging directory, given by \$\{D\}.

\begin{description}
\item[dobin] Installs the given files into \t{DESTTREE/bin}, where \t{DESTTREE} defaults to
    \t{/usr}. Also makes the files executable.

\item[doconfd] Installs the given files into /etc/conf.d/.

\item[dodir] Creates the given directories, by default with file mode 0755. This can be overridden
    by setting \t{DIROPTIONS} with the \t{diropts} function.

\item[dodoc] Installs the given files into \t{/usr/share/doc/\$PF/\$DOCDESTTREE/}, where
    \t{DOCDESTTREE} defaults to the empty string. These files may optionally be compressed by the
    package manager.

\item[doexe] Installs the given files into \t{EXEDESTTREE}, where \t{EXEDESTTREE} defaults to
    the empty string.

\item[dohard] Takes two parameters. Creates a hardlink from the second to the first.

\item[dohtml] \TODO{Write dohtml.}

\item[doinfo] Installs a GNU Info file into the \t{/usr/share/info} area, optionally compressed.

\item[doinitd] Installs an initscript into \t{/etc/init.d}.

\item[doins] Takes any number of files as arguments and installs them into \t{\$INSDESTTREE}. If
    the first argument is \t{-r}, then operates recursively, descending into any directories given.

\item[dolib] For each argument, installs it into the appropriate library directory as determined by
    Algorithm \ref{ebuild-libdir}. Any symlinks are installed into the same directory as relative
    links to their original target.

\begin{algorithm}
\caption{Determining the library directory} \label{ebuild-libdir}
\begin{algorithmic}[1]
\IF{CONF\_LIBDIR\_OVERRIDE is set in the environment}
    \STATE return CONF\_LIBDIR\_OVERRIDE
\ENDIF
\IF{CONF\_LIBDIR is set in the environment}
    \STATE let LIBDIR\_default=CONF\_LIBDIR
\ELSE
    \STATE let LIBDIR\_default=``lib''
\ENDIF
\IF{ABI is set in the environment}
    \STATE let abi=ABI
\ELSIF{DEFAULT\_ABI is set in the environment}
    \STATE let abi=DEFAULT\_ABI
\ELSE
    \STATE let abi=``default''
\ENDIF
\STATE return the value of LIBDIR\_\$abi
\end{algorithmic}
\end{algorithm}

\item[doman] Installs a man page into the appropriate subdirectory of \t{/usr/share/man} depending
    upon its apparent section suffix. Optionally compresses it.

\item[domo] Installs a \t{.mo} file into the appropriate subdirectory of \t{DESTTREE/share/locale},
    generated by taking the basename of the file, removing the \t{.*} suffix, and appending
    \t{/LC\_MESSAGES}.

\item[donewins] Deprecated synonym for \t{newins}.

\item[dosbin] Installs a file into \t{DESTTREE/sbin}, making it executable if necessary.

\item[dosym] Creates a symbolic link named as for its second parameter, pointing to the first. If
    the directory containing the new link does not exist, creates it.

\item[fowners] Acts as for \t{chown}, but takes paths relative to the image directory.

\item[fperms] Acts as for \t{chmod}, but takes paths relative to the image directory.

\item[newbin] As for \t{dobin}, but takes two parameters. The first is the file to install; the
    second is the new filename under which it will be installed.

\item[newconfd] As for \t{doconfd}, but takes two parameters as for \t{newbin}.

\item[newdoc] As above, for \t{dodoc}.

\item[newenvd] As above, for \t{doenvd}.

\item[newexe] As above, for \t{doexe}.

\item[newinitd] As above, for \t{doinitd}.

\item[newins] As above, for \t{doins}.

\item[newman] As above, for \t{doman}.

\item[newsbin] As above, for \t{dosbin}.

\item[keepdir] Creates a directory as for \t{dodir}, and an empty \t{.keep} file in it to ensure
    that the directory does not get removed by the package manager should it be empty at any point.

\item[into] Sets the value of \t{DESTTREE} for future invocations of the above utilities.

\item[insinto] Sets the value of \t{INSDESTTREE} for future invocations of the above utilities.

\item[exeinto] Sets the value of \t{EXEDESTTREE} for future invocations of \t{doexe} and \t{newexe}.

\item[docinto] Sets the value of \t{DOCDESTTREE} for future invocations of \t{dodoc} et al.

\item[insopts] Sets the options passed by \t{doins} et al. to the \t{install} command.

\item[diropts] Sets the options passed by \t{dodir} et al. to the \t{install} command.

\item[exeopts] Sets the options passed by \t{doexe} et al. to the \t{install} command.

\item[libopts] Sets the options passed by \t{dolib} et al. to the \t{install} command.

\end{description}

\subsubsection{List Functions}
These functions work on variables containing space-separated lists (e.g. \t{USE}).

\begin{description}
\item[use] Returns true (0) if the first argument (a \t{USE} flag name) is enabled, false otherwise.
Also, the use flag can be negated (by prefixing it with a \t{!}).
\item[usev] The same as \t{use()}, only it also outputs the use flag if it is enabled.
\item[useq] Behaves the same as \t{use()}.
\\
\item[has] Returns true (0) if the first argument (a word) is found in the second argument (a
variable, e.g. \t{USE} or \t{FEATURES}). Returns false (1) otherwise.
\item[hasv] The same as \t{has()}, only it also outputs the found variable 
\item[hasq] Behaves the same as \t{has()}.
\\
\item[use\_with] Outputs a suitable \t{--with-} or \t{--without-} argument to be passed to
\t{econf()}, based upon whether a given \t{USE} flag is enabled or disabled. Takes 2 arguments: The
first is the name of the \t{USE} flag to check. The second is the name of the \t{--with\{,out\}-}
argument. If not given, defaults to the \t{USE} flag name. As with \t{use()}, flag names can be
inverted by prefixing them with a \t{!}.
\item[use\_enable] Works the same as \t{use\_with()}, only it outputs \t{--enable-} or \t{--disable-}
instead.
\end{description}

\subsubsection{Misc Commands}
The following commands are always available in the ebuild environment, but don't really fit in any
of the above categories.

\begin{description}
\item[dosed] Uses \t{sed} to remove any references to the image directory in each file named by its
    arguments.
\item[unpack] Unpacks a source archive into the current directory. Must be able to unpack the
    following file formats, if the relevant binaries are available:
    \begin{itemize}
    \item tar files (\t{*.tar})
    \item gzip-compressed tar files (\t{*.tar.gz, *.tgz, *.tar.Z})
    \item bzip2-compressed tar files (\t{*.tar.bz2, *.tbz2})
    \item zip files (\t{*.zip, *.ZIP, *.jar}) (with \t{unzip} available)
    \item gzip files (\t{*.gz, *.Z, *.z})
    \item bzip2 files (\t{*.bz2})
    \item rar files (\t{*.rar, *.RAR}) (with \t{unrar} available)
    \item LHA archives (\t{*.LHA, *.LHa, *.lha, *.lhz}) (with \t{lha} available)
    \item ar archives (\t{*.a, *.deb})
    \end{itemize}
    It is up to the ebuild to ensure that the relevant external utilities are available, whether by
    being in the system set or via dependencies.
\end{description}

% vim: set filetype=tex fileencoding=utf8 et tw=100 spell spelllang=en :
