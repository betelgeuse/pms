\chapter{Configuration File Protection}
\label{config-protect}

The package manager should provide a means to prevent user configuration files from being
overwritten by any package updates. The profile variables \t{CONFIG\_PROTECT} and
\t{CONFIG\_PROTECT\_MASK} (section \ref{profile-variables}) control the directories for which this
should be enforced.

In order to ensure interoperability with configuration update tools, the following scheme should be
used by all package managers when merging any (regular) file:

\begin{enumerate}
\item If the directory containing the file to be merged is not listed in \t{CONFIG\_PROTECT}, and is
    not a subdirectory of any such directory, the file is merged normally.
\item If the directory containing the file to be merged is listed in \t{CONFIG\_PROTECT\_MASK}, or
    is a subdirectory of such a directory, the file is merged normally.
\item If no existing file with the intended filename exists, or the existing file has identical
    content to the one being merged, the file is installed normally.
\item Otherwise, prepend the filename with \t{.\_cfg0000\_}. If no file with the new name exists,
    then the file is merged with this name.
\item Otherwise, increment the number portion (to form \t{.\_cfg0001\_<name>}) and repeat step 4.
    Continue this process until a usable filename is found.
\item If 9999 is reached in this way, behaviour is undefined.
\end{enumerate}

% vim: set filetype=tex fileencoding=utf8 et tw=100 spell spelllang=en :
