\chapter{Ebuild-defined Functions}
\label{ebuild-functions}

\TODO{Call order}

\section{List of Functions}
\label{functions}

The following is a list of functions that an ebuild, or eclass, may define, and which will be called
by the package manager as part of the build and/or install process. In all cases the package manager
must provide a default implementation of these functions; unless otherwise stated this should be a
no-op. Most functions should assume only that they have write access to the package's working
directory (the \t{WORKDIR} environment variable; see section \ref{env-var-WORKDIR}), and the
temporary directory \t{T}; exceptions are noted below. All functions may assume that they have read
access to all system libraries, binaries and configuration files that are accessible to normal
users.

Some functions may assume that their initial working directory is set to a particular location;
these are noted below. If no initial working directory is mandated, it may be set to anything and
the ebuild must not rely upon a particular location for it.

The environment for functions run outside of the build sequence (that is, \t{pkg\_config},
\t{pkg\_prerm} and \t{pkg\_postrm}) must be the environment used for the build of the package,
not the current configuration.

\subsection{pkg\_setup}
\label{pkg-setup-function}
The \t{pkg\_setup} function sets up the ebuild's environment for all following functions, before
the build process starts. Further, it checks whether any necessary prerequisites not covered
by the package manager, e.g. that certain kernel configuration options are enabled are fulfilled.

\t{pkg\_setup} must be run with full filesystem permissions, including the ability to add new users
and/or groups to the system.

\subsection{src\_unpack}
\label{src-unpack-function}

The \t{src\_unpack} function extracts all of the package's sources, applies patches and sets up the
package's build system for further use.

The initial working directory must be \t{WORKDIR}, and the default implementation used when
the ebuild lacks the \t{src\_unpack} function is:

\begin{lstlisting}
src_unpack() {
    if [ "${A}" != "" ]; then
        unpack ${A}
    fi
}
\end{lstlisting}

\subsection{src\_compile}
\label{src-compile-function}

The \t{src\_compile} function configures the package's build environment and builds the package.

The initial working directory must be \t{S} if that exists, falling back to \t{WORKDIR} otherwise.
The default implementation used when the ebuild lacks the \t{src\_compile} function is:

\begin{lstlisting}
src_compile() {
    if [ -x ./configure ]; then
        econf
    fi
    if [ -f Makefile ] || [ -f GNUmakefile ] || [ -f makefile ]; then
        emake || die "emake failed"
    fi
}
\end{lstlisting}

\subsection{src\_test}
\label{src-test-function}

The \t{src\_test} function runs unit tests for the newly build but yet uninstalled package as
provided.

The initial working directory must be \t{S} if that exists, falling back to \t{WORKDIR} otherwise.
The default implementation used when the ebuild lacks the \t{src\_test} function must, if tests are
enabled, run \t{make check} or \t{make test}, whichever exists, and abort if this fails.

\subsection{src\_install}
\label{src-install-function}

The \t{src\_install} function installs the package's content to a directory specified in
\t{\${D}}.

The initial working directory must be \t{S} if that exists, falling back to \t{WORKDIR} otherwise.
The default implementation used when the ebuild lacks the \t{src\_install} function is a no-op.

\subsection{pkg\_preinst}
\label{pkg-preinst-function}

The \t{pkg\_preinst} function performs any special tasks that are required immediately before
merging the package to the live filesystem. It must not write outside of the directories specified
by the \t{ROOT} and \t{D} environment variables.

\t{pkg\_preinst} must be run with full access to all files and directories below that specified by
the \t{ROOT} and \t{D} environment variables.

\subsection{pkg\_postinst}
\label{pkg-postinst-function}

The \t{pkg\_postinst} function performs any special tasks that are required immediately after
merging the package to the live filesystem. It should not write outside of the directory specified
in the \t{ROOT} environment variable.

\t{pkg\_postinst}, like, \t{pkg\_preinst}, must be run with full access to all files and directories
below that specified by the \t{ROOT} environment variable.

\subsection{pkg\_config}
\label{pkg-config-function}

The \t{pkg\_config} performs any custom steps required to configure a package after it has been
fully installed. It is the only ebuild function which may be interactive and prompt for user input.

\t{pkg\_config} must be run with full access to all files and directories inside of \t{ROOT}.

\subsection{pkg\_nofetch}
\label{pkg-nofetch-function}

The \t{pkg\_nofetch} function is run when the fetch phase of an fetch-restricted ebuild is run, and
the relevant source files are not available. It should direct the user to download all relevant
source files from their respective locations, with notes concerning licensing if applicable.

\t{pkg\_nofetch} should require no write access to any part of the filesystem.

\TODO{Check that pkg\_nofetch doesn't write anywhere.}

% vim: set filetype=tex fileencoding=utf8 et tw=100 spell spelllang=en :
