\chapter{Metadata Cache}
\label{metadata-cache}

\section{Directory Contents}

The \t{profiles/metadata/cache} directory, if it exists, contains directories whose names are the
same as categories in the repository. Each subdirectory may optionally contain one file per package
version in that category, named \t{<package>-<version>}, in the format described below.

The metadata cache may be incomplete or non-existent, and may contain additional bogus entries.

\section{Cache File Format}

Each cache file contains the textual values of various metadata keys, one per line, in the following
order. Other lines may be present following these; their meanings are not defined here.

\begin{compactenum}
\item Build-time dependencies (\t{DEPEND})
\item Run-time dependencies (\t{RDEPEND})
\item Slot (\t{SLOT})
\item Source tarball URIs (\t{SRC\_URI})
\item \t{RESTRICT}
\item Package homepage (\t{HOMEPAGE})
\item Package license (\t{LICENSE})
\item Package description (\t{DESCRIPTION})
\item Package keywords (\t{KEYWORDS})
\item Inherited eclasses (\t{INHERITED})
\item Use flags that this package respects (\t{IUSE})
\item No longer used; this line is to be ignored.
\item Post dependencies (\t{PDEPEND})
\item Old-style virtuals provided by this package (\t{PROVIDE})
\item The ebuild API version to which this package conforms (\t{EAPI})
\item Properties (\t{PROPERTIES}). In some EAPIs, may optionally be blank, regardless of ebuild
    metadata; see table~\ref{tab:properties-table}.
\item Defined phases (\t{DEFINED\_PHASES}). In some EAPIs, may optionally be blank, regardless of
    ebuild metadata; see table~\ref{tab:defined-phases-table}.
\item Blank lines to pad the file to 22 lines long
\end{compactenum}

Future EAPIs may define new variables, remove existing variables, change the line number or
format used for a particular variable, add or reduce the total length of the file and so on.
Any future EAPI that uses this cache format will continue to place the EAPI value on
line 15 if such a concept makes sense for that EAPI, and will place a value that is clearly
not a supported EAPI on line 15 if it does not.

% vim: set filetype=tex fileencoding=utf8 et tw=100 spell spelllang=en :

%%% Local Variables:
%%% mode: latex
%%% TeX-master: "pms"
%%% End:
