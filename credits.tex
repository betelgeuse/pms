\section*{Acknowledgements}

Thanks to Mike Kelly (package manager provided utilities, section~\ref{sec:pkg-mgr-commands}), Danny van
Dyk (ebuild functions, section~\ref{sec:ebuild-functions}), David Leverton (various sections) and
Petteri Räty (environment state, section~\ref{sec:ebuild-env-state}) for contributions. Thanks to
Christian Faulhammer for fixing some of the more horrible formatting screwups. Thanks also to Mike
Frysinger and Brian Harring for proof-reading and suggestions for fixes and/or clarification.

\section*{Copyright and Licence}

The bulk of this document is \textcopyright{} 2007, 2008 Stephen Bennett and Ciaran McCreesh.
Contributions are owned by their respective authors, and may have been changed substantially before
inclusion.

This document is released under the Creative Commons Attribution-Share Alike 3.0 Licence. The full
text of this licence can be found at \url{http://creativecommons.org/licenses/by-sa/3.0/}.

\section*{Reporting Issues}

Issues (inaccuracies, wording problems, omissions etc.)\ in this document should be reported via
Gentoo Bugzilla using product \e{Gentoo Hosted Projects}, component \e{PMS/EAPI} and the default
assignee. There should be one bug per issue, and one issue per bug.

Patches (in \t{git format-patch} form if possible) may be submitted either via Bugzilla or to the
\t{gentoo-pms@gentoo.org} mailing list. Patches will be reviewed by the PMS team, who will do one of
the following:

\begin{compactitem}
\item Accept and apply the patch.
\item Explain why the patch cannot be applied as-is. The patch may then be updated and resubmitted
if \mbox{appropriate}.
\item Reject the patch outright.
\item Take special action merited by the individual circumstances.
\end{compactitem}

When reporting issues, remember that this document is not the appropriate place for pushing
through changes to the tree or the package manager, except where those changes are bugs.

If any issue cannot be resolved by the PMS team, it may be escalated to the Gentoo Council.

% vim: set filetype=tex fileencoding=utf8 et tw=100 spell spelllang=en :

%%% Local Variables:
%%% mode: latex
%%% TeX-master: "pms"
%%% End:
