\chapter{Eclasses}
\label{sec:eclasses}

Eclasses serve to store common code that is used by more than one ebuild, which greatly aids
maintainability and reduces the tree size. However, due to metadata cache issues, care must be taken
in their use. In format they are similar to an ebuild, and indeed are sourced as part of any ebuild
using them. The interpreter is therefore the same, and the same requirements for being parseable
hold.

Eclasses must be located in the \t{eclass} directory in the top level of the repository---see
section~\ref{sec:eclass-dir}. Each eclass is a single file named \t{<name>.eclass}, where \t{<name>} is
the name of this eclass, used by \t{inherit} and \t{EXPORT\_FUNCTIONS} among other places.

\section{The inherit command}
\label{sec:inherit}

An ebuild wishing to make use of an eclass does so by using the \t{inherit} command in global scope.
This will cause the eclass to be sourced as part of the ebuild---any function or variable
definitions in the eclass will appear as part of the ebuild, with exceptions for certain metadata
variables, as described below.

The \t{inherit} command takes one or more parameters, which must be the names of eclasses (excluding
the \t{.eclass} suffix and the path). For each parameter, in order, the named eclass is sourced.

Eclasses may end up being sourced multiple times.

The \t{inherit} command must also ensure that:

\begin{compactitem}
\item The \t{ECLASS} variable is set to the name of the current eclass, when sourcing that eclass.
\item Once all inheriting has been done, the \t{INHERITED} metadata variable contains the name of
    every eclass used, separated  by whitespace.
\end{compactitem}

\section{Eclass-defined Metadata Keys}

The \t{IUSE}, \t{DEPEND}, \t{RDEPEND} and \t{PDEPEND} variables are handled specially
when set by an eclass. They must be accumulated across eclasses, appending the value set by each
eclass to the resulting value after the previous one is loaded. Then the eclass-defined value is
appended to that defined by the ebuild. In the case of \t{RDEPEND}, this is done after the
implicit \t{RDEPEND} rules in section~\ref{sec:rdepend-depend} are applied.

\section{EXPORT\_FUNCTIONS}

There is one command available in the eclass environment that is neither available nor meaningful
in ebuilds---\t{EXPORT\_FUNCTIONS}. This can be used to alias ebuild phase functions from the
eclass so that an ebuild inherits a default definition whilst retaining the ability to override and
call the eclass-defined version from it. The use of it is best illustrated by an example; this is
given in listing~\ref{lst:export-functions} and is a snippet from a hypothetical \t{foo.eclass}.

\begin{listing}
  \caption{EXPORT\_FUNCTIONS example: foo.eclass}\label{lst:export-functions}
  \begin{verbatim}
foo_src_compile()
{
    econf --enable-gerbil \
            $(use_enable fnord)
    emake gerbil || die "Couldn't make a gerbil"
    emake || die "emake failed"
}

EXPORT_FUNCTIONS src_compile
  \end{verbatim}
\end{listing}

This example defines an eclass \t{src\_compile} function and uses \t{EXPORT\_FUNCTIONS} to alias
it. Then any ebuild that inherits \t{foo.eclass} will have a default \t{src\_compile} defined, but
should the author wish to override it he can access the function in \t{foo.eclass} by calling
\t{foo\_src\_compile}.

\t{EXPORT\_FUNCTIONS} must only be used on ebuild phase functions. The function that is aliased
must be named \t{eclassname\_phasefunctionname}, where \t{eclassname} is the name of the eclass.

% vim: set filetype=tex fileencoding=utf8 et tw=100 spell spelllang=en :

%%% Local Variables:
%%% mode: latex
%%% TeX-master: "pms"
%%% LaTeX-indent-level: 4
%%% LaTeX-item-indent: 0
%%% TeX-brace-indent-level: 4
%%% End:
